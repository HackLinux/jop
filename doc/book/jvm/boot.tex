\subsection{Booting the JVM}

One interesting issue for an embedded system is how the startup or
boot-up is performed. On power-up, the FPGA starts the configuration
state machine to read the FPGA configuration data either from a Flash
or via a download cable (for development). When the configuration has
finished, an internal reset is generated. After that reset, microcode
instructions are executed starting from address 0. At this stage, we
have not yet loaded any application program (Java bytecode). The
first sequence in microcode performs this task. The Java application
can be loaded from an external Flash or via a serial line (or USB
port) from a PC. The microcode assembly configured the mode.
Consequently, the Java application is loaded into the main memory. To
simplify the startup code we perform the rest of the startup in Java
itself, even when some parts of the JVM are not yet setup.

In the next step, a minimal stack frame is generated and the special
method \code{Startup.boot()} is invoked. From now on JOP runs in
Java mode. The method \code{boot()} performs the following steps:
\begin{samepage}
\begin{enumerate}
    \item Send a greeting message to \emph{stdout}
    \item Detect the size of the main memory
    \item Initialize the data structures for the garbage collector
    \item Initialize \code{java.lang.System}
    \item Print out JOP's version number, detected clock speed, and
    memory size
    \item Invoke the static class initializers in a predefined order
    \item Invoke the \code{main} method of the application class
\end{enumerate}
\end{samepage}

%\subsubsection{CMP Boot}
%
%The boot-up process is the same for all processors until the
%generation of the internal reset and the execution of the first
%microcode instruction. From that point on, we have to take care that
%\emph{only one} processor performs the initialization steps.
%
%All processors in the CMP are functionally identical. Only one
%processor is designated to boot-up and initialize the whole system.
%Therefore, it is necessary to distinguish between the different
%CPUs. We assign a unique CPU identity number (CPU ID) to each
%processor. Only processor CPU0 is designated to do all the boot-up
%and initialization work. The other CPUs have to wait until CPU0
%completes the boot-up and initialization sequence. At the beginning
%of the booting sequence, CPU0 loads the Java application. Meanwhile,
%all other processors are waiting for an \emph{initialization
%finished} signal of CPU0. This busy wait is performed in microcode.
%When the other CPUs are enabled, they will run the same sequence as
%CPU0. Therefore, the initialization steps are guarded by a condition
%on the CPU ID.
