
In this chapter we will undertake a short review of the project and
summarize the contributions. Java for real-time systems is a very
new and active research area. This chapter is completed by
suggestions for future research, based on the proposed Java
processor.

The research contributions made by this work are related to two
areas: real-time Java and resource-constrained embedded systems.

\section{A Real-Time Java Processor}

The goal of time-predictable execution of Java programs was a
first-class guiding principle throughout the development of JOP:

\begin{itemize}

        \item
The execution time for Java bytecodes can be exactly predicted in
terms of the number of clock cycles.
% The execution time for Java bytecodes is known cycle-accurate.
JOP is therefore a straightforward target for low-level WCET
analysis. There is no mutual dependency between consecutive
bytecodes that could result in unbounded timing effects.

    \item
In order to provide time-predictable execution of Java bytecodes,
the processor pipeline is designed without any prefetching or
queuing. This fact avoids hard-to-analyze and possibly unbounded
pipeline dependencies. There are no pipeline stalls, caused by
interrupts or the memory subsystem, to complicate the WCET analysis.

    \item
A pipelined processor architecture calls for higher memory
bandwidth. A standard technique to avoid processing bottlenecks due
to the higher memory bandwidth is caching.
%In order to fill the gap between processor speed and the memory
%access time, caches are mandatory, even in embedded systems.
However, standard cache organizations improve the average execution
time but are difficult to predict for WCET analysis. Two
time-predictable caches are proposed for JOP: a \emph{stack cache}
as a substitution for the data cache and a \emph{method cache} to
cache the instructions.

As the stack is a heavily accessed memory region, the stack -- or
part of it -- is placed in local memory. This part of the stack is
referred to as the \emph{stack cache} and described in
Section~\ref{sec:stack}. Fill and spill of the stack cache is
subjected to microcode control and therefore time-predictable.

In Section~\ref{sec:cache}, a novel way to organize an instruction
cache, as \emph{method cache}, is given. The cache stores complete
methods, and cache misses only occur on method invocation and
return. Cache block replacement depends on the call tree, instead of
instruction addresses. This \emph{method cache} is easy to analyze
with respect to worst-case behavior and still provides substantial
performance gain when compared against a solution without an
instruction cache.

    \item
The above described time-predictable processor provides the basis
for real-time Java. The issues with standard Java and the Real-Time
Specification for Java were analyzed in Chapter~\ref{chap:rtjava}.
To enable real-time Java to operate on resource-constrained devices,
a simple real-time profile was defined in Section~\ref{sec:rtprof}
and implemented in Java on JOP. The beauty of this approach is in
implementing functions usually associated with an RTOS in Java. This
means that real-time Java is not based on an RTOS, and therefore not
restricted to the functionality provided by the RTOS. With JOP, a
self-contained real-time system in pure Java becomes possible.

The tight integration of the scheduler and the hardware that
generates schedule events results in low latency and low jitter of
the task dispatch.

    \item
%The timer interrupt in JOP generates interrupts at the release times
%of the tasks. The scheduler is responsible for reprogramming the
%timer after each occurrence of a timer interrupt. Controlling the
%timer interrupt as part of the scheduling results in the low jitter
%of periodic tasks.
%
The defined real-time profile suggests a new way to handle hardware
interrupts to avoid interference between blocking device drivers and
application tasks. Hardware interrupts other than the timer
interrupt are represented as asynchronous events with an associated
thread. These events are \emph{normal} schedulable objects and
subject to the control of the scheduler. With a minimum interarrival
time, these events, and the associated device drivers, can be
incorporated into the priority assignment and schedulability
analysis in the same way as normal application tasks.

\end{itemize}

The above-described contributions result in a time-predictable
execution environment for real-time applications written in Java,
without the resource implications and unpredictability of a
JIT-compiler. The proposed processor architecture is a
straightforward target for low-level WCET analysis.

%New applications, such as multimedia streaming, result in
%\emph{soft} real-time systems that need a more flexible scheduler
%than the traditional fixed priority-based ones.

Implementing a real-time scheduler in Java opens up new
possibilities. The scheduler is extended to provide a framework for
user-defined scheduling in Java. In Section~\ref{sec:usersched}, we
analyzed which events are exposed to the scheduler and which
functions from the JVM need to be available in the user space. A
simple-to-use framework to evaluate new scheduling concepts is
given.



\section{A Resource-Constrained Processor}

Embedded systems are usually very resource-constrained. Using a
low-cost FPGA as the main target technology forced the design to be
small. The following architectural features address this issue:

\begin{itemize}

    \item
The architecture of JOP is best described as:
\begin{quote}
    The JVM is a CISC stack architecture, whereas JOP is a RISC stack
    architecture.
\end{quote}
JOP contains its own instruction set, called microcode in this
thesis, with a novel way of mapping bytecodes to microcode
addresses. This mapping has zero overheads as described in
Section~\ref{sec:microcode}. Basic bytecode instructions have a
one-to-one mapping to microcode instructions and therefore execute
in a single cycle. The stack architecture allows compact encoding of
microinstructions in 8 bit to save internal memory.

This approach allows flexible implementation of Java bytecodes in
hardware, as a microcode sequence or even in Java itself.

    \item
The analysis of the JVM stack usage pattern in
Section~\ref{sec:stack} led to the design of a resource-efficient
two-level stack cache. This two-level stack cache fits to the
embedded memory technologies of current FPGAs and ASICs and ensures
fast execution of basic instructions.

Part of the stack cache, which is implemented in an on-chip memory,
is also used for microcode variables and constants. This resource
sharing does not only reduce the number of memory blocks needed for
the processor, but also the number of data paths to and from the
execution unit.

    \item
Interrupts are considered hard to handle in a pipelined processor,
resulting in a complex (and therefore resource consuming)
implementation. In JOP, the above mentioned bytecode-microcode
mapping is used in a clever way to avoid interrupt handling in the
core pipeline.
%
%An implementation of interrupts at the bytecode-microcode mapping
%keeps interrupts transparent in the core pipeline and avoids complex
%logic. Therefore,
%
Interrupts generate special bytecodes that are inserted in a
transparent way in the bytecode stream. Interrupt handlers can be
implemented in the same way as bytecodes are implemented: in
microcode or in Java.


\end{itemize}

The above design decisions where chosen to keep the size of the
processor small without sacrificing performance. JOP is the smallest
Java processor available to date that provides the basis for an
implementation of the CLDC specification (see
Section~\ref{subsec:cldc}). JOP is a fast execution environment for
Java, without the resource implications and unpredictability of a
JIT-compiler. The average performance of JOP is similar to that of
mainstream, non real-time Java systems.

JOP is a flexible architecture that allows different configurations
for different application domains. Therefore, size can be traded
against performance. As an example, resource intensive instructions,
such as floating point operations, can be implemented in Java. The
flexibility of an FPGA implementation also allows adding
application-specific hardware accelerators to JOP.

The small size of the processor allows usage of low-cost FPGAs in
embedded systems that can compete against standard microcontroller.
JOP has been implemented in several different FPGA families and is
used in different real-world applications.

Programs for embedded and real-time systems are usually
multi-threaded and a small design provides a path to a
multi-processor system in a mid-sized FPGA or in an ASIC.

A tiny architecture also opens new application fields when
implemented in an ASIC. Smart sensors and actuators, for example,
are very sensitive to cost, which is proportional to the die area.


\section{Future Research Directions}

JOP provides a basis for various directions for future research.
Some suggestions are given below:
%
\begin{description}
    \item[Real-time garbage collector:]
In Section~\ref{chap:rtgc}, a real-time garbage collector was
presented. Hardware support of a real-time GC would be an
interesting topic for further research.

Another question that remains with a real-time GC is the analysis of
the worst-case memory consumptions of tasks (similar to the WCET
values), and scheduling the GC so that it can keep up with the
allocation rate.

    \item[Hardware accelerator:]
The flexibility of an FPGA implementation of a processor opens up
new possibilities for hardware accelerators. We have shown in
Section~\ref{sec:hwsw:co} how the implementation of a bytecode can
be moved between hardware and software. A further step would be to
generate an application specific-system in which part of the
application code is moved to hardware. Ideally, the hardware
description should be extracted automatically from the Java source.
Preliminary work in this area, using JOP as its basis, can be found
in \cite{jop:sac05}.

    \item[Hardware scheduler:]
In JOP, scheduling and dispatch is done in Java (with some microcode
support). For tasks with very short periods, the scheduling
overheads can prove to be too high. A scheduler implemented in
hardware can shorten this time, due to the parallel nature of the
algorithm.

    \item[Multiprocessor JVM:]
In order to generate a small and predictable processor, several
advanced and resource-consuming features (such as instruction
folding or branch prediction) were omitted from the design. The
resulting low resource usage of JOP makes it possible to integrate
more than one processor in an FPGA. Since embedded applications are
naturally multi-threaded systems, the performance can easily be
enhanced using a multi-processor solution. A multi-processor JVM
with shared memory offers following research possibilities:
scheduling of Java threads and synchronization between the
processors; WCET analysis for the shared memory access.

First results on a JOP CMP are described in \cite{jop:dma, jop:cmp}.

    \item[Instruction cache:]
The cache solution proposed in Section~\ref{sec:cache} provides
predictable instruction cache behavior while, in the average case,
still performing in a similar way to a direct-mapped cache. However,
an analysis tool for the worst-case behavior is still needed. With
this tool, and a more complex analysis tool for traditional
instruction caches, we also need to verify that the worst-case miss
penalty is lower than with a traditional instruction cache.

A second interesting aspect of the proposed method cache is the fact
that the replacement decision on a cache miss only occurs on method
invoke and return. The infrequency of this decision means that more
time is available for more advanced replacement algorithms.


    \item[Real-time Java:]
Although there is already a definition for real-time Java, i.e.\ the
RTSJ \cite{rtsj}, this definition is not necessarily adequate. There
is ongoing research on how memory should be managed for real-time
Java applications: scoped memory, as suggested by the RTSJ, usage of
a real-time GC, or application managed memory through memory pools.
However, almost no research has been done into how the Java library
which has proven a major part of Java's success, can be used in
real-time systems or how it can be adapted to do so. The question of
what the best memory management is for the Java standard library
remains unanswered.

    \item[Java computer:]
How would a processor architecture and operating system architecture
look in a `Java only' system? Here, we need to rethink our approach
to processes, protection, kernel- and user-space, and virtual
memory. The standard approach of using memory protection between
different processes is necessary for applications that are
programmed in languages that use memory addresses as data, i.e.\
pointer usage and pointer manipulation. In Java, no memory addresses
are visible and pointer manipulation is not possible. This very
important feature of Java makes Java a \emph{safe} language.
Therefore, an error-free JVM means we do not need memory protection
between processes and we do not need to make a distinction between
kernel and user space (with all the overhead) in a Java system.
Another reason for using virtual addresses is link addresses.
However, in Java this issue does not exist, as all classes are
linked dynamically and the code itself (i.e.\ the bytecodes) only
uses relative addressing.

Another issue here is the paging mechanism in virtual memory system,
which has to be redesigned for a Java computer. For this, we need to
merge the virtual memory management with the GC. It does not make
sense to have a virtual memory manager that works with plain (e.g.\
4~KB) memory pages without knowledge about object lifetime. We
therefore need to incorporate the virtual memory paging with a
generational GC. The GC knows which objects have not been accessed
for a long time and can be swapped out to the disc. Handling paging
as part of the GC process also avoids page fault exceptions and
thereby simplifies the processor architecture.

Another question is whether we can substitute the process notation
with threads, or whether we need several JVMs on a Java only system.
It depends. If we can live with the concept of shared static class
members, we can substitute heavyweight processes with lightweight
threads. It is also possible that we would have to define some
further thread local data structures in the operation system.

\end{description}
%
It is the opinion of the author that Java is a promising language
for future real-time systems. However, a number of issues remain to
be solved. JOP, with its time-predictable execution of Java
bytecodes, is an important but nevertheless only a small part of a
real-time Java system.
