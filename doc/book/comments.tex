\documentclass[a4paper,12pt,twocolumn]{scrartcl}
\usepackage{pslatex} % -- times instead of computer modern

\usepackage[colorlinks=true,linkcolor=black,citecolor=black]{hyperref}
\usepackage{booktabs}
\usepackage{graphicx}


\begin{document}

\title{Comments on the handbook}
\maketitle \thispagestyle{empty}

\subsection*{Paulo 20.12.2007}



By the way, about your book: the diagram for the internal format of
classes inside JOP memory is not up to date: there are old fields
still there and some new fields missing,in the beginning of the
figure about the class format.

And here's another suggestion. I spent one hour or so learning more
about the internal format of the .jop file. Maybe it's a good idea to
insert one diagram showing this structure (if there's already one and
I missed, then sorry;). If I got it right, currently it's like this:
\begin{verbatim}
    - Application size
    - Pointer to "special pointers"
    - Application code (method bodies in sequence, without headers)
    - Table of special pointers
    - Table of pointers for static "class init" methods
    - String table
    - Table for static primitive fields
    - Table for static reference fields (isolated to help during garbage collection)
    - List of class descriptions
\end{verbatim}

Of course it can be depicted in a bit more detail, for some of the
structures above. How about that?


\end{document}
