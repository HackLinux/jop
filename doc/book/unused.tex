

% a changed copy from the thesis

%\documentclass[a4paper,10pt,twocolumn]{article}
% use option draft to check for typesetting problems.
% similar to original word document

% don't use textheight... with koma-script
\documentclass[%draft,
%    a4paper,12pt, % 'Standard'
    11pt, % use explicit paper format for Java book format
%    b5paper,10pt, % B5 printout
    % use BCOR to compensate for the two-side margin!
%    a4paper,12pt,BCOR37pt,DIV15,
%    a4paper,12pt,BCOR40pt,DIV14,
%    b5paper,10pt,DIV12,
    headinclude, footexclude,
    twoside, % this produces strange margins!
    openright, % for new chapters
    cleardoubleempty,
% normalheadings - smaller chapter or smallheadings - too small
    % abstracton, % the author of koma-script argues against the title
    headsepline,
%    5headlines, % standard is 1.25 -- wirkt net!
    pointlessnumbers,
    ]{scrreprt}

%\usepackage{html}

% headings
\usepackage{scrpage2} % for headers
 \setkomafont{pagehead}{\scshape\small}
 \setkomafont{pagenumber}{\scshape\small}
 \automark[section]{chapter}
 \ohead[]{\pagemark}
 \chead[]{}
 \ihead[]{\headmark}
 \ofoot[]{} \cfoot[]{} \ifoot[]{}

%\tolerance=500 % to avoid lines sticking out into the margin
               % needed for 'high-performance' in Intro - contributions
\emergencystretch=2em
% or tol. to 500 and emerg. to 1em?
% pagebreak was ok with 500 and 1em
\interfootnotelinepenalty=10000


% use BCOR = (paperwidth-textwidth)/4
% A4: 210mm x 297mm
% B5: 176mm x 250mm
% Java book: 185mm x 232mm
% Engblom: 120x188 (without head)
% Java: 127x187 (without head)
% 1pt = 1/72.27 in = 0.351 mm


% for book
% 'Java-format' 526pt x 660pt (Ghostscript)
\setlength{\paperwidth}{185mm} \setlength{\paperheight}{232mm}
% use that BCOR setting with twoside to compensate the margin
\areaset[13.75mm]{130mm}{200mm} % Java book format

% for book to A4 conversion:
% Set the following sizes and export with
% 'Variable Page Size' in gs.
% Print with 'fit to paper' in Acrobat - results in 110%
% and an effective text area of 142x215
%\setlength{\paperwidth}{180mm} \setlength{\paperheight}{232mm}
%\areaset[12.5mm]{130mm}{200mm} % Java book format

% for two page printout in .pdf
%\setlength{\paperwidth}{134mm} \setlength{\paperheight}{206mm}
%\areaset[1mm]{130mm}{200mm} % Java book format


% use 10pt for code instead of 11pt - but I still would prefer Lucida Typewriter
%\newfont{\myttfont}{cmss10 scaled 1000}
%\newfont{\myttbfont}{cmssdc10 scaled 1000}
%
% This IS Lucida Typewriter
%\newfont{\myttfont}{plsr8r scaled 950}
%\newfont{\myttbfont}{plsb8r scaled 950}
%\newfont{\myttifont}{plsro8r scaled 950}
%%\newfont{\mytttextfont}{plsr8r}

% Lucida is perhaps available in the new Tex installation!!!!
% does not really work!!!
\newfont{\myttfont}{hlsrt8r scaled 950}
\newfont{\myttbfont}{hlsbt8r scaled 950}
\newfont{\myttifont}{hlsrot8r scaled 950}

% I used these .ttf for the official Thesis
%..\ttf2pt1 -e -b LucidaTypewriterRegular.ttf plsr8a
%..\ttf2pt1 -e -b LucidaTypewriterBold.ttf plsb8a
%..\ttf2pt1 -e -b LucidaTypewriterOblique.ttf plsro8a
%..\ttf2pt1 -e -b LucidaTypewriterBoldOblique.ttf plsbo8a

%\newcommand{\javatt}{\myttfont}
%\newcommand{\javattb}{\myttbfont}
%\newcommand{\javatti}{\myttifont}
%\newcommand{\javatext}{\myttfont}
%
%\newcommand{\picscale}{0.909}
%\newcommand{\excelwidth}{11cm}

% end book

% for B5
%\newfont{\javatt}{cmss10}
%\newfont{\javattb}{cmssdc10}
%\newcommand{\picscale}{0.833}
%\newcommand{\excelwidth}{10cm}



% for A4
% 12pt A4 scaled from book
%\areaset[17.05mm]{142mm}{219mm}
\newfont{\javatt}{cmss12}
\newfont{\javattb}{cmssdc10 scaled 1200}
% TODO find an italic
\newfont{\javatti}{cmss12}
\newcommand{\javatext}{\javatt}

\newcommand{\picscale}{1}
\newcommand{\excelwidth}{12cm}


% for chapter head without a number
% \renewcommand{\chaptermark}[1]{\def\myleftmark{#1}}
% \ihead{\myleftmark} \chead{} \ohead{{\rightmark}}

\setkomafont{captionlabel}{\sffamily\bfseries}



% Do I need this package?
\usepackage{float}

% is this a correction for the <> problem?
% \usepackage[T1]{fontenc}

\usepackage{pslatex} % -- times instead of computer modern
% pslatex should be replaced by this:
%\usepackage{mathptmx}
%\usepackage[scaled=.90]{helvet}
%\usepackage{courier}
% pslatex does not work with T1 encoding. <> Problem?


\usepackage{latexsym}
\usepackage{graphicx}
\usepackage{amsmath}
\usepackage{longtable}
\usepackage{booktabs}

% I would need Lucida Console!!!
%
%\newfont{\javatt}{pltt12} % lucida teletype, better than normal but with serifs
%\newfont{\javatt}{plss12} % lucida no serifes, but variable spacing
%\newfont{\javatt}{plss10 scaled 1200}
%\newfont{\javattb}{plssdc10 scaled 1200}
% cmss is NOT a tt font....


\usepackage{listings}
\lstset{language=Java,keywordstyle=,
basicstyle=\javatt,emphstyle=\javattb,commentstyle=\javatti,
showstringspaces=false,captionpos=b}

\usepackage{array}
\usepackage{dcolumn}
\newcommand{\cc}[1]{\multicolumn{1}{c}{#1}}
\newcolumntype{d}[1]{D{.}{.}{#1}}

% f�r die Umlaute in der Kurzfassung
% bekomme ich dadurch Probleme???
%\usepackage[ansinew]{inputenc}


\usepackage{capt-of}
\usepackage[colorlinks=true,linkcolor=black,citecolor=black]{hyperref}
%\usepackage{hyperref}

% ----------------------

%\usepackage{makeidx}
%\makeindex


\usepackage{import} % for subimport text and graphics from subdirectory
% does not work with latex2html!


\newcommand{\codefoot}{\textsf}
\newcommand{\code}[1]{{\javatext#1}} % for LaTeX
\newcommand{\cmd}[1]{{\texttt{#1}}}
\newcommand{\dirent}[1]{{\texttt{#1}}}
%\newcommand{\menuitem}[1]{\textsf{\textbf{#1}}}
\newcommand{\menuitem}[1]{\textsf{\textsl{#1}}}

% for flow.tex - part of index helper
\newcommand{\eei}[1]{%
\index{extension!\texttt{#1}}\texttt{#1}}

% JVs et al
%\newcommand{\ea}{et al.\xspace}
\newcommand{\ea}{et al.\ }

%\begin{htmlonly}
%\renewcommand{\code}[1]{{\texttt{#1}}} % for html2LaTeX
%\newcommand{\toprule}{\hline}
%\newcommand{\midrule}{\hline}
%\newcommand{\bottomrule}{\hline}
%\end{htmlonly}

% net wirklich notwendig -- h�ngt von code generierung ab
%\begin{htmlonly}
%\renewcommand{\javatt}{\texttt}
%\renewcommand{\javattb}{\texttt\bfseries}
%\end{htmlonly}

%\code{\hyphenchar\font=-1}

\newcommand{\mycomment}[1]{}

\newcommand{\instr}[6]{
    \begin{table}
        \begin{tabular}{ll}
            \emph{\large\textbf{#1}} & \\
            \\ \\
            \textbf{Operation} & #2 \\ \\
            \textbf{Opcode} & \texttt{#3} \\ \\
            \textbf{Dataflow} & \parbox[t]{9.5cm}{\(#4\)}\\ \\
            \textbf{JVM equivalent} & \parbox[t]{9.5cm}{\code{#5}} \\ \\
            \textbf{Description} & \parbox[t]{9.5cm}{#6}\\
        \end{tabular}
    \end{table}
}


\begin{document}




\chapter{Introduction}
\label{chap:intro}


\emph{Copy some stuff from the thesis and rewrite it.}
    

This handbook introduces the concept of a Java processor for
embedded real-time systems, in particular the design of a small
processor for resource-constrained devices with time-predictable
execution of Java programs. This Java processor is called JOP --
which stands for Java Optimized Processor --, based on the
assumption that a full native implementation of all Java bytecode
instructions is not a useful approach.

%\section{Motivation}
\section{Justification for Development}

To justify Java's use in embedded real-time systems we quote from a
document published by the National Institute of Standards and
Technology \cite{nist99}:

\begin{itemize}
    \item Java's higher level of abstraction allows for increased programmer
productivity (although recognizing that the tradeoff is runtime
efficiency)
    \item Java is relatively easier to master than C++
    \item Java is relatively secure, keeping software components (including
the JVM itself) protected from one another
    \item Java supports dynamic loading of new classes
    \item Java is highly dynamic, supporting object and thread creation at
runtime
    \item Java is designed to support component integration and reuse
    \item The Java technologies have been developed with careful
consideration, erring on the conservative side using concepts and
techniques that have been scrutinized by the community
    \item The Java programming language and Java platforms support
application portability
    \item The Java technologies support distributed applications
    \item Java provides well-defined execution semantics
\end{itemize}

Based on the NIST document, the Real-Time for Java Experts Group has
published the Real Time Specification for Java (RTSJ) \cite{rtsj} to
add real-time extensions to Java.

Despite the above, to date Java is rarely used in embedded real-time
systems. High resource requirements for the Java virtual machine and
unpredictable real-time behavior are the main issues surrounding the
use of Java for embedded systems. JOP addresses both issues, and the
proposed Java processor makes a strong case for the use of Java in
embedded systems.

\section{Embedded Real-Time Systems}

An embedded system is a special-purpose computer system that is part
of a larger system or machine. An embedded system is designed to
perform a narrow range of functions with no, or minimal user
intervention.

Since many embedded systems are produced in large quantities, the
need to reduce costs is a major concern. Embedded systems often have
significant energy constraints, and many are battery-powered. As a
result of these constraints, embedded systems use a slow processor
and small memory size to minimize costs and energy consumption.

Embedded systems interact with the environment and often have to
produce output within a given timeframe. Therefore, most embedded
systems are real-time systems. Here is a general definition of a
real-time system (John A. Stankovic \cite{50811}):

\begin{quote}
In real-time computing the correctness of the system depends not
only on the logical result of the computation but also on the time
at which the result is produced.
\end{quote}

%(Donald Gillies \cite{rt:definition}):
%% realtime FAQ - Donald Gillies
%A real-time system is one in which the correctness of the
%computations not only depends upon the logical correctness of the
%computation but also upon the time at which the result is produced.
%If the timing constraints of the system are not met, system failure
%is said to have occurred.
% Donald W. Gillies website
%In a hard real-time system the correctness of a computation depends
%not only on the computed results but also on the time at which they
%are produced. A result produced on or after the deadline is
%typically useless.
%
However, it should be noted that `real-time' does not mean `really
fast'. In pure real-time systems (i.e.\ without non real-time
tasks), there is no additional value in producing results earlier
than required.

Embedded real-time systems often have to handle concurrent tasks,
such as communication, calculating values for a control loop, user
interface and supervision. A natural way to handle these concurrent
jobs is to model them as individual tasks. These tasks are executed
on a preemptive multi-tasking system. Each task is assigned a
priority and the multi-tasking system is responsible for scheduling
individual tasks according to their priority.
%A schedulability test shows that all tasks would meet their deadlines.

%\subsection{Scheduling}

To fulfil the time constraints for a real-time system, an
appropriate schedule needs to be found. This problem was solved in
the classic paper by Liu and Layland \cite{321743} on independent
periodic tasks. The optimal priority assignment for a set of tasks
is called the rate monotonic priority order, in which a task with a
shorter period is assigned a higher priority. If the Worst-Case
Execution Time (WCET) of each task is known, the schedule is
feasible and all tasks will meet their deadline\footnote{The period
of a periodic task is the time between consecutive activations of
the task. The deadline of the task is assumed to be at the end of
the tasks period.}, if:

\begin{samepage}
\begin{equation}
\nonumber
    \frac{C_1}{T_1}+\dots+\frac{C_n}{T_n} \le U(n) = n(2^{\frac{1}{n}}-1)
\end{equation}
%
where
\begin{equation}
\nonumber
    \begin{split}
        C_i & = \mbox{worst-case execution time of } task_i \\
        T_i & = \mbox{period of } task_i \\
        U(n) & = \mbox{utilization bound for $n$ tasks.}
    \end{split}
\end{equation}
\end{samepage}
%
In theory, this test is both elegant and simple. For concrete
systems, two issues have to be solved:
%
\begin{itemize}
    \item There are very few systems in existence that do not require
    communication between tasks.
    As a result, tasks cannot be seen as independent and blocking
    needs to be incorporated into the schedulability analysis.
    \item The WCET of each task has to be known. This is not a
    trivial task. Simple measurements of execution times never fully
    guarantee a correct value. The tasks therefore have to be analyzed
    using  the correct model of the target system. It is almost
    impossible to provide an accurate and correct model of modern
    processors and memory systems.
\end{itemize}
%
Several standard textbooks on real-time systems \cite{
book:klein-real-time-analysis-ratetm, 558498} deal with the first
issue. JOP is intended to resolve the second issue. It should be
noted that there are a number of scheduling approaches and
schedulability tests. However, as a rule, these approaches all
assume that the WCET of each task is known.


\section{Research Objectives and Contributions}


This book presents a hardware implementation of the Java Virtual
Machine (JVM), targeting small embedded systems with real-time
constraints. The processor is designed from the ground up for low
WCET of bytecodes, in order to give tasks low WCET values. The
following list summarizes the research objectives for the proposed
Java processor:
%
\paragraph{Primary Objectives:}
    \begin{itemize}
        \item Time-predictable Java platform for embedded real-time
        systems
        \item Small design that fits into a low-cost FPGA
        \item A working processor, not merely a proposed architecture
    \end{itemize}
\paragraph{Secondary Objectives:}
    \begin{itemize}
        \item Acceptable performance compared with mainstream non
        real-time Java systems
        \item A flexible architecture that allows different
        configurations for different application domains
        \item Definition of a real-time profile for Java
    \end{itemize}

\subsubsection{Contributions:}

JOP is a stack computer with its own instruction set, called
microcode in this book. Java bytecodes are translated into microcode
instructions or sequences of microcode. The difference between the
JVM and JOP is best described as the following:
\begin{quote}
The JVM is a CISC stack architecture, whereas JOP is a RISC stack
architecture.
\end{quote}


JOP will help to increase the acceptance of Java for embedded
real-time systems. JOP is implemented as a soft-core in a Field
Programmable Gate Array (FPGA). Using an FPGA as the processor for
embedded systems is uncommon, because of the high costs, compared
with a microcontroller. However, if the core is small enough, unused
FPGA resources can be used to implement periphery in the FPGA,
resulting in a lower chip count and hence lower overall costs.

The main contributions are as follows:

\begin{itemize}

    \item
The execution time for Java bytecodes can be exactly predicted in
terms of the number of clock cycles.
%The execution time for Java bytecodes is known cycle-accurate.
There is no mutual dependency between consecutive bytecodes.
Therefore, no pipeline analysis -- with possible unbound timing
effects -- is necessary. These properties greatly simplify low-level
WCET analysis.

In order to fill the gap between processor speed and the memory
access time, caches are mandatory. In Section~\ref{sec:cache}, a
novel way to organize an instruction cache, as \emph{method cache},
is provided. This method cache is simple to analyze with respect to
worst-case behavior and still provides a substantial performance
gain when compared against a solution without an instruction cache.

The proposed processor architecture results in a predictable and
high-performance
% fast
execution of real-time tasks in Java, without the resource
implications and unpredictability of a JIT-compiler.

    \item
JOP is microprogrammed using a novel way of mapping bytecodes to
microcode addresses. This mapping has zero overheads, even for
complex bytecodes.

A two-level stack cache, described in Section~\ref{sec:stack}, which
fits to the embedded memory technologies of current FPGAs and ASICs,
ensures the fast execution of basic instructions with minimum
resource requirements. Fill and spill of the stack cache is
subjected to microcode control and therefore time-predictable.

JOP is the smallest hardware implementation of the JVM available to
date. This fact enables low-cost FPGAs to be used in embedded
systems. The resource usage of JOP can be configured to trade size
against performance for different application domains.

    \item
The definition of standard Java does not fit hard real-time
applications. Therefore, a real-time profile for Java (with
restrictions) is defined in Section~\ref{sec:rtprof} and implemented
on JOP. Tight integration of the scheduler and the hardware that
generates schedule events results in low latency and low jitter of
the task dispatch.

In this profile, hardware interrupts are represented as asynchronous
events with associated threads. These events are subject to the
control of the scheduler and can be incorporated into the priority
assignment and schedulability analysis in the same way as normal
application tasks.

    \item
One contribution made as part of this work is the concrete
implementation of the proposed architecture. The author is aware
that it is not usually considered necessary to provide a complete
implementation in a research project. However, it is the opinion of
the author that a simulation-only approach would lead to mistakes or
small glitches. By providing a concrete implementation, we are not
only confronted with the full complexity of real-life processes, but
also with one or more major issues that would often be generously
overlooked in a simulation. In Section~\ref{sec:applications}, the
usage of JOP in a real-world application is described.

\end{itemize}

\section{Outline of the Book}

Chapter~\ref{chap:java} provides background information on the Java
programming language and the execution environment, the Java virtual
machine, for Java applications.

The related work is presented in Chapter~\ref{chap:related}.
Different hardware solutions from both academia and industry for
accelerating Java in embedded systems are analyzed. This chapter
concludes with the research question.

Standard Java is not suitable for the resource-constrained world of
embedded systems. Chapter~\ref{chap:rtjava} gives an overview of the
different restrictions of Java for embedded and real-time systems.

Chapter~\ref{chap:arch} is the main chapter in which the
architecture of JOP is described. The motivation behind different
design decisions is given.

A Java processor alone is not a complete JVM.
Chapter~\ref{chap:runtime} describes the runtime environment on top
of JOP, including the definition of a real-time profile for Java and
a framework for a user-defined scheduler in Java.

In Chapter~\ref{chap:results}, JOP is evaluated with respect to
size, performance and WCET. This is followed by a description of the
first commercial real-world application of JOP.

Finally, in Chapter~\ref{chap:conclusions}, the work undertaken is
reviewed and the major contributions are presented. This chapter
concludes with directions for future research using JOP and
real-time Java.


\subsection{Real-Time Performance}
\label{subsec:rt:perf}

In this section, the implementation of the simple real-time profile
(from Section~\ref{sec:rtprof}) with JOP is compared with the
Reference Implementation (RI) of the RTSJ (see
Section~~\ref{sec:rtsj}) on top of Linux. We use the Linux platform
for the comparison, as it is the only platform for which the RTSJ is
available. The RI is an interpreting implementation of the JVM that
is, however, not optimized for performance. A commercial version of
the RTSJ, JTime by TimeSys, should perform better. However, it was
not possible to get a license of JTime for research purposes. JOP is
implemented in Altera's low-cost Cyclone EP1C6 FPGA, and clocked
with 100MHz. The test results for the RI were obtained on an Intel
Pentium MMX 266MHz, running Linux with two different kernels: a
generic kernel version 2.4.20 and the real-time kernel from TimeSys
\cite{TimeSysLinux}, as recommended for the RI. For each test, 500
measurements were taken. Time was measured using a hardware counter
in JOP and the time stamp counter of the Pentium processor under
Linux.

\subsubsection{Periodic Threads}


Many activities in real-time systems must be performed periodically.
Low release jitter is of major importance for tasks such as control
loops. The test setting is similar to the periodic thread test in
\cite{828497}. A single real-time thread only calls
\code{waitForNextPeriod()} in a loop and records the time between
subsequent calls. A second idle thread, with a lower priority,
merely consumes processing time. This test setting results in two
context switches per period.
\tablename~\ref{tab_results_periodic_jop} shows the average,
standard deviation and extreme values for different period times on
JOP. The same values are shown in
\tablename~\ref{tab_results_periodic_ri} for the RI. Please note
that the values are in $\mu$s for JOP and in ms for the RI.

\begin{table}
    \centering
    \begin{tabular}{rrd{2.0}rr}
        \toprule
        \cc{Period} & \cc{Avg.} & \cc{Std. Dev.} &  \cc{Min.} &   \cc{Max.} \\
        \cc{[$\mu$s]} & \cc{[$\mu$s]} & \cc{[$\mu$s]} & \cc{[$\mu$s]} & \cc{[$\mu$s]} \\
        \midrule
%        50   & 50  & 14 &  36  &  66\\ & pre cache version
        50   & 50  & 13 &  35  &  63\\
        70   & 70  &  0   & 70 & 70\\
        100  & 100  & 0   & 100&  100\\
        500  & 500  & 0   & 500 & 500\\
        1,000 & 1,000&  0   &  1,000 &  1,000\\
        \bottomrule
    \end{tabular}
    \caption{Jitter of periodic threads with JOP}
    \label{tab_results_periodic_jop}
\end{table}



\begin{table}
    \centering
    \begin{tabular}{d{1}d{2}d{2.3}d{3}d{2}}
        \toprule
        \cc{Period} & \cc{Avg.} & \cc{Std. Dev.} &  \cc{Min.} &   \cc{Max.} \\
        \cc{[ms]} & \cc{[ms]} & \cc{[ms]} & \cc{[ms]} & \cc{[ms]} \\
        \midrule
        5  & 4.0    & 7.92  &0.017  &   19.90 \\
        10  & 6.6   & 9.34  &0.019  &   19.94 \\
        20  & 20.0   & 0.015  &   19.87  &   20.14 \\
        35  & 35.0   & 5.001  &   29.75  &   40.25 \\
        50  & 50.0   & 0.018  &   49.95  &   50.06 \\
        100  & 100.0   & 0.002  &   99.94  &   100.1 \\
        \bottomrule
    \end{tabular}
    \caption{Jitter of periodic threads with RI/RTSJ}
    \label{tab_results_periodic_ri}
\end{table}


Using microsecond accurate timer interrupts, programmed by the
scheduler, results in excellent performance of periodic threads in
JOP. No jitter from the scheduler can be seen with a single thread
at periods longer than 70$\mu$s.

The measurement for the RI excludes the first values measured. The
first values are misleading as the RI behaves unpredictably at
\emph{startup}. The RI performs inaccurately at periods below 20ms.
This effect has also been observed in \cite{701668}. Larger periods
that are multiples of 10ms have very low jitter. However, using a
period such as 35ms shows a standard deviation of five ms. A
detailed look into the collected samples only shows values of 30 and
40ms. This implies a timer tick of 10ms in the underlying operating
system. No significant difference is observed when running this test
on the generic Linux kernel and the TimeSys kernel. The commercial
version of the TimeSys Linux kernel should perform better as the
resolution of the timer tick is 1ms and a programmable time can be
used for periodic threads. However, it was not possible to obtain a
license to evaluate the combination of JTime on the commercial Linux
kernel. \tablename~\ref{tab_results_periodic_ri} represents the
measurements on the generic kernel. This comparison shows the
advantage of an adjustable timer interrupt over a fixed timer tick.

\subsubsection{Context Switch}

This test setting consists of two threads. A low priority thread
continuously stores the current time in a shared variable. A high
priority periodic thread measures the time difference between this
value and the time immediately after \code{waitForNextPeriod()}.
\tablename~\ref{tab_results_context} gives the times for the context
switch in processor clock cycles.


\begin{table}
    \centering
    \begin{tabular}{lrd{1}rr}
        \toprule
         & \cc{Avg.} & \cc{Std. Dev.} & \cc{Min.} & \cc{Max.} \\
        \midrule
%        JOP & 2,878 & 7.97 & 2,876 & 2,909 \\
%        JOP & 2,878 & 8 & 2,876 & 2,909 \\ % pre cache version
        JOP & 2,686 & 14 & 2,676 & 2,709 \\
        RI Linux & 4,253 & 1,239 & 3,232 & 19,628 \\
        RI TS Linux & 12,923 & 1,145 & 11,529 & 21,090 \\
        \bottomrule
    \end{tabular}
    \caption{Time for a thread switch in clock cycles}
    \label{tab_results_context}
\end{table}



This test did not produce the expected behavior from the RI on the
generic Linux kernel. When the low priority thread ran in this tight
loop, the high priority thread was not scheduled. After inserting a
\code{Thread.yield()} and an operating system call, such as
\code{System.out.print()}, in this loop, the test performed as
expected. This indicates a major problem in either the RI or the
operating system scheduler. This problem did not occur when the RI
was run on the TimeSys Linux kernel. However, the context switch
time on the TimeSys kernel is three times longer than on the
standard kernel.

\subsubsection{Asynchronous Event Handler}

In this test setting, a high priority event handler is triggered by
a low priority periodic thread. As \code{AsynchEventHandler}
performs poorly in the RI (see \cite{701668}), a
\code{BoundAsynchEventHandler} is used for the RI test program. The
time elapsed between the invocation of \code{fire()} and the first
statement of the event handler was measured.
\tablename~\ref{tab_results_event} shows the elapsed times in clock
cycles for JOP and the RTSJ RI.


\begin{table}
    \centering
    \begin{tabular}{lrd{1}rr}
        \toprule
         & \cc{Avg.} & \cc{Std. Dev.} & \cc{Min.} & \cc{Max.} \\
        \midrule
%        JOP & 2,986 & 7.3 & 2,822 & 2,986 \\
%        JOP & 2,986 & 7 & 2,822 & 2,986 \\ % pre cache version
        JOP & 2,935 & 7 & 2,773 & 2,935 \\
        RI Linux & 53,685 & 7,014 & 47,400 & 87,196 \\
        RI TS Linux & 69,273 & 7,832 & 63,060 & 101,292 \\
        \bottomrule
    \end{tabular}
    \caption{Dispatch latency of event handlers in clock cycles}
    \label{tab_results_event}
\end{table}


The time taken to dispatch an asynchronous event is similar to the
context switch time in JOP. This is to be expected as events are
scheduled and dispatched as threads. The minimum value only occurred
in the first event, all following events having been dispatched in
the maximum time.

In the RI, the dispatch time is about 12 times larger than a context
switch with a significant variation in time. This indicates that the
implementation of \code{fire()} and the communication of the event
to the underlying operating system are not optimal. The time factor
between context switch and event handling on the TimeSys kernel is
lower than on the standard kernel, but is nevertheless still
significant.

\subsubsection{Summary}

In this section, we have compared the RTSJ on top of Linux with the
implementation of a simple real-time profile on top of JOP. The RTSJ
addresses several issues relating to the use of Java for real-time
systems. However, the RTSJ is a specification too large and complex
to be implemented in small embedded systems. We therefore use the
simpler real-time profile for JOP. Tight integration of the
real-time scheduler with the supporting processor results in an
efficient platform for Java in embedded real-time systems. A
performance comparison between this implementation and the RTSJ
showed that a dedicated Java processor without an underlying
operating system is more predictable than trying to adopt a general
purpose OS for real-time systems. Time will show if an
implementation of the RTSJ on a \emph{real} RTOS will outperform the
presented solution.



\end{document}
