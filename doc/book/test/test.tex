\subsection{Test Cases for JOP's GC}

\subsubsection{SimpGC1}

\begin{description}
    \item[Purpose:]
Check if the GC triggers the allocation of memory.
Check if the GC uses all the available heap memory.
    \item[Expected Results:]
The tester should see a message reporting that
the garbage collector was triggered. The tester should see a message reporting
the memory size that should match the heap memory size reported at initialization of the
GC.
    \item[Description:]
The application consists of just one thread which continuously
allocates memory for new objects in the heap. The objects contain
only two integer fields and a constructor that initializes them. The
references to objects are not stored in any variable. The objective
is to have an idea of how many objects of a certain class are
necessary to fill the heap. This measure should be useful for
inducing GC allocation in the next test cases.
    \item[Results Observed:]
9728 objects were created before GC was triggered. GC seems to free
the same amount of memory every time it runs (it always triggers
after the allocation of 9728 objects).
\end{description}

\subsubsection{SimpGC2}

\begin{description}
    \item[Purpose:]
Check if the GC triggers the allocation of memory. Check if
unreferenced local (method scope) objects are garbage collected.
    \item[Expected Results:]
The tester should a set of messages reporting that the garbage
collector was triggered.
    \item[Description:]
Similar to the previous test case. The difference here is that the
allocation of memory is performed through several method calls.
After each method call the objects created are unreferenced. After
GC is activated it should be possible to create the same number of
objects as before allocating any object.
    \item[Results Observed:]
29190 objects were created, 9730 objects before GC triggers, each
time.

\emph{Run out of handles in new Object!}

\emph{GC allocation triggered}

Thus, all the unreferenced objects local to the method were garbage
collected each time GC run.

\end{description}

\subsubsection{SimpGC3}

\begin{description}
    \item[Purpose:]
Test concurrent request of memory from different threads (now using
inheritance instead of interfaces).
    \item[Expected Results:]
The tester should see no exceptions and but several GC messages as
explained below.
    \item[Description:]
We have two threads each creating a linked list. Both threads
add(concurrently) a certain amount of elements to the linked list.
Each element in the linked list has a field with a number. Each
thread keeps an increasing order (with no duplication) of those
fields.  GC is triggered. The Linked lists are checked to see if
they remain correct. We alternate among creation and verification of
the elements in the list, so eventually we run out of memory and GC
is triggered.
    \item[Results Observed:]
No exceptions were thrown during execution, thus object integrity is
fine.
\end{description}

\subsubsection{SimpGC4}

\begin{description}
    \item[Purpose:]
Check the behaviour of the GC when several references to an object
are present.
    \item[Expected Results:]
The tester should see no exceptions.
    \item[Description:]
An object is created and then a reference to the object is "taken"
by one thread each time. A reference to the object is always present
but stored in different places each time. At first the reference is
stored in a static field, then it is stored in a thread's class
variable. The static variable is set to null when a thread acquires
the object reference. In the same way when a thread releases the
object it sets its reference to null. Every time that the reference
is changed, GC is triggered and the object integrity is verified.
    \item[Results Observed:]
GC seems to work fine with this test case. Anyway I think that a
finer granularity in the interaction of threads could make a better
scenario for obtaining a better test (I believe that granularity
here is fixed by the length of the run() method of each runnable
thread).
\end{description}

\subsubsection{SimpGC5}

\begin{description}
    \item[Purpose:]
Check if the GC correctly detects garbage objects.
    \item[Expected Results:]
The tester should see a message verifying that non-garbage objects
are alive.
    \item[Description:]
A data structure similar to a tree is created then references to
certain levels of the tree are stored in class variables. GC is
triggered. The code checks that the objects below the level of the
node for which a reference is held are still alive(and correct).
    \item[Results Observed:]
No issues arose when the test case was run.

Note: It's easy to run out of memory when initializing the tree
(i.e. making the tree "grow" ). I used a binary tree with a depth of
8.

\end{description}

\subsubsection{SimpGC6}

\begin{description}
    \item[Purpose:]
Check if the GC detects cycles.
    \item[Expected Results:]
The tester should verify that the same amount of memory is available
before and after the creation of a set of chained objects.
    \item[Description:]
A cycle is formed by a closed chain of objects, one holding a
reference to the next. To verify that the unreferenced chained
objects are garbage collected, we measure the maximum allocable
object size before and after the allocation of the chained objects.
They should match.
    \item[Results Observed:]
The amount of objects created before GC allocation is triggered was
measured before and after the creation of a set of chained objects.
Through a log file it was verified that the number was the same. For
the Cyclone board and JOP version 01/08/07, the number was 9723.
Note: To obtain a logfile, make japp > myLogfile.
\end{description}


\subsubsection{Ifacmp}
\begin{description}
   \item[Purpose:]
Test if\_acmp
   \item[Description:]
The test case works by testing both an equal and a non
equal(ifacmpeq = 165 and 0xa5 ifacmpne) with true and false
conditions(like a branch and condition coverage test). This leads to
4 if branches.

Following, we have an extract of the .class :

36:aload\_2 \newline 37:aload\_3 \newline 38:if\_acmpne       43 $
\longleftarrow $ \newline 41:iconst\_0\newline 42:istore\_1\newline
43:aload\_3 \newline 44:astore\_2\newline 45:aload\_2\newline
46:aload\_3        \newline 47:if\_acmpeq       52 $ \longleftarrow
$ \newline 50:iconst\_0        \newline 51:istore\_1        \newline
52:iload\_1         \newline 53:ireturn         \newline

We see there that both eq and ne conditions are produced.
   \item[Results Observed:]
The test run Ok.
\end{description}


\subsubsection{BranchTest1}
\begin{description}
   \item[Purpose:]
Test Branchs on reference comparison.
   \item[Description:]
The test case works by testing the following bytecodes:
\newline If\_acmp\textless cond\textgreater, Ifnonnull, Ifnull. All
conditions are provided(both true and false) to every variant of the
bytecodes (i.e., eq or neq). In this way, every possibile execution
is covered.
  \item[Results Observed:]
The test run Ok.
\end{description}

\subsubsection{BranchTest2}
\begin{description}
   \item[Purpose:]
Test Branch on int comparison.
   \item[Description:]
The test case works by testing the following bytecode: \newline
If\_icmp\textless cond\textgreater. All conditions are provided(both
true and false) to every variant of the bytecode (i.e., eq, neq, gt,
lt, etc). In this way, every possibile execution is covered.
  \item[Results Observed:]
The test run Ok.
\end{description}


\subsubsection{BranchTest3}
\begin{description}
   \item[Purpose:]
Test Branch on int comparison with zero(if\textless
cond\textgreater).
   \item[Description:]
The test case works by testing the following bytecode: \newline
If\textless cond\textgreater. All conditions are provided(both true
and false) to every variant of the bytecode (i.e., eq, neq, gt, lt,
etc). In this way, every possibile execution is covered.
  \item[Results Observed:]
The test run Ok.
\end{description}




\subsubsection{ArrayTest2}
\begin{description}
   \item[Purpose:]
Test arrays of references. Test array initialization, length,
storing and loading reference values to and from arrays of
references. Test array related features of the following bytecodes:
arraylength, aastore.
   \item[Description:]
A set of arrays of related types is created, then operations are
performed on them. By storing elements in the arrays, all cases of
aastore are produced. As a means of verification, the elements
stored in the array are checked in the code.

   \item[Results Observed:]
The test run Ok. Issue related to CheckCast.
\end{description}

\subsubsection{ArrayTest3}
\begin{description}
   \item[Purpose:]
Test arrays of primitives. Test array initialization, length,
storing and loading reference values to and from arrays of
references. Test array related features of the following bytecodes:
baload, caload, iaload, laload, saload,
bastore,castore,iastore,lastore.
   \item[Description:]
A set of arrays of primitive types is created, then load and store
operations are performed on them. By storing and loading elements
into and from the arrays, all supported Xastore and Xaload
instruction are exercised. As a means of verification, the elements
stored in the array are checked in the code.
  \item[Note:]
The throw of NullPointerException still needs to be tested.
   \item[Results Observed:]
The test run Ok.
\end{description}

\subsubsection{Conversion}
\begin{description}
   \item[Purpose:]
Test primitive type conversion. Bytecodes exercised: i2b, i2c, i2l,
i2s, l2i.
   \item[Description:]
This testcase extends the existent Conversion testcase to include
conversion between all the supported primitive types.
   \item[Results Observed:]
The test run Ok.
\end{description}


\subsubsection{Logic1}
\begin{description}
   \item[Purpose:]
Test shift operators. Bytecodes exercised: ishl, ishr, iushr.
   \item[Description:]
   \item[Results Observed:]
The test run Ok.
\end{description}


\subsubsection{Logic2}
\begin{description}
   \item[Purpose:]
Test arithmetic and logic operations. Bytecodes exercised: ineg,
ior, irem, isub, ixor, iand.
   \item[Description:]
Basic bit levels operations are performed.
   \item[Results Observed:]
The test run Ok. The throw of the proper exceptions was manually
verified.
\end{description}


\subsubsection{Logic3}
\begin{description}
   \item[Purpose:]
Test arithmetic and logic operations.
   \item[Description:]
Jump based on int comparison is tested.
   \item[Results Observed:]
Issue arrised: Some comparisons failed due to a failure handling the
overflow.
\end{description}


\subsubsection{Switch2}
\begin{description}
   \item[Purpose:]
Test switch statement. Bytecodes exercised: tableswitch,
lookupswitch.
   \item[Description:]
Two switch expression are used to test switch related bytecodes.
   \item[Results Observed:]
The test run Ok.
\end{description}

\subsubsection{JDK Test Cases}
\begin{description}
    \item[Note on name convention]

: to avoid collision among the name of the test case and the name of
the class being tested, we just use the first capital letter of the
first word of the class under test. So the test case for
ByteArrayInputStream is named BArrayInputStream.
    \item[Package: java.io]

    \item[Missing Classes]
The following is a list of missing classes as for the Jdk 1.1.8
version.
Package java.io\\
BufferedInputStream, BufferedOutputStream, BufferedReader, BufferedWriter.\\
\large \bf \begin{center} Package java.io \end{center}

\end{description}

\subsubsection{BArrayInputStream}
\begin{description}
   \item[Purpose:]
Test the ByteArrayInputStream class of JDK.
    \item[Description:]
Two objects of the ByteArrayInputStream class are created and
initialized. All the method are excercised under multiple conditions
and verified.
    \item[Results Observed:]
    The test run ok.
    \item[Coverage:]
    method: 78\%(7/9), block: 86\%(122/142), line: 80\%(23,9/30).
    \item[Note:]
The behaviour of the mark(int) method is different from that in the
JDK 1.1.8 specification. The API specification states that the
method should update the current mark in the stream to the provided
parameter, but the result observed is that the mark is set to the
current position.

\end{description}

\subsubsection{BArrayOutputStream}
\begin{description}
   \item[Purpose:]
Test the ByteArrayInputStream class of JDK.
    \item[Description:]
Two objects of the ByteArrayInputStream class are created and
initialized. All the method are excercised under multiple conditions
and verified to produce the proper output.
    \item[Results Observed:]
    The test run ok. \\ The following methods were not found:\\
    \textbullet toString(String)\\
    \textbullet writeTo(OutputStream)
    It was necessary to remove the try-catch for UnsupportedEncodingException in the
    String toString() method in ByteArrayInputStream in order for the testcase to compile.
    \item[Coverage:]
    method: 100\%(9/9), block: 75\%(93/124), line: 81\%(25/31).
    \item[Note:]
Accordding to JDK 1.1.8 specification, the toString(int) method is
deprecated. This method was
not tested.\\
Accordding to JDK 1.1.8 specification, the constructor String(byte
bytes[],int offset,int length) in class String should not throw the
UnsupportedEncodingException exception.


\end{description}



\subsubsection{DInputStream}
\begin{description}
   \item[Purpose:]
Test the DataInputStream class of JDK.
    \item[Description:]
An object of the myInputStream class is created. The myInputStream
class extends the InputStream class and is used to simulate an
InputStream. Methods in the myInputStream
class are included to produce necessary testing conditions as follows:\\
\textbullet setError() : simulates an I/O error
and causes an exception to be thrown when trying to read from the Stream.\\
\textbullet SetCorrect() : disables the effect of the previous method.\\
\textbullet setEndReached(): simulates an end of
stream. The call to this methods results in a -1 returned from the read() method.\\
\textbullet SetEndNotReached() : disables the effect of the previous method.\\

All the method are excercised under multiple conditions and verified
to produce the proper output.
    \item[Results Observed:]
    The test run ok.
    \item[Coverage:]
    method: 96\%(25/26), block: 69\%(345/498), line: 72\%(65/90).
    \item[Note:]
    Comment and uncomment the methods above to manually test the Exception behaviour.
\end{description}

\subsubsection{DOutputStream}
\begin{description}
   \item[Purpose:]
Test the DataOutputStream class of JDK.
    \item[Description:]
An object of the myOutputStream class is created. The myOutputStream
class extends the OutputStream class and is used to simulate an
OutputStream. Methods in the myOutputStream
class are included to produce necessary testing conditions as follows:\\
\textbullet setError() : simulates an I/O error
and causes an exception to be thrown when trying to write to the Stream.\\
\textbullet SetCorrect() : disables the effect of the previous method.\\
The following methods are used to verify contents written to the OutputStream:\\
\textbullet checkData(byte data,int position): returns a boolean value of true\\
if the provided data is found at that position in the stream.\\
\textbullet checkData(byte data[],int offset, int len): returns true if the data\\
of the provided array is found in the OutputStream starting at the
provided offset. All the method are excercised under several
conditions and verified to produce the proper output.
    \item[Results Observed:]
    The test run ok. \\ The following methods were not found:\\
    \textbullet int size()\\
    \item[Coverage:]
    method: 93\%(13/14), block: 70\%(284/406), line: 78\%(62/80).
    \item[Note:]
Comment and uncomment the methods above to manually test the
Exception behaviour.

\large \bf \begin{center} Package java.lang \end{center}
\end{description}

\subsubsection{PrimitiveClasses}
\begin{description}
   \item[Purpose:]
Test the primitive types wrapper classes of JDK.
    \item[Description:]
We create objects of the primitive wrapper classes and perform operations on them.\\
Class methods are excercised under several conditions and verified
to produce the proper output.
    \item[Results Observed:]
    Possible issue in constructor Boolean(String): according to JVM specification
    1.1.8 when the argument to the method is something different from the string "true"
    a Boolean representing false is allocated.\\
    The problem arises when the argument is the null pointer. An exception is thrown in
    JOP in that case. According to the specification, this constructor should not throw
    any exception, it just must create a Boolean representing false.\\
    When running on Sun's JVM the constructor creates the false Boolean as expected.\\
    The constructor also fails to create a proper true object when given a string as an argument.
    It should ignore case, but "tRue" gives a different result from "true", when they should both
    produce a Boolean true object.\\
    Also the static method valueOf should receive a string instead of a boolean as its argument.\\
    The constructor Byte(String) was not found.
    \item[Coverage:]
    Found in table 1.
    \item[Note:]
The method Boolean.valueOf(boolean) was not tested.
\end{description}

\begin{table}
\begin{center}
\caption{\label{MyTable} Testing Coverage Results}
\begin{tabular}[t]{|l|l|l|l|l|}
\hline
Package & Class & Method & Block & Line \\
\hline
java.lang &  &  &  &  \\
\hline
  & Boolean & 88\% (7/8) & 91\% (64/70)  & 93\% (13/14) \\
\hline
  & Byte & 88\% (7/8) & 91\% (64/70)  & 93\% (13/14) \\
\hline
\end{tabular}
\end{center}
\end{table}
