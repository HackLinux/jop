% just for now for the printout with old title page
\setcounter{page}{1}

\begin{flushleft}
\pagestyle{empty}
\ \\
\vspace{1cm}
{\usekomafont{title}\mdseries\Large JOP Reference Handbook\\
\medskip
\large Building Embedded Systems with a Java Processor}
\cleardoublepage
\end{flushleft}


\begin{flushleft}
\pagestyle{empty}
\ \\
\vspace{1cm}
{\usekomafont{title}\Huge JOP Reference Handbook\\
\mdseries
{\Large Building Embedded Systems with a Java Processor}\\
\bigskip
{\large\itshape Beta Edition}\\
\bigskip
{\usekomafont{title}\huge Martin Schoeberl}
\medskip\\
{\large\itshape martin@jopdesign.com} }


\vspace{10cm} \emph{Version: \today}
\newpage
\end{flushleft}




\thispagestyle{empty}
\begin{flushleft}
{\small

Copyright \copyright \ 2007 Martin Schoeberl
\medskip

Martin Schoeberl\\
Strausseng. 2-10/2/55\\
A-1050 Vienna, Austria\\
\medskip

Email: \url{martin@jopdesign.com}\\
Visit the accompanying web site on \url{http://www.jopdesign.com/}
and the JOP Wiki at \url{http://www.jopwiki.com/}
\medskip

%Published 2007 by Virtualbookworm.com Publishing Inc.,\\
%P.O. Box 9949, College Station, TX 77842, US.
Published 2008 by CreateSpace,\\
\url{http://www.createspace.com/}



\medskip

Published 2007, Beta edition 2008 (preprint)
\medskip

All rights reserved. No part of this publication may be reproduced,
stored in a retrieval system, or transmitted in any form or by any
means, electronic, mechanical, recording or otherwise, without the
prior written permission of Martin Schoeberl.
\medskip

%``JOP Reference Handbook" by Martin Schoeberl. .

\textbf{Library of Congress Cataloging-in-Publication Data}
\medskip

Schoeberl, Martin
\begin{quote}
    JOP Reference Handbook : Building Embedded Systems\\
    with a Java Processor / Martin Schoeberl\\
    Includes bibliographical references and index.\\
    ISBN 1438239696
%    ISBN 978-1-60264-XXX
\end{quote}

\bigskip


Manufactured in the United States of America.

Typeset in 11pt Times by Martin Schoeberl }
\end{flushleft}


\addchap{Foreword}

This book is about JOP, the Java Optimized Processor. JOP began as a
research project for a PhD thesis. JOP has been used in several
industrial applications and, due to the fact that it is an
open-source project, has a growing user base. This book is written
for all of you who build this lively community. The book is based to
some extent on the PhD thesis. For a long time the thesis, some
research papers, and the web site  have been the only available
documentation for JOP. A thesis is quite different from a reference
manual. Its focus is on research results and implementation details
are usually omitted. This book fills the gap and provides insight
into the implementation of JOP and the accompanying Java virtual
machine (JVM). It also gives you an idea how to build an embedded
real-time system based on JOP.
