

This handbook introduces the concept of a Java processor for
embedded real-time systems, in particular the design of a small
processor for resource-constrained devices with time-predictable
execution of Java programs. This Java processor is called JOP --
which stands for Java Optimized Processor --, based on the
assumption that a full native implementation of all Java bytecode
instructions is not a useful approach.

%\section{Motivation}
\section{Justification for Development}

To justify Java's use in embedded real-time systems we quote from a
document published by the National Institute of Standards and
Technology \cite{nist99}:

\begin{itemize}
    \item Java's higher level of abstraction allows for increased programmer
productivity (although recognizing that the tradeoff is runtime
efficiency)
    \item Java is relatively easier to master than C++
    \item Java is relatively secure, keeping software components (including
the JVM itself) protected from one another
    \item Java supports dynamic loading of new classes
    \item Java is highly dynamic, supporting object and thread creation at
runtime
    \item Java is designed to support component integration and reuse
    \item The Java technologies have been developed with careful
consideration, erring on the conservative side using concepts and
techniques that have been scrutinized by the community
    \item The Java programming language and Java platforms support
application portability
    \item The Java technologies support distributed applications
    \item Java provides well-defined execution semantics
\end{itemize}

Based on the NIST document, the Real-Time for Java Experts Group has
published the Real Time Specification for Java (RTSJ) \cite{rtsj} to
add real-time extensions to Java.

Despite the above, to date Java is rarely used in embedded real-time
systems. High resource requirements for the Java virtual machine and
unpredictable real-time behavior are the main issues surrounding the
use of Java for embedded systems. JOP addresses both issues, and the
proposed Java processor makes a strong case for the use of Java in
embedded systems.

\section{Embedded Real-Time Systems}

An embedded system is a special-purpose computer system that is part
of a larger system or machine. An embedded system is designed to
perform a narrow range of functions with no, or minimal user
intervention.

Since many embedded systems are produced in large quantities, the
need to reduce costs is a major concern. Embedded systems often have
significant energy constraints, and many are battery-powered. As a
result of these constraints, embedded systems use a slow processor
and small memory size to minimize costs and energy consumption.

Embedded systems interact with the environment and often have to
produce output within a given timeframe. Therefore, most embedded
systems are real-time systems. Here is a general definition of a
real-time system (John A. Stankovic \cite{50811}):

\begin{quote}
In real-time computing the correctness of the system depends not
only on the logical result of the computation but also on the time
at which the result is produced.
\end{quote}

%(Donald Gillies \cite{rt:definition}):
%% realtime FAQ - Donald Gillies
%A real-time system is one in which the correctness of the
%computations not only depends upon the logical correctness of the
%computation but also upon the time at which the result is produced.
%If the timing constraints of the system are not met, system failure
%is said to have occurred.
% Donald W. Gillies website
%In a hard real-time system the correctness of a computation depends
%not only on the computed results but also on the time at which they
%are produced. A result produced on or after the deadline is
%typically useless.
%
However, it should be noted that `real-time' does not mean `really
fast'. In pure real-time systems (i.e.\ without non real-time
tasks), there is no additional value in producing results earlier
than required.

Embedded real-time systems often have to handle concurrent tasks,
such as communication, calculating values for a control loop, user
interface and supervision. A natural way to handle these concurrent
jobs is to model them as individual tasks. These tasks are executed
on a preemptive multi-tasking system. Each task is assigned a
priority and the multi-tasking system is responsible for scheduling
individual tasks according to their priority.
%A schedulability test shows that all tasks would meet their deadlines.

%\subsection{Scheduling}

To fulfil the time constraints for a real-time system, an
appropriate schedule needs to be found. This problem was solved in
the classic paper by Liu and Layland \cite{321743} on independent
periodic tasks. The optimal priority assignment for a set of tasks
is called the rate monotonic priority order, in which a task with a
shorter period is assigned a higher priority. If the Worst-Case
Execution Time (WCET) of each task is known, the schedule is
feasible and all tasks will meet their deadline\footnote{The period
of a periodic task is the time between consecutive activations of
the task. The deadline of the task is assumed to be at the end of
the tasks period.}, if:

\begin{samepage}
\begin{equation}
\nonumber
    \frac{C_1}{T_1}+\dots+\frac{C_n}{T_n} \le U(n) = n(2^{\frac{1}{n}}-1)
\end{equation}
%
where
\begin{equation}
\nonumber
    \begin{split}
        C_i & = \mbox{worst-case execution time of } task_i \\
        T_i & = \mbox{period of } task_i \\
        U(n) & = \mbox{utilization bound for $n$ tasks.}
    \end{split}
\end{equation}
\end{samepage}
%
In theory, this test is both elegant and simple. For concrete
systems, two issues have to be solved:
%
\begin{itemize}
    \item There are very few systems in existence that do not require
    communication between tasks.
    As a result, tasks cannot be seen as independent and blocking
    needs to be incorporated into the schedulability analysis.
    \item The WCET of each task has to be known. This is not a
    trivial task. Simple measurements of execution times never fully
    guarantee a correct value. The tasks therefore have to be analyzed
    using  the correct model of the target system. It is almost
    impossible to provide an accurate and correct model of modern
    processors and memory systems.
\end{itemize}
%
Several standard textbooks on real-time systems \cite{
book:klein-real-time-analysis-ratetm, 558498} deal with the first
issue. JOP is intended to resolve the second issue. It should be
noted that there are a number of scheduling approaches and
schedulability tests. However, as a rule, these approaches all
assume that the WCET of each task is known.


\section{Research Objectives and Contributions}


This book presents a hardware implementation of the Java Virtual
Machine (JVM), targeting small embedded systems with real-time
constraints. The processor is designed from the ground up for low
WCET of bytecodes, in order to give tasks low WCET values. The
following list summarizes the research objectives for the proposed
Java processor:
%
\paragraph{Primary Objectives:}
    \begin{itemize}
        \item Time-predictable Java platform for embedded real-time
        systems
        \item Small design that fits into a low-cost FPGA
        \item A working processor, not merely a proposed architecture
    \end{itemize}
\paragraph{Secondary Objectives:}
    \begin{itemize}
        \item Acceptable performance compared with mainstream non
        real-time Java systems
        \item A flexible architecture that allows different
        configurations for different application domains
        \item Definition of a real-time profile for Java
    \end{itemize}

\subsubsection{Contributions:}

JOP is a stack computer with its own instruction set, called
microcode in this book. Java bytecodes are translated into microcode
instructions or sequences of microcode. The difference between the
JVM and JOP is best described as the following:
\begin{quote}
The JVM is a CISC stack architecture, whereas JOP is a RISC stack
architecture.
\end{quote}


JOP will help to increase the acceptance of Java for embedded
real-time systems. JOP is implemented as a soft-core in a Field
Programmable Gate Array (FPGA). Using an FPGA as the processor for
embedded systems is uncommon, because of the high costs, compared
with a microcontroller. However, if the core is small enough, unused
FPGA resources can be used to implement periphery in the FPGA,
resulting in a lower chip count and hence lower overall costs.

The main contributions are as follows:

\begin{itemize}

    \item
The execution time for Java bytecodes can be exactly predicted in
terms of the number of clock cycles.
%The execution time for Java bytecodes is known cycle-accurate.
There is no mutual dependency between consecutive bytecodes.
Therefore, no pipeline analysis -- with possible unbound timing
effects -- is necessary. These properties greatly simplify low-level
WCET analysis.

In order to fill the gap between processor speed and the memory
access time, caches are mandatory. In Section~\ref{sec:cache}, a
novel way to organize an instruction cache, as \emph{method cache},
is provided. This method cache is simple to analyze with respect to
worst-case behavior and still provides a substantial performance
gain when compared against a solution without an instruction cache.

The proposed processor architecture results in a predictable and
high-performance
% fast
execution of real-time tasks in Java, without the resource
implications and unpredictability of a JIT-compiler.

    \item
JOP is microprogrammed using a novel way of mapping bytecodes to
microcode addresses. This mapping has zero overheads, even for
complex bytecodes.

A two-level stack cache, described in Section~\ref{sec:stack}, which
fits to the embedded memory technologies of current FPGAs and ASICs,
ensures the fast execution of basic instructions with minimum
resource requirements. Fill and spill of the stack cache is
subjected to microcode control and therefore time-predictable.

JOP is the smallest hardware implementation of the JVM available to
date. This fact enables low-cost FPGAs to be used in embedded
systems. The resource usage of JOP can be configured to trade size
against performance for different application domains.

    \item
The definition of standard Java does not fit hard real-time
applications. Therefore, a real-time profile for Java (with
restrictions) is defined in Section~\ref{sec:rtprof} and implemented
on JOP. Tight integration of the scheduler and the hardware that
generates schedule events results in low latency and low jitter of
the task dispatch.

In this profile, hardware interrupts are represented as asynchronous
events with associated threads. These events are subject to the
control of the scheduler and can be incorporated into the priority
assignment and schedulability analysis in the same way as normal
application tasks.

    \item
One contribution made as part of this work is the concrete
implementation of the proposed architecture. The author is aware
that it is not usually considered necessary to provide a complete
implementation in a research project. However, it is the opinion of
the author that a simulation-only approach would lead to mistakes or
small glitches. By providing a concrete implementation, we are not
only confronted with the full complexity of real-life processes, but
also with one or more major issues that would often be generously
overlooked in a simulation. In Section~\ref{sec:applications}, the
usage of JOP in a real-world application is described.

\end{itemize}

\section{Outline of the Book}

Chapter~\ref{chap:java} provides background information on the Java
programming language and the execution environment, the Java virtual
machine, for Java applications.

The related work is presented in Chapter~\ref{chap:related}.
Different hardware solutions from both academia and industry for
accelerating Java in embedded systems are analyzed. This chapter
concludes with the research question.

Standard Java is not suitable for the resource-constrained world of
embedded systems. Chapter~\ref{chap:rtjava} gives an overview of the
different restrictions of Java for embedded and real-time systems.

Chapter~\ref{chap:arch} is the main chapter in which the
architecture of JOP is described. The motivation behind different
design decisions is given.

A Java processor alone is not a complete JVM.
Chapter~\ref{chap:runtime} describes the runtime environment on top
of JOP, including the definition of a real-time profile for Java and
a framework for a user-defined scheduler in Java.

In Chapter~\ref{chap:results}, JOP is evaluated with respect to
size, performance and WCET. This is followed by a description of the
first commercial real-world application of JOP.

Finally, in Chapter~\ref{chap:conclusions}, the work undertaken is
reviewed and the major contributions are presented. This chapter
concludes with directions for future research using JOP and
real-time Java.
