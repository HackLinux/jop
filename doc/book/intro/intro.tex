

This handbook introduces a Java processor for embedded real-time
systems, in particular the design of a small processor for
resource-constrained devices with time-predictable execution of Java
programs. This Java processor is called JOP -- which stands for Java
Optimized Processor --, based on the assumption that a full native
implementation of all Java bytecode instructions is not a useful
approach.

\section{A Quick Tour on JOP}

In the following section we will give a quick overview on JOP and a
short description how to get JOP running within an FPGA. A detailed
description of the build process can be found in
Chapter~\ref{chap:build}. JOP is a soft-core written in VHDL plus
tools in Java, a simplified Java library (JDK), and application
examples. JOP is delivered in source only.

\subsection{Building JOP and Running ``Hello World"}

To build JOP you first have to download the source tree. A
\emph{Makefile} (or an Ant file) contains all necessary steps to
build the tools, the processor, and the application. Configuration of
the FPGA and downloading the Java application is also part of the
Makefile.

In this description we assume the FPGA board Cycore (see
Appendix~\ref{appx:cycore}). This board is the default target for
the Makefile. The board has to be connected to the power supply and
to the PC via a ByteBlaster download cable and a serial cable.

The FPGA is configured via the ByteBlaster cable. The Java
application is downloaded after the FPGA configuration via the serial
cable. Besides the download the serial cable is also used as a
communication link between JOP and the PC. \code{System.out} and
\code{System.in} represent this serial link on JOP.

In order to build the whole system you need a Java
compiler\footnote{Download the Java SE Development Kit (JDK) from
\url{http://java.sun.com/javase/downloads/index.jsp}.} and an FPGA
compiler. In our case we use the free web edition of Quartus from
Altera.\footnote{\url{http://www.altera.com/}} As we use \cmd{make}
and the preprocessor from the GNU compiler collection,
Cygwin\footnote{\url{http://www.cygwin.com/}} should be installed
under Windows.

When all tools are setup correctly\footnote{Check at the command
prompt that \cmd{javac} is in the path.} a simple \cmd{make} should
build the tools, the processor, compile the ``Hello World" example,
configure the FPGA and download the application. The whole build
process will take a few minutes. After typing
\begin{lstlisting}
    make
\end{lstlisting}
you see a lot of messages from the various tools. However, the last
lines should be actual messages received from JOP. It should look
similar to the following:
\begin{lstlisting}
    JOP start V 20080821
    60 MHz, 1024 KB RAM, 1 CPUs
    Hello World from JOP!

    JVM exit!
\end{lstlisting}
Note that JOP prints some internal information, such as version and
memory size, at startup. After that, the message ``Hello World from
JOP!" can be seen. Our first program runs on JOP!

As a next step, locate the Hello World example in the source
tree\footnote{\dirent{.../jop/java/target/src/test/test/HelloWorld.java}}
and change the output message. The tools and the processor have been
built already. So we do not need to compile everything from scratch.
Use the following make target to just compile the Java application
and download the processor and the application:
\begin{lstlisting}
    make japp
\end{lstlisting}
The compile process should now be faster and the output similar to
before.

The Hello World application is the default target in the Makefile.
See Chapter~\ref{chap:build} for a description how this target can be
changed. In case you use a different FPGA board you can find
information on how to change the build process also in
Chapter~\ref{chap:build}.

\subsection{The Design Structure}

Browsing the source tree of JOP can give the impression that the
design is complex. However, the basic structure is not that complex.
The design consists of three entities:
\begin{enumerate}
    \item The processor JOP
    \item Supporting tools
    \item The Java library and applications
\end{enumerate}

The different entities are also reflected during the configuration
and download process. The download is a two step process:
\begin{enumerate}
    \item Configuration of the FPGA: JOP is downloaded via a FPGA
        download cable (e.g., ByteBlaster on the PCs parallel
        port). After FPGA configuration the processor
        automatically starts and listens to the second channel
        (the serial line) for the software download.
    \item Java application download: the compiled and linked
    application is downloaded usually via a serial line. JOP stores
    the application in the main memory and starts execution at
    \code{main()} after the download.
\end{enumerate}

Further details of the source structure can be found in
Section~\ref{sec:directory}.

\section{A Short History}

The first version of JOP was created in 2000 based on the adaptation
of earlier processor designs created between 1995 and 2000. The first
version was written in Altera's proprietary AHDL language. The first
\emph{program} (3 bytecode instructions) ran on JOP on October 2,
2000. The first approach was a general purpose accumulator/register
machine with 16-bit instructions, 32-bit registers, and a pipeline
length of 3. It used the on-chip block memory to implement (somehow
unusual) 1024 registers.

The JVM was implemented in the assembler of that machine. That
concept was similar to the microcode in the current JOP version. The
decoding of the bytecode was performed by a long jump table. In the
best case (assuming a local, single cycle memory) a simple bytecode
(e.g.\ \code{iadd}) took 12 cycles for fetch and decode and
additional 11 cycles for execution.


A redesign followed in April 2001, now coded in VHDL. The second
version of JOP introduced features to speed up the implementation of
the JVM with specific instructions for the stack access and a
dedicated stack pointer. The register file was reduced to 16 entries
and the instruction width reduced to 8 bits. The pipeline contained 5
stages and special support for decoding bytecode instructions was
added -- a first version of the dynamic bytecode to microcode address
translation as it is used in the current version of JOP. The
enhancements within JOP2 resulted in the reduction of the execution
time for a simple bytecode to 3 cycles. A great enhancement, compared
to the 23 cycles in JOP1.

The next redesign (JOP3) followed in June 2001. The challenge was to
execute simple bytecodes fully pipelined in a single cycle. The
microcode instruction set was changed to implement a stack machine
and the execution stage combined with the on-chip stack cache.
Microcode instructions where coded in 16 bit and the pipeline was
reduced to four stages. JOP3 is the basis of JOP as it is described
in this handbook. The later changes have not been so radical to call
them a redesign.

The first real-world application of JOP was in the project
\emph{Kippfahrleitung} (see Section~\ref{sec:app:kfl}). At the start
of the project (October 2001) JOP could only execute a single static
method stored in the on-chip memory. The project greatly pushed the
development of JOP. After successful deployment of the JOP-based
control system in the field, several projects followed (TeleAlarm,
Lift, the railway control system). The source of the commercial
applications is part of the JOP distribution. Some of these
applications are now used as a test bench for embedded Java
performance and to benchmark WCET analysis tools.

More details and the source code of
JOP1\footnote{\url{http://www.jopdesign.com/jop1.jsp}},
JOP2\footnote{\url{http://www.jopdesign.com/jop2.jsp}} and the first
JOP3\footnote{\url{http://www.jopdesign.com/jop3.jsp}} version are
available on the web site.


\section{JOP Features}

This book presents a hardware implementation of the Java virtual
machine (JVM), targeting small embedded systems with real-time
constraints. JOP is designed from the ground up with time-predictable
execution of Java bytecode as a major design goal. All functional
units, and especially the interactions between them, are carefully
designed to avoid any time dependency between bytecodes.

JOP is a stack computer with its own instruction set, called
microcode in this book. Java bytecodes are translated into microcode
instructions or sequences of microcode. The difference between the
JVM and JOP is best described as the following:
\begin{quote}
The JVM is a CISC stack architecture, whereas JOP is a RISC stack
architecture.
\end{quote}

The architectural features and highlights of JOP are:

\begin{itemize}

    \item Dynamic translation of the CISC Java bytecodes to a
        RISC, stack based instruction set (the microcode) that
        can be executed in a 3 stage pipeline.

    \item The translation takes exactly one cycle per bytecode
        and is therefore pipelined. Compared to other forms of
        dynamic code translation the translation does not add any
        variable latency to the execution time and is therefore
        time predictable.

    \item Interrupts are inserted in the translation stage as
        special bytecodes and are transparent to the microcode
        pipeline.

    \item The short pipeline (4 stages) results in short
        conditional branch delays and a hard to analyze branch
        prediction logic or branch target buffer can be avoided.

    \item Simple execution stage with the two topmost stack
        elements as discrete registers. No write back stage or
        forwarding logic is needed.

    \item Constant execution time (one cycle) for all microcode
        instructions. The microcode pipeline never stalls. Loads
        and stores of object fields are handled explicitly.

    \item No time dependencies between bytecodes result in a
        simple processor model for the low-level WCET analysis.

    \item Time predictable instruction cache that caches whole
        methods. Only invoke and return instruction can result in
        a cache miss. All other instructions are guaranteed cache
        hits.

    \item Time predictable data cache for local variables and the
        operand stack. Access to local variables is a guaranteed
        hit and no pipeline stall can happen. Stack cache fill
        and spill is under microcode control and analyzable.

    \item No prefetch buffers or store buffers that can introduce
        unbounded time dependencies of instructions. Even simple
        processors can contain an instruction prefetch buffer
        that prohibits exact WCET values. The design of the
        method cache and the translation unit avoids the variable
        latency of a prefetch buffer.

    \item Good average case performance compared with other non
        real-time Java processors.

    \item Avoidance of hard to analyze architectural features
        results in a very small design. Therefore an available
        real estate can be used for a chip multi-processor
        solution.

    \item JOP is the smallest hardware implementation of the JVM
        available to date. This fact enables usage of low-cost
        FPGAs in embedded systems. The resource usage of JOP can
        be configured to trade size against performance for
        different application domains.


    \item JOP is actually in use in several real-world
        applications showing that a Java based embedded system
        implemented in an FPGA is a viable option.

\end{itemize}

JOP is implemented as a soft-core in a field programmable gate array
(FPGA) giving a lot of flexibility for the overall hardware design.
The processor can easily be extended by peripheral components inside
the same chip. Therefore, it is possible to customize the solution
exactly to the needs of the system.


\section{Is JOP the Solution for Your Problem?}

I had a lot of fun, and still have, developing and using JOP.
However, should you use JOP? JOP is a processor design intended as a
time predictable solution for hard real-time systems. If your
application or research focus is on those systems and you prefer Java
as programming language, JOP is the right choice. If you are
interested in larger, dynamic systems, JOP is the wrong choice. If
average performance is important for you and you do not care about
worst-case performance other solutions will probably do a better job.

\section{Outline of the Book}

Chapter~\ref{chap:build} gives a detailed introduction into the
design flow of JOP. It explains how the individual parts are compiled
and which files have to be changed when you want to extend JOP or
adapt it to a new hardware platform. The chapter is concluded by an
exercise to explore the different steps in the design flow.

Chapter~\ref{chap:java} provides background information on the Java
programming language, the execution environment, and the Java virtual
machine, for Java applications. If you are already familiar with Java
and the JVM, feel free to skip this chapter.

Chapter~\ref{chap:arch} is the main chapter in which the architecture
of JOP is described. The motivation behind different design decisions
is given. A Java processor alone is not a complete JVM.
Chapter~\ref{chap:runtime} describes the runtime environment on top
of JOP, including the definition of a real-time profile for Java and
the description of the scheduler in Java.

In Chapter~\ref{chap:wcet} worst-case execution time (WCET) analysis
for JOP is presented. It is shown how the time-predictable bytecode
instructions form the basis of WCET analysis of Java applications.

Garbage collection (GC) is an important part of the Java technology.
Even in real-time systems new real-time garbage collectors emerge. In
Chapter~\ref{chap:rtgc} the formulas to calculate the correct
scheduling of the GC thread are given and the implementation of the
real-time GC for JOP is explained.

JOP uses a simple and efficient system-on-chip interconnection called
SimpCon to connect the memory controller and peripheral devices to
the processor pipeline. The definition of SimpCon and the rationale
behind the SimpCon specification is given in
Chapter~\ref{chap:simpcon}. Based on a SimpCon memory arbiter,
chip-multiprocessor (CMP) versions of JOP can be configured.
Chapter~\ref{chap:cmp} gives some background information on the JOP
CMP system.

%Chapter~\ref{chap:ejip} sketches the implementation of an embedded
%TCP/IP stack called \code{ejip}.

In Chapter~\ref{chap:results}, JOP is evaluated with respect to size
and performance. This is followed by a description of some commercial
real-world applications of JOP. Other hardware implementations of the
JVM are presented in Chapter~\ref{chap:related}. Different hardware
solutions from both academia and industry for accelerating Java in
embedded systems are analyzed.

Finally, in Chapter~\ref{chap:conclusions}, the work is summarized
and the major contributions are presented. This chapter concludes
with directions for future work using JOP and real-time Java. A more
theoretical treatment of the design of JOP can be found in the PhD
thesis \cite{jop:thesis}, which is also available as book
\cite{jop:thesis:book}.
