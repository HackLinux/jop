

This handbook introduces a Java processor for embedded real-time
systems, in particular the design of a small processor for
resource-constrained devices with time-predictable execution of Java
programs. This Java processor is called JOP -- which stands for Java
Optimized Processor --, based on the assumption that a full native
implementation of all Java bytecode instructions is not a useful
approach.

\section{A Quick Tour on JOP}

In the following section we will give a quick overview on JOP and a
short description how to get JOP running within an FPGA. A detailed
description of the build process can be found in
Chapter~\ref{chap:build}.

JOP is the soft-core written in VHDL plus tools, a simplified Java
library (JDK), and some application examples. JOP is delivered in
source only. The source contains a bunch of VHDL files for the
processor core and Java files. The JOP sources are hosted at
Opencores\footnote{\url{http://www.opencores.org/projects.cgi/web/jop}}.


\subsection{Building JOP and Running ``Hello World"}

To build JOP you first have to download the source tree from
Opencores. A \emph{Makefile} (or an Ant file) contains all necessary
steps to build the tools, the processor, and the application.
Configuration of the FPGA and downloading the Java application is
also part of the Makefile.

In this description we assume the FPGA board Cycore (see
Appendix~\ref{appx:cycore}). This board is the default target for
the Makefile. The board has to be connected to the power supply and
to the PC via a ByteBlaster download cable and a serial cable.

The FPGA is configured via the ByteBlaster cable. The Java
application is downloaded after the FPGA configuration via the
serial cable. Besides the download the serial cable is also used as
communication link between JOP and the PC. \code{System.out} and
\code{System.in} represent this serial link on JOP.

In order to build the whole system you need a Java
compiler\footnote{Download the Java SE Development Kit (JDK) from
\url{http://java.sun.com/javase/downloads/index.jsp}.} and an FPGA
compiler. In our case we use the free web edition of Quartus from
Altera\footnote{\url{http://www.altera.com/}}. As we use \cmd{make}
and the preprocessor from the GNU compiler collection
Cygwin\footnote{\url{http://www.cygwin.com/}} should be installed
under Windows.

When all tools are setup correctly\footnote{Check at the command
prompt that \cmd{javac} is in the path.} a simple \cmd{make} should
build the tools, the processor, compile the ``Hello World" example,
configure the FPGA and download the application. The whole build
process will take a few minutes. After typing
\begin{verbatim}
    make
\end{verbatim}
you should see a lot of messages from the various tools. However,
the last lines should be actual messages received from JOP. It
should look similar to the following:
\begin{verbatim}
    JOP start
    ...
    V 20070831 - 60 MHz, 1024 KB RAM
    Hello World from JOP!

    JVM exit!
\end{verbatim}
Note that JOP prints some internal information, such as version and
memory size, at the startup. After that, the message ``Hello World
from JOP!" can be seen. Our first program runs on JOP!

As next step locate the hello world example in the source
tree\footnote{\dirent{.../jop/java/target/src/test/test/HelloWorld.java}}
and change the output message. The tools and the processor have been
built already. So we do not need to compile everything from scratch.
Use the following make target to just compile the Java application
and download the processor and the application:
\begin{verbatim}
    make japp
\end{verbatim}
The compile process should now be way faster and the output similar
to before.

The hello world application is the default target in the Makefile.
See Chapter~\ref{chap:build} for a description how this target can
be changed. In case you use a different FPGA board you find
information how to change the build process also in
Chapter~\ref{chap:build}.

\subsection{The Design Structure}

Browsing the source tree of JOP can give the impression that the
design is complex. However, the basic structure is not that complex.
The design consists of three entities:
\begin{enumerate}
    \item The processor JOP
    \item Supporting tools
    \item The Java library and application
\end{enumerate}

The different entities are also reflected during the configuration
and download process. The download is a two step process:
\begin{enumerate}
    \item Configuration of the FPGA: JOP is downloaded via a
    FPGA download cable (e.g.\ ByteBlaster on the PCs parallel
    port). After FPGA configuration the processor automatically starts and
    listens to the second channel (the serial line) for the software download.
    \item Java application download: the compiled and linked
    application is downloaded usually via a serial line. JOP stores
    the application in the main memory and starts execution at
    \code{main()} after the download.
\end{enumerate}

Further details of the source structure can be found in
Section~\ref{sec:directory}.

\section{A Short History}

The first version of JOP was created in 2000 based on the adaption
of earlier processor designs created between 1995 and 2000. The
first version was written in Altera's proprietary AHDL language. The
first \emph{program} (3 bytecode instructions) run on JOP October 2,
2000. The first approach was general purpose accu/register machine
with 16-bit instructions, 32-bit registers, and a pipeline length of
3. It used the on-chip block memory to implement (somehow unusual)
1024 registers.

The JVM was implemented in the assembler of that machine. That
concept was similar to the microcode in the current JOP version. The
decoding of the bytecode was performed by a long jump table. In the
best case (assuming a local, single cycle memory) a simple bytecode
(e.g.\ \code{iadd}) took 12 cycles for fetch and decode and
additional 11 cycles for execution.


A redesign followed in April 2001, now coded in VHDL. The version 2
of JOP introduced features to speedup the implementation of the JVM
with specific instructions for the stack access and a dedicated
stack pointer. The register file was reduced to 16 entries and the
instruction width reduced to 8 bit. The pipeline contained 5 stages
and special support for decoding bytecode instruction was added -- a
first version of the dynamic bytecode to microcode address
translation as it is used in the current version of JOP. The
enhancements within JOP2 resulted in the reduction of the execution
time for a simple bytecode to 3 cycles. A great enhancement compared
to the 23 cycles in JOP1.

The next redesign (JOP3) followed in June 2001. The challenge was to
execute simple bytecodes fully pipelined in a single cycle. The
microcode instruction set was changed to implement a stack machine
and the execution stage combined with the on-chip stack cache.
Microcode instructions where coded in 16 bit and the pipeline was
reduced to four stages. JOP3 is the basis of JOP as it is described
in this handbook. The later changes have not been so radical to name
them a redesign.

The first real-world application of JOP was in the project
\emph{Kippfahrleitung} (see Section~\ref{sec:app:kfl}). At the start
of the project (October 2001) JOP could only execute a single static
method stored in the on-chip memory. The project greatly pushed the
development of JOP. After successful deployment of the JOP based
control system in the field several projects followed (TeleAlarm,
Lift, the railway control system). The source of the commercial
applications is part of the JOP distribution. Some of these
applications are now used as a test bench for embedded Java
performance and to benchmark WCET analysis tools.

More details and the sourc code of
JOP1\footnote{\url{http://www.jopdesign.com/jop1.jsp}},
JOP2\footnote{\url{http://www.jopdesign.com/jop2.jsp}} and the first
JOP3\footnote{\url{http://www.jopdesign.com/jop3.jsp}} version are
available on the web site.


\section{JOP Features}

This book presents a hardware implementation of the Java virtual
machine (JVM), targeting small embedded systems with real-time
constraints. The processor is designed from the ground up for low
worst-case execution time (WCET) of bytecodes, in order to give
tasks low WCET values.

JOP is a stack computer with its own instruction set, called
microcode in this book. Java bytecodes are translated into microcode
instructions or sequences of microcode. The difference between the
JVM and JOP is best described as the following:
\begin{quote}
The JVM is a CISC stack architecture, whereas JOP is a RISC stack
architecture.
\end{quote}

JOP will help to increase the acceptance of Java for embedded
real-time systems. JOP is implemented as a soft-core in a field
programmable gate array (FPGA). Implementing a processor for
embedded systems in an FPGA gives a lot of flexibility for the
overall hardware design. The processor can easily be extended by
peripheral components inside the same chip. Therefore, it is
possible to customize the solution exactly to the needs of the
system.

JOP is designed from ground up with time predictable execution of
Java bytecode as major design goal. All function units, and
especially the interaction between them, are carefully designed to
avoid any time dependency between bytecodes. The architectural
features and highlights are:

\begin{itemize}

    \item
The execution time for Java bytecodes can be exactly predicted in
terms of the number of clock cycles. There is no mutual dependency
between consecutive bytecodes. Therefore, no pipeline analysis --
with possible unbound timing effects -- is necessary. These
properties result in a simple processor model for the low-level WCET
analysis.

    \item
In order to fill the gap between processor speed and the memory
access time, caches are mandatory. In Section~\ref{sec:cache}, a
novel way to organize an instruction cache, as \emph{method cache},
is provided. The time predictable instruction cache caches whole
methods. Only invoke and return instruction can result in a cache
miss. All other instructions are guaranteed cache hits. This method
cache is simple to analyze with respect to worst-case behavior and
still provides a substantial performance gain.


Prefetch buffers or store buffers that can introduce unbound time
dependencies between instructions are completely avoided in the
design. Even simple processors can contain an instruction prefetch
buffer that prohibits exact WCET values. The design of the method
cache and the translation unit avoids the variable latency of a
prefetch buffer.



    \item
JOP is microprogrammed using a novel way of mapping bytecodes to
microcode addresses. This mapping has minimum overheads, even for
complex bytecodes. CISC Java bytecodes are dynamically translation
to a RISC, stack based instruction set (the microcode) that can be
executed in a 3 stage pipeline.


The translation takes exactly one cycle per bytecode and is
therefore pipelined. Compared to other forms of dynamic code
translation the proposed translation does not add any variable
latency to the execution time and is therefore time predictable.

    \item
The short pipeline (4 stages) results in short conditional branch
delays and a hard to analyze branch prediction logic or a branch
target buffer can be avoided. All microcode instructions are
executed in constant time (one cycle). There are no stalls in the
microcode pipeline. Loads and stores of object fields are handled
explicitly. Interrupts are inserted in the translation stage as
special bytecodes and are transparent to the microcode pipeline.


    \item
A two-level stack cache, described in Section~\ref{sec:stack}, which
fits to the embedded memory technologies of current FPGAs and ASICs,
ensures the fast execution of basic instructions with minimum
resource requirements.


The first level consists of the two topmost stack elements as
discrete registers. Those two registers are the basis of the
execution stage. The combination of the first level stack cache and
the execution unit does not need a write back stage or any
forwarding logic.


The second level provides fast and time predictable access to local
variables and the operand stack. Access to local variables is a
guaranteed hit and no pipeline stall can happen. Fill and spill of
the stack cache is subjected to microcode control and therefore
analyzable.


    \item
JOP is the smallest hardware implementation of the JVM available to
date. This fact enables usage of low-cost FPGAs in embedded systems.
The resource usage of JOP can be configured to trade size against
performance for different application domains.

Avoidance of hard to analyze architectural features results in the
very small design. Therefore, an available real estate can be used
for a chip multi-processor version of JOP.


    \item
JOP is actually in use in several real-world applications showing
that a Java based embedded system implemented in an FPGA is a viable
option. In Section~\ref{sec:applications} the usage of JOP in a
real-world application is described.

\end{itemize}

The proposed Java processor architecture results in time predictable
and high-performance execution of real-time tasks in Java, without
the resource implications and unpredictability of a JIT-compiler.

\section{Is JOP the Solution for Your Problem?}

I had a lot of fun, and still have, developing and using JOP.
However, should you use JOP? JOP is a processor design intended as a
time predictable solution for hard real-time systems. If your
application or research focus is on those systems and you prefer
Java as programming language JOP is the right choice. If you are
interested in larger, dynamic systems JOP is the wrong choice. If
average performance is important for you and you do not care about
worst-case performance other solutions will probably do a better
job.

\section{Outline of the Book}

Chapter~\ref{chap:build} gives a detailed introduction into the
design flow of JOP. You find the explanation how the individual
parts are compiled and which files have to be changed when you want
to extend JOP or adapt it to a new hardware platform. The chapter is
concluded by an exercise to evaluate the different steps in the
design flow.

Chapter~\ref{chap:java} provides background information on the Java
programming language and the execution environment, the Java virtual
machine, for Java applications. If you are already familiar with
Java and the JVM feel free to skip this chapter.

Standard Java is not suitable for the resource-constrained world of
embedded systems. Chapter~\ref{chap:rtjava} gives an overview of
various definitions embedded Java and the real-time specification of
Java (RTSJ). It is an extended version of \cite{jop:rtjava} and
Chapter~4 of \cite{jop:thesis}.

Chapter~\ref{chap:arch} is the main chapter in which the
architecture of JOP is described. The motivation behind different
design decisions is given. A Java processor alone is not a complete
JVM. Chapter~\ref{chap:runtime} describes the runtime environment on
top of JOP, including the definition of a real-time profile for Java
and a framework for a user-defined scheduler in Java.



Garbage collection (GC) is an important part of the Java technology.
Even in real-time systems new real-time garbage collectors emerge.
In Chapter~\ref{chap:rtgc} the formulas to calculate the correct
scheduling of the GC thread are given and the implementation of the
real-time GC for JOP is explained.

In Chapter~\ref{chap:wcet} WCET analysis of the individual Java
bytecodes is performed. It is shown how this bytecode instruction
timings build the basis for the WCET analysis of Java applications.

JOP uses a simple and efficient system-on-chip interconnection
called SimpCon to connect the memory controller and peripheral
devices to the processor pipeline. The definition of SimpCon and the
rationale behind the SimpCon specification is given in
Chapter~\ref{chap:simpcon}.

%Chapter~\ref{chap:ejip} sketches the implementation of an embedded
%TCP/IP stack called \code{ejip}.

In Chapter~\ref{chap:results}, JOP is evaluated with respect to size
and performance. This is followed by a description of some
commercial real-world applications of JOP.

Other solutions are presented in Chapter~\ref{chap:related}.
Different hardware solutions from both academia and industry for
accelerating Java in embedded systems are analyzed.

Finally, in Chapter~\ref{chap:conclusions}, the work is summarized
and the major contributions are presented. This chapter concludes
with directions for future research using JOP and real-time Java.
