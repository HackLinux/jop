\section{The JOP ``Hello World" Exercise}

This exercise gives an introduction into the design flow of JOP. JOP
will be built from the sources and a simple \emph{Hello World}
program will run on it.

To understand the build process you have to run the build manually.
This understanding will help you to find the correct files for
changes in JOP and to adjust the \code{Makefile} for your needs.

\subsection{Manual build}

Manual build does not mean entering all commands, but calling the
correct make target with the required arguments (if any) in the
correct order. The idea of this exercise is to obtain knowledge of
the directory structure and the dependencies of various design units.

Inspect the Makefile targets and the ones that are called from it
before running them.

\begin{enumerate}
    \item Create your working directory
    \item Download the sources from the opencores CVS server
    \item Connect the FPGA board (with the bread board expansion) to the PC and the power supply
    \item Perform the build as described in the \emph{Getting Started} section
    from Chapter~\ref{chap:build}
\end{enumerate}

As a result you should see a message at your command prompt.
% and a blinking LED on the FPGA board.

\subsection{Using make}

In the root directory (\code{jop}) there is a \code{Makefile}. Open
it with an editor and try to find the corresponding lines of code for
the steps you did in first exercise. Reset the FPGA by cycling the
power and run the build with a simple
\begin{verbatim}
    make
\end{verbatim}

The whole process should run without errors and the result should be
identical to the previous exercise.

\subsection{Change the Java Program}

The whole build process is not necessary when changing the Java
application. Once the processor is built, a Java application can be
built and downloaded with the following make target:
\begin{verbatim}
    make japp
\end{verbatim}
Change \code{HelloWorld.java} and run it on JOP. Now change the class
name that contains the \code{main()} methods (\code{HelloWorld.java})
to \code{Hello.java} and rerun the Java application build. Now an
embedded version of ``Hello World" should run on JOP. Besides the
usual greeting on the standard output, the LED on the FPGA board
should blink at a frequency of 1~Hz. The first periodic task, an
essential abstraction for real-time systems, is running on JOP!

\subsection{Change the Microcode}

The JVM is written in microcode and several \code{.vhdl} files are
generated during assembly. For a test change only the version
string\footnote{The actual version date will probably be different
from the actual sources.} in \code{jvm.asm} to the actual date and
run a full make.
\begin{verbatim}
    version     = 20070831
\end{verbatim}
    change to:
\begin{verbatim}
    version     = 20110101
\end{verbatim}
The start message should reflect your change. As the microcode was
changed a full make run is necessary. The microcode assembly
generates VHDL files and the complete processor has to be rebuilt.


\subsection{Use a Different Target Board}

In this exercise, you will alter the \code{Makefile} for a different
target board. Disconnect the first board and connect the board with
an USB port (e.g.\ the dspio or Lego board) and the ByteBlaster
cable.

Table~\ref{tab:boards} lists the  differences between the first
board (simpexp) and the new one (called dspio).
%
\begin{table}
    \centering

\begin{tabular}{lll}
    \toprule
       & simpexp & dspio \\
    \midrule
    FPGA & EP1C6 & EP1C12 \\
    IO & UART & UART, USB, audio codec, sigma-delta codec \\
    FPGA configuration & ByteBlaster & still ByteBlaster \\
    Java download & serial line & USB \\

    \bottomrule

\end{tabular}
    \caption{Differences between the two target boards}
    \label{tab:boards}

\end{table}
%
The correct FPGA is already selected in the Quartus project files
(\code{jop.qpf}). Alter the \code{Makefile} as follows:

\begin{enumerate}
    \item Change the Quartus project from \code{cycmin} to
    \code{usbmin}
    \item The Java download is now over USB instead of the serial
    line. Change the main target (\code{all}) to use sub target
    \code{jopusb} instead of \code{jopser}
    \item Change the parameters for the download via \code{down.exe}
    to use the virtual com-port of the USB driver (look into Windows
    hardware manager to get the correct number). Add the switch
    \code{-usb} for the download
\end{enumerate}

Now build the whole project with \code{make}. Change the Java
program and perform only the necessary build step.

\subsection{Compile a Different Java Application}

The class that contains the main method is described by three
arguments:
\begin{enumerate}
    \item The first directory relative to \code{java/target/src}
    (e.g. \code{app} or \code{test})
    \item The package name (e.g. \code{dsp})
    \item The main class (e.g. \code{HalloWorld})
\end{enumerate}

These three values are used by the \code{Makeile} and are set in the
variables \code{P1}, \code{P2}, and \code{P3} in the
\code{Makefile}.

Change the \code{Makefile} to compile the embedded Java benchmark
\code{jbe.DoAll}. The parameters for the Java application can also
be given to the \code{make} with following command line arguments:
\begin{verbatim}
    make -e P1=bench P2=jbe P3=DoAll
\end{verbatim}

The three variables \code{P1}, \code{P2}, and \code{P3} are a
shortcut to set the main class of the application. You can also
directly set the variables \code{TARGET\_APP\_PATH} and
\code{MAIN\_CLASS}.


\subsection{Simulation}

This optional exercise will give you a full view of the possibilities
to debug JOP system code or the processor itself. There are two ways
to simulate JOP: A simple debugging JVM written in Java
(\code{JopSim} as part of the tool package) that can execute
\emph{jopized} applications and VHDL level simulation with ModelSim.
The make targets are \code{jsim} and \code{sim}.

\subsection{WCET Analysis}

An important step in real-time system development is the analysis of
the WCET of the individual tasks. Compile and run the WCET example
\code{Loop.java} in package \code{wcet}. You can analyze the WCET of
the method \code{measure()} with following make command:
\begin{verbatim}
    make wcet -e P1=test P2=wcet P3=Loop
\end{verbatim}
Change the code in \code{Loop.java} to enable measurement of the
execution time and compare it with the output of the static analysis.
In this simple example the WCET can be measured. However, be aware
that most non-trivial code needs static analysis for safe estimates
of WCET values.
