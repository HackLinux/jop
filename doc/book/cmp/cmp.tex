\index{CMP}

In order to generate a small and predictable processor, several
advanced and resource-consuming features (such as instruction folding
or branch prediction) were omitted from the design. The resulting low
resource usage of JOP makes it possible to integrate more than one
processor in an FPGA. Since embedded applications are naturally
multi-threaded systems, the performance can easily be enhanced using
a multi-processor solution.

This chapter describes the configuration of a chip multiprocessor
(CMP) version of JOP. The various SimpCon based arbiters have been
developed by Christof Pitter and are described in \cite{jop:dma,
jop:cmp, jop:cmp:eval}. A multi-processor JVM with shared memory
offers research possibilities such as: scheduling of Java threads and
synchronization between the processors or WCET analysis for the
shared memory access.

The project file to start with is \code{cyccmp}, a configuration for
three processors with a TDMA based arbiter in the Cycore board with
the EP1C12.

\section{Memory Arbitration}

A central arbiter regulates the synchronization on memory read and
write operations.

\subsection{Main Memory}
\index{CMP!TDMA arbiter}

Three different arbiter are available for the access policy to the
main memory: priority based, fairness based, and a time division
multiple access (TDMA) arbiter. The main memory is shared between all
cores.

The TDMA based memory arbiter provides a static schedule for the
memory access. Therefore, access time to the memory is independent of
tasks running on other cores. The worst-case execution time (WCET) of
a memory loads or stores can be calculated by considering the
worst-case phasing of the memory access pattern relative to the TDMA
schedule \cite{tdma:arbiter:jtres2008}.


In the default configuration each processor cores has an equally
sized slot for the memory access. The TDMA schedule can also be
optimized for different utilizations of processing cores. The TDMA
schedule can be optimized to distribute slack time of tasks to other
tasks with a tighter deadline \cite{jop:cmp:single:thread}.


\subsection{I/O Devices}

Each core contains a set of local I/O devices, needed for the runtime
system (e.g., timer interrupt, lock support). The serial interface
for program download and a \emph{stdio} device is connected to the
first core.

For additional I/O devices two options exist: either they are
connected to one core, or shared by all/some cores. The first option
is useful when the bandwidth requirement of the I/O device is high.
As I/O devices are memory mapped they can be connected to the main
memory arbiter in the same way as the memory controller. In that case
the I/O devices are shared between the cores and standard
synchronization for the access is needed. For high bandwidth demands
a dedicated arbiter for I/O devices or even for a single device can
be used.

An interrupt line of an I/O device can be connected to a single core
or to several cores. As interrupts can be individually disabled in
software, a connection of all interrupt lines to all cores provides
the most flexible solution.







\section{Booting a CMP System}

\index{CMP!booting}

One interesting aspect of a CMP system is how the startup or boot-up
is performed. On power-up, the FPGA starts the configuration state
machine to read the FPGA configuration data either from a Flash
memory or via a download cable from the PC during the development
process. When the configuration has finished, an internal reset is
generated. After this reset, microcode instructions are executed,
starting from address 0. At this stage, we have not yet loaded any
application program (Java bytecode). The first sequence in microcode
performs this task. The Java application can be loaded from an
external Flash memory, via a PC serial line, or an USB-port.

In the next step, a minimal stack frame is generated and the special
method \code{Startup.boot()} is invoked, even though some parts of
the JVM are not yet setup. From now on JOP runs in Java mode. The
method \code{boot()} performs the following steps:
\begin{samepage}
\begin{itemize}
    \item Send a greeting message to \emph{stdout}
    \item Detect the size of the main memory
    \item Initialize the data structures for the garbage
        collector
    \item Initialize \code{java.lang.System}
    \item Print out JOP's version number, detected clock speed,
        and memory size
    \item Invoke the static class initializers in a predefined
        order
    \item Invoke the \code{main} method of the application class
\end{itemize}
\end{samepage}

The boot-up process is the same for all processors until the
generation of the internal reset and the execution of the first
microcode instruction. From that point on, we have to take care that
\emph{only one} processor performs the initialization steps.

All processors in the CMP are functionally identical. Only one
processor is designated to boot-up and initialize the whole system.
Therefore, it is necessary to distinguish between the different CPUs.
We assign a unique CPU identity number (CPU ID) to each processor.
Only processor CPU0 is designated to do all the boot-up and
initialization work. The other CPUs have to wait until CPU0 completes
the boot-up and initialization sequence. At the beginning of the
booting sequence, CPU0 loads the Java application. Meanwhile, all
other processors are waiting for an \emph{initialization finished}
signal of CPU0. This busy wait is performed in microcode. When the
other CPUs are enabled, they will run the same sequence as CPU0.
Therefore, the initialization steps are guarded by a condition on the
CPU ID.

\section{CMP Scheduling}

\index{CMP!scheduling}

There are two possibilities to run multiple threads on the CMP
system:

\begin{enumerate}
  \item A single thread per processor
  \item Several threads on each processor
\end{enumerate}

For the configuration of one thread per processor the scheduler does
not need to be started. Running several threads on each core is
managed via the JOP real-time threads \code{RtThread}.

The scheduler on each core is a preemptive, priority based real-time
scheduler. As each thread gets a unique priority, no FIFO queues
within priorities are needed. The best analyzable real-time CMP
scheduler does not allow threads to migrate between cores. Each
thread is pinned to a single core at creation. Therefore, standard
scheduling analysis can be performed on a per core base. Threads
cannot migrate from one core to another one.

Similar to the uniprocessor version of JOP, the application is
divided into an initialization phase and a mission phase. During the
initialization phase, a predetermined core executes only one thread
that has to create all data structures and the threads for the
mission phase. During transition to the mission phase all created
threads are started.

The uniprocessor real-time scheduler for JOP has been enhanced to
facilitate the scheduling of threads in the CMP configuration. Each
core executes its own instance of the scheduler. The scheduler is
implemented as \code{Runnable}, which is registered as an interrupt
handler for the core local timer interrupt. The scheduling is not
tick-based. Instead, the timer interrupt is reprogrammed after each
scheduling decision. During the mission start, the other cores and
timer interrupts are enabled.

Another interesting option to use a CMP system is to execute exactly
one thread per core. In this configuration scheduling overheads can
be avoided and each core can reach an utilization of 100\% without
missing a deadline. To explore the CMP system without a scheduler, a
mechanism is provided to register objects, which implement the
\code{Runnable} interface, for each core. When the other cores are
enabled, they execute the \code{run} method of the \code{Runnable} as
their \emph{main} method.

\subsection{One Thread per Core}

The first processor executes, as usual, \code{main()}. To execute
code on the other cores a \code{Runnable} has to be registered for
each core. After registering those Runnables the other cores need to
be started. The code in Listing~\ref{lst:cmp:hello} shows an example
that can be found in \dirent{test/cmp/HelloCMP.java}.

\begin{lstlisting}[float=t,caption={A CMP version of Hello World},
label=lst:cmp:hello]

public class HelloCMP implements Runnable {

    int id;
    static Vector msg;

    public HelloCMP(int i) {
        id = i;
    }

    public static void main(String[] args) {

        msg = new Vector();
        System.out.println("Hello World from CPU 0");

        SysDevice sys = IOFactory.getFactory().getSysDevice();
        for (int i=0; i<sys.nrCpu-1; ++i) {
            Runnable r = new HelloCMP(i+1);
            Startup.setRunnable(r, i);
        }

        // start the other CPUs
        sys.signal = 1;
        // print their messages
        for (;;) {
            int size = msg.size();
            if (size!=0) {
                StringBuffer sb = (StringBuffer) msg.remove(0);
                System.out.println(sb);
            }
        }
    }

    public void run() {
        StringBuffer sb = new StringBuffer();
        sb.append("Hello World from CPU ");
        sb.append(id);
        msg.addElement(sb);
    }
}
\end{lstlisting}

\subsection{Scheduling on the CMP System}

Running several threads on each core is possible with \code{RtThread}
and setting the core for each thread with
\code{RtThread.setProcessor(nr)}. The example in
Listing~\ref{lst:cmp:hello:rtthread}
(\code{test/cmp/RtHelloCMP.java}) shows registering of 50 threads on
all available cores. On \code{missionStart()} the threads are
distributed to the cores, a scheduler for each core registered as
timer interrupt handler, and the other cores started.

\begin{lstlisting}[float=t,caption={A CMP version of Hello World with the scheduler},
label=lst:cmp:hello:rtthread]

public class RtHelloCMP extends RtThread {

    public RtHelloCMP(int prio, int us) { super(prio, us); }

    int id;
    public static Vector msg;
    final static int NR_THREADS = 50;

    public static void main(String[] args) {
        msg = new Vector();
        System.out.println("Hello World from CPU 0");
        SysDevice sys = IOFactory.getFactory().getSysDevice();
        for (int i=0; i<NR_THREADS; ++i) {
            RtHelloCMP th = new RtHelloCMP(1, 1000*1000);
            th.id = i;
            th.setProcessor(i%sys.nrCpu);
        }
        RtThread.startMission();    // start mission and other CPUs
        for (;;) {                  // print their messages
            RtThread.sleepMs(5);
            int size = msg.size();
            if (size!=0) {
                StringBuffer sb = (StringBuffer) msg.remove(0);
                for (int i=0; i<sb.length(); ++i) {
                    System.out.print(sb.charAt(i));
    }}}}

    public void run() {
        StringBuffer sb = new StringBuffer();
        StringBuffer ping = new StringBuffer();
        sb.append("Thread "); sb.append((char) ('A'+id)); sb.append(" start on CPU ");
        sb.append(IOFactory.getFactory().getSysDevice().cpuId); sb.append("\r\n");
        msg.addElement(sb);
        waitForNextPeriod();
        for (;;) {
            ping.setLength(0);
            ping.append((char) ('A'+id));
            msg.addElement(ping);
            waitForNextPeriod();
    }}
}
\end{lstlisting}
