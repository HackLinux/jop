SimpCon \cite{simpcon} is the main interconnection interface used
for JOP. The IO modules and the main memory are connected via this
standard. In the following chapter an introduction to SimpCon is
presented.

The VHDL files in \dirent{vhdl/scio} are SimpCon IO components
(e.g.\ \code{sc\_uart.vhd} is a simple UART) and SimpCon IO
configurations. The IO configurations define the IO devices and the
address mapping for a JOP system. All those configurations start
with \code{scio\_}. The IO components start with \code{sc\_}.
Configuration \code{scio\_min} contains the minimal IO components
for a JOP system: the system module \code{sc\_sys.vhd} and a UART
\code{sc\_uart.vhd} for program download and basic communication
(\code{System.in} and \code{System.out}).

The system module \code{sc\_sys.vhd} contains the clock counter, the
$\mu$s counter, timer interrupt, the SW interrupt, exception
interrupts, the watchdog port, and the connection to the
multiprocessor synchronization unit (\code{cmpsync.vhd}).

In directory \dirent{vhdl/memory} the memory controller
\code{mem\_sc.vhd} is a SimpCon master that can be connected to
various SRAM memory controllers \code{sc\_sram*.vhd}. Other memory
controller (e.g.\ the free Altera SDRAM interface) can be connected
via SimpCon bridges to Avalon, Wishbone, and AHB slave (available in
\dirent{vhdl/simpcon}).
