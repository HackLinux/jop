
\section{A Real-Time Profile for Embedded Java}
\label{sec:rtprof}

As standard Java is under-specified for real-time systems and the
RTSJ does not fit for small embedded systems a new and simpler
real-time profile is defined in this section and implemented on JOP.
The guidelines of the specification are:

\begin{itemize}
\item High-integrity profile
\item Easy syntax, simplicity
\item Easy to implement
\item Low runtime overhead
\item No syntactic extension of Java
\item Minimum change of Java semantics
\item Support for time measurement if a WCET analysis tool is not available
\item Known overheads (documentation of runtime behavior and memory
requirements of every JVM operation and all methods have to be
provided)
\end{itemize}

The real-time profile under discussion is inspired by the restricted
versions of the RTSJ described in \cite{Pusch01} and \cite{583825}
(see Section~\ref{subsec:restr:rtsj}). It is intended for
high-integrity real-time applications and as a test case to evaluate
the architecture of JOP as a Java processor for real-time systems.

The proposed definition is not compatible with the RTSJ. Since the
application domain for the RTSJ is different from high-integrity
systems, it makes sense for it \emph{not} to be compatible with the
RTSJ. Restrictions can be enforced by defining new classes (e.g.\
setting thread priority in the constructor of a real-time thread
alone, enforcing minimum interarrival times for sporadic events).

%Hardware interrupts, that are usually handled by device drivers, are
%part of this profile. These interrupts are mapped to events and
%scheduled in the same way as application threads. This feature
%allows priority assignment to the device drivers and the execution
%time can be incorporated in the schedulability analysis with normal
%tasks. This solution also avoids problems with preemption latency
%caused by device drivers.

All hardware interrupts are represented by threads under the control
of the scheduler. With this solution, a priority is assigned to the
device drivers and the execution time can be incorporated in the
schedulability analysis with normal tasks. This solution also avoids
problems with preemption latency provoked by device drivers. One
example of this problem is the \emph{caps-lock} issue in Linux
\cite{REDLinux2003}: A device driver performs a spinlock wait for
keyboard acknowledgement and produces preemption latency up to
9166$\mu$s. With the proposed concept of hardware interrupts under
scheduler control, a lower assigned priority to such a device driver
avoids preemption delays of \emph{more important} real-time threads
and events.

%This specification is functional compatible with Ravenscar-Java (RJ)
%\cite{583825}, but avoids inheritance of complex RTSJ classes. In
%fact, it is possible (and has been done) to implement RJ with the
%additional necessary RTSJ classes on top of it.

To verify that this specification is expressive enough for
high-integrity real-time applications, Ravenscar-Java (RJ)
\cite{583825} (see Section~\ref{subsec:rj}), with the additional
necessary RTSJ classes, has been implemented on top of it. However,
RJ inherits some of the complexity of the RTSJ. Therefore, the
implementation of RJ has a larger memory and runtime overhead than
this simple specification.

\subsection{Application Structure}

The application is divided in two different phases:
\emph{initialization} and \emph{mission}. All non time-critical
initialization, global object allocations, thread creation and
startup are performed in the initialization phase. All classes need
to be loaded and initialized in this phase. The mission phase starts
after invocation of \code{startMission()}. The number of threads is
fixed and the assigned priorities remain unchanged. The following
restrictions apply to the application:

\begin{itemize}
\item Initialization and mission phase
\item Fixed number of threads
\item Threads are created at initialization phase
\item All shared objects are allocated at initialization
\end{itemize}

\subsection{Threads}

Concurrency is expressed with two types of \emph{schedulable
objects}:
\begin{description}
    \item[Periodic activities] are represented by threads that execute
in an infinite loop invoking \code{waitForNextPeriod()} to get
rescheduled in predefined time intervals.

    \item[Asynchronous sporadic activities] are represented by event
handlers. Each event handler is in fact a thread, which is released
by an hardware interrupt or a software generated event (invocation
of \code{fire()}). Minimum interarrival time has to be specified on
creation of the event handler.

\end{description}
%
The classes that implement the \emph{schedulable objects} are:
%
\begin{description}
    \item[RtThread] represents a periodic task. As usual task
work is coded in \code{run()}, which gets invoked on
\code{missionStart()}. A scoped memory object can be attached to an
\code{RtThread} at creation.

    \item[HwEvent] represents an interrupt with a minimum
interarrival time. If the hardware generates more interrupts, they
get lost.

    \item[SwEvent] represents a software-generated event.
It is triggered by \code{fire()} and needs to override
\code{handle()}.

\end{description}
%
Listing~\ref{lst:arch:rt:profile:schobj} shows the definition of the
basic classes.

\begin{lstlisting}[float,caption={Schedulable objects},
label=lst:arch:rt:profile:schobj,{emph=RtThread,enterMemory,
exitMemory,run,waitForNextPeriod,startMission,HwEvent,handle,
SwEvent,fire,handle}]

public class RtThread {

    public RtThread(int priority, int period)
    public RtThread(int priority, int period, int offset)
    public RtThread(int priority, int period, Memory mem)
    public RtThread(int priority, int period, int offset,
                    Memory mem)

    public void enterMemory()
    public void exitMemory()

    public void run()
    public boolean waitForNextPeriod()

    public static void startMission()
}

public class HwEvent extends RtThread {

    public HwEvent(int priority, int minTime, int number)
    public HwEvent(int priority, int minTime, Memory mem,
                   int number)

    public void handle()
}

public class SwEvent extends RtThread {

    public SwEvent(int priority, int minTime)
    public SwEvent(int priority, int minTime, Memory mem)

    public final void fire()
    public void handle()
}
\end{lstlisting}

Listing~\ref{lst:arch:rt:profile:example} shows the principle coding
of a worker thread. An example for creation of two real-time threads
and an event handler can be seen in
Listing~\ref{lst:arch:rt:profile:usage}.


\subsection{Scheduling}


The scheduler is a preemptive, priority-based scheduler with
unlimited priority levels and a unique priority value for each
schedulable object. No real-time threads or events are scheduled
during the initialization phase.

The design decision to use unique priority levels, instead of FIFO
within priorities, is based on following facts: Two common ways to
assign priorities are rate monotonic and, in a more general form,
deadline monotonic assignment. When two tasks are given the same
priority, we can choose one of them and assign a higher priority to
that task and the task set will still be schedulable. This results
in a strictly monotonic priority order and we do not need to deal
with FIFO order. This eliminates queues for each priority level and
results in a single, priority ordered task list with unlimited
priority levels.

Synchronized blocks are executed with priority ceiling emulation
protocol. An object, used for synchronization, for which the
priority is not set, top priority is assumed. This avoids priority
inversions on objects that are not accessible from the application
(e.g. objects inside a library).

In addition, the scheduler contains methods for worst-case time
measurement for both the periodic work and handler methods. These
measured execution times can be used during development when no WCET
analysis tool is available.

\subsection{Memory}

The profile does not support a garbage collector. All memory should
be allocated at the initialization phase. Without a garbage
collector, the heap implicitly becomes immortal memory (as defined
by the RTSJ). For objects created during the mission phase, a scoped
memory is provided\footnote{As we now consider real-time GC as the
better solution scopes are not supported in the current
implementation of the profile}. Each scoped memory area is assigned
to one \code{RtThread}. A scoped memory area cannot be shared
between threads. No references are allowed from the heap to scoped
memory. Scoped memory is explicitly entered and left using
invocations from the application logic. Memory areas are cleared
both on creation and when leaving the scope (invocation of
\code{exitMemory()}), leading to a memory area with constant
allocation time, as opposed to memory with linear allocation time
(as the memory type \code{LTMemory} in the RTSJ)
\cite{Corsaro:2003:DPR}.


\subsection{Restriction of Java}

A list of some of the language features that should be avoided for
WCET analyzable real-time threads and bound memory usage:

\begin{description}
    \item[WCET:] Only analyzable language constructs are allowed (see \cite{84850}).

    \item[Static class initialization:] Since the definition when
to call the static class initializer is problematic (see
Section~\ref{para:restrict:clinit}), they are disallowed. Move this
code to a static method (e.g. \code{init()}) and invoke it explicit
in the initialization phase.

    \item[Inheritance:] Reduce usage of interfaces and overridden methods.

    \item[String concatenation:] In immortal memory scope only string concatenation with string
literals is allowed.

    \item[Finalization:] \code{finalize()} has a weak definition
in Java. Because real-time systems run \emph{forever}, objects in
the heap, which is immortal in this specification, will never be
finalized. Objects in scoped memory are released on
\code{exitMemory()}. However, finalizations on these objects
complicate WCET analysis of \code{exitMemory()}.

    \item[Dynamic Class Loading:] Due to the implementation and WCET analysis
complexity dynamic class loading is avoided.

\end{description}
%
A program analysis tool can greatly help in enforcing these
restrictions.


\begin{lstlisting}[float,caption={A periodic real-time thread},
label=lst:arch:rt:profile:example]

public class Worker extends RtThread {

    private SwEvent event;

    public Worker(int p, int t,
                    SwEvent ev) {

        super(p, t,
            // create a scoped memory area
            new Memory(10000)
        );
        event = ev;
        init();
    }

    private void init() {
        // all initialzation stuff
        // has to be placed here
    }

    public void run() {

        for (;;) {
            work();       // do some work
            event.fire(); // and fire an event

            // some work in scoped memory
            enterMemory();
            workWithMem();
            exitMemory();

            // wait for next period
            if (!waitForNextPeriod()) {
                missedDeadline();
            }
        }
        // should never reach this point
    }
}
\end{lstlisting}

\begin{lstlisting}[float,caption={Start of the application},
label=lst:arch:rt:profile:usage]
    // create an Event
    Handler h = new Handler(3, 1000);

    // create two worker threads with
    // priorities according to their periods
    FastWorker fw = new FastWorker(2, 2000);
    Worker w = new Worker(1, 10000, h);

    // change to mission phase for all
    // periodic threads and event handler
    RtThread.startMission();

    // do some non real-time work
    // and invoke sleep() or yield()
    for (;;) {
        watchdogBlink();
        Thread.sleep(500);
    }
\end{lstlisting}

\subsection{Implementation Results}

The initial idea was to implement scheduling and dispatching in
microcode. However, many Java bytecodes have a one to one mapping to
a microcode instruction, resulting in a single cycle execution. The
performance gain of an algorithm coded in microcode is therefore
negligible. As a result, almost all of the scheduling is implemented
in Java. Only a small part of the dispatcher, a memory copy, is
implemented in microcode and exposed with a special bytecode.

Experimental results of basic scheduling benchmarks, such as
periodic thread jitter, context switch time for threads and
asynchronous events, can be found in Section~\ref{subsec:rt:perf}.

To implement system functions, such as scheduling, in Java, access
to JVM and processor internal data structures have to be available.
However, Java does not allow memory access or access to hardware
devices. In JOP, this access is provided by way of additional
bytecodes. In the Java environment, these bytecodes are represented
as static native methods. The compiled invoke instruction for these
methods (\code{invokestatic}) is replaced by these additional
bytecodes in the class file. This solution provides a very efficient
way to incorporate low-level functions into a pure Java system. The
translation can be performed during class loading to avoid
non-standard class files.

A pure Java system, without an underlying RTOS, is an unusual system
with some interesting new properties. Java is a safer execution
environment than C (e.g.\ no pointers) and the boundary between
\emph{kernel} and \emph{user space} can become quite loose.
Scheduling, usually part of the operating system or the JVM, is
implemented in Java and executed in the same context as the
application. This property provides an easy path to a framework for
user-defined scheduling.
