
Several projects provide solutions to speedup execution of Java
programs in embedded systems. Two different approaches can be found
to improve Java bytecode execution by hardware. The first type
operates as a Java coprocessor in conjunction with a general-purpose
microprocessor. This coprocessor is placed in the instruction fetch
path of the main processor and translates Java bytecodes to sequences
of instructions for the host CPU or directly executes basic Java
bytecodes. The complex instructions are emulated by the main
processor. Java chips in the second category replace the
general-purpose CPU. All applications therefore have to be written in
Java. While the first type enables systems with mixed code
capabilities, the additional component significantly raises costs.
This chapter gives an overview of the most important Java processors
and coprocessors from academia and industry.
%
%\tablename~\ref{tab_related_proc} provides an overview of the
%described Java hardware.
%
%Blank fields in the table indicate that the information is not
%available or not applicable (e.g. for simulation-only projects).
%Minimum CPI is the number of clock cycles for a simple instruction
%such as \code{nop}. One entry, the TINI system, is not a real Java
%hardware, but is included in the table since it is often
%incorrectly\footnote{TINI is a standard interpreting JVM running on
%an enhanced 8051 processor.} cited as an embedded Java processor.
%
%
%
%\begin{table}
%    \centering
%{\footnotesize
%\begin{tabular}
%    {|>{\bfseries}p{1.6cm}|m{1.5cm}|>{\raggedright}m{1.6cm}|>{\raggedright}m{1.6cm}
%    |r|>{\raggedright}m{1.5cm}|r|}
%
%    \hline
%         & Type & Target  & Size & Speed & Java     & Min. \\
%         &      & technology &      & (MHz) & standard & CPI  \\
%    \hline
%    Hard-Int & Translation & Simulation only &  &  &  &  \\
%    \hline
%    DELFT & Translation & Simulation only &  &  &  &    \\
%    \hline
%    JIFFY & Translation & Xilinx FPGA & 3800 LCs, 1KB RAM &  &  &   \\
%    \hline
%    Jazelle & Co-processor & ASIC 0.18$\mu$ & 12K gates & 200 &  &  \\
%    \hline
%    JSTAR & Co-processor & ASIC 0.18$\mu$ Softcore & 30K gates + 7KB & 104 & J2ME CLDC\footnotemark[2] &  \\
%    \hline
%    TINI & Software JVM & Enhanced 8051 clone &  &  & Java 1.1 subset &   \\
%    \hline
%    picoJava & Processor & No realization & 128K gates + memory &  & Full & 1  \\
%    \hline
%    aJile & Processor & ASIC 0.25$\mu$ & 25K gates + ROM & 100 & J2ME CLDC\footnotemark[2] &   \\
%    \hline
%    Cjip & Processor & ASIC 0.35$\mu$ & 70K gates + ROM, RAM & 67 & J2ME CLDC\footnotemark[2] & 6 \\
%    \hline
%    Ignite & Stack processor & Xilinx FPGA & 9700 LCs &  &  &   \\
%    \hline
%    Moon & Processor & Altera FPGA & 3660 LCs, 4KB RAM &  &  &   \\
%    \hline
%    Lightfoot & Processor & Xilinx FPGA & 3400 LCs & 40 &  &   \\
%    \hline
%    LavaCORE & Processor & Xilinx FPGA & 3800 LCs 30K gates & 20 &  &   \\
%    \hline
%    Komodo & Processor & Xilinx FPGA & 2600 LCs & 20 & Subset: 50 bytecodes & 4  \\
%    \hline
%    FemtoJava & Processor & Altera Flex 10K & 2000 LCs & 4 & Subset: 69 bytecodes, 16-bit ALU & 3 \\
%    \hline
%    JSM \cite{JSM01} & Processor & Xilinx FPGA &  & 3.5 & Java Card &   \\
%    \hline
%%    \hline
%%    JOP & Processor & Altera, Xilinx FPGA & 2100 LCs + 3KB RAM & 100 & J2ME CLDC & 1 & Typical configuration on a Cyclone FPGA \\
%
%\end{tabular}
%}
%    \caption{Java hardware}
%    \label{tab_related_proc}
%\end{table}
%
%\footnotetext[2]{J2ME CLDC stands for Java2 Micro Edition, Connected
%Limited Device Configuration, which is described in
%Section~\ref{subsec:cldc}.}


% Change this: \emph{JOP is included with a typical configuration as a
% reference. Further details of the resource usage of JOP is described
% in Section~xxx.}


\section{Java Coprocessors}

The simplest enhancement for Java is a translation unit, which
substitutes the switch statement of an interpreter JVM (bytecode
decoding) through hardware and/or translates simple bytecodes
to a sequence of RISC instructions on the fly.

A standard JVM interpreter contains a loop with a large switch
statement that decodes the bytecode (see
Listing~\ref{lst:intro:java:intprt}). This switch statement is
compiled to an indirect branch. The destinations of these indirect
branches change frequently and do not benefit from branch-prediction
logic. This is the main overhead for simple bytecodes on modern
processors. The following approaches enhance the execution of Java
programs on a standard processor through the substitution of the
memory read and switch statement with bytecode fetch and decode
through hardware.



\subsection{Jazelle} \index{Java coprocessor!Jazelle}


Jazelle \cite{Jazelle} is an extension of the ARM 32-bit RISC
processor, similar to the Thumb state (a 16-bit mode for reduced
memory consumption). The Jazelle coprocessor is integrated into the
same chip as the ARM processor. The hardware bytecode decoder logic
is implemented in less than 12K gates. It accelerates, according to
ARM, some 95\% of the executed bytecodes. 140 bytecodes are executed
directly in hardware, while the remaining 94 are emulated by
sequences of ARM instructions. This solution also uses code
modification with \textit{quick} instructions to substitute certain
object-related instructions after link resolution. All Java
bytecodes, including the emulated sequences, are re-startable to
enable a fast interrupt response time.


A new ARM instruction puts the processor into the Java state.
Bytecodes are fetched and decoded in two stages, compared to a single
stage in ARM state. Four registers of the ARM core are used to cache
the top stack elements. Stack spill and fill is handled automatically
by the hardware. Additional registers are reused for the Java stack
pointer, the variable pointer, the constant pool pointer and locale
variable 0 (the \textit{this} pointer in methods). Keeping the
complete state of the Java mode in ARM registers simplifies its
integration into existing operating systems.


\section{Java Processors}

Java Processors are primarily used in an embedded system. In such a
system, Java is the native programming language and all operating
system related code, such as device drivers, are implemented in
Java. Java processors are simple or extended stack architectures
with an instruction set that resembles more or less the bytecodes
from the JVM.

\subsection{picoJava}
\label{subsec:related:picojava} \index{Java processor!picoJava}

Sun's picoJava is the Java processor used as a reference for new Java
processors and as the basis for research into improving various
aspects of a Java processor. Ironically, this processor was never
released as a product by Sun. After Sun decided to not produce
picoJava in silicon, Sun licensed picoJava to Fujitsu, IBM, LG
Semicon and NEC. However, these companies also did not produce a chip
and Sun finally provided the full Verilog code under an open-source
license.

Sun introduced the first version of picoJava \cite{pJ1} in 1997. The
processor was targeted at the embedded systems market as a pure Java
processor with restricted support of C. picoJava-I contains four
pipeline stages. A redesign followed in 1999, known as picoJava-II.
This is the version described below. picoJava-II was freely available
with a rich set of documentation \cite{pjMicroArch, pjProgRef}. The
probably first implementation of picoJava-II has been done by
Wolfgang Puffitsch \cite{master:puffitsch, pjfpga}. This
implementation enabled the comparison of JOP with picoJava-II in a
similar FPGA (see Chapter~\ref{chap:results})


The architecture of picoJava is a stack-based CISC processor
implementing 341 different instructions and is the most complex Java
processor available. The processor can be implemented in about 440K
gates \cite{Sekar2000}.
%
Simple Java bytecodes are directly implemented in hardware, most of
them execute in one to three cycles. Other performance critical
instructions, for instance invoking a method, are implemented in
microcode. picoJava traps on the remaining complex instructions, such
as creation of an object, and emulates this instruction. A trap is
rather expensive and has a minimum overhead of 16 clock cycles. This
minimum value can only be achieved if the trap table entry is in the
data cache and the first instruction of the trap routine is in the
instruction cache. The worst-case trap latency is 926 clock cycles
\cite{pjProgRef}. This great variation in execution times for a trap
hampers tight WCET estimates.

picoJava provides a 64-entry stack cache as a register file. The core
manages this register file as a circular buffer, with a pointer to
the top of stack. The stack management unit automatically performs
spill to and fill from the data cache to avoid overflow and underflow
of the stack buffer. To provide this functionality the register file
contains five memory ports. Computation needs two read ports and one
write port, the concurrent spill and fill operations the two
additional read and write ports. The processor core consists of
following six pipeline stages:
%
\begin{description}

\item[Fetch:]
Fetch 8 bytes from the instruction cache or 4 bytes from the bus
interface to the 16-byte-deep prefetch buffer.

\item[Decode:]
Group and precode instructions (up to 7 bytes) from the prefetch
buffer. Instruction folding is performed on up to four bytecodes.

\item[Register:]
Read up to two operands from the register file (stack cache).

\item[Execute:]
Execute simple instructions in one cycle or microcode for
multi-cycle instructions.

\item[Cache:]
Access the data cache.

\item[Writeback:]
Write the result back into the register file.

\end{description}
%
The integer unit together with the stack unit provides a mechanism,
called instruction folding, to speed up common code patterns found in
stack architectures. When all entries are contained in the stack
cache, the picoJava core can fold these four instructions into one
RISC-style single cycle operation.


\subsection{aJile JEMCore} \index{Java processor!aJile}

aJile's JEMCore is a direct-execution Java processor that is
available as both an IP core and a stand alone processor \cite{aJile,
aJile:paper}. It is based on the 32-bit JEM2 Java chip developed by
Rockwell-Collins. JEM2 is an enhanced version of JEM1, created in
1997 by the Rockwell-Collins Advanced Architecture Microprocessor
group. Rockwell-Collins originally developed JEM for avionics
applications by adapting an existing design for a stack-based
embedded processor. Rockwell-Collins decided not to sell the chip on
the open market. Instead, it licensed the design exclusively to aJile
Systems Inc., which was founded in 1999 by engineers from
Rockwell-Collins, Centaur Technologies, Sun Microsystems, and IDT.

The core contains 24 32-bit wide registers. Six of them are used to
cache the top elements of the stack. The datapath consists of a
32-bit ALU, a 32-bit barrel shifter and the support for floating
point operations (disassembly/assembly, overflow and NaN detection).
The control store is a 4K by 56 ROM to hold the microcode that
implements the Java bytecode. An additional RAM control store can be
used for custom instructions. This feature is used to implement the
basic synchronization and thread scheduling routines in microcode. It
results in low execution overhead with a thread-to-thread yield in
less than one $\mu$s (at 100~MHz). An optional Multiple JVM Manager
(MJM) supports two independent, memory protected JVMs. The two JVMs
execute time-sliced on the processor. According to aJile, the
processor can be implemented in 25K gates (without the microcode
ROM). The MJM needs additional 10K gates.

Two silicon versions of JEM exist today: the aJ-80 and the aJ-100.
Both versions comprise a JEM2 core, the MJM, 48~KB zero wait state
RAM and peripheral components, such as timer and UART. 16~KB of the
RAM is used for the writable control store. The remaining 32~KB is
used for storage of the processor stack. The aJ-100 provides a
generic 8-bit, 16-bit or 32-bit external bus interface, while the
aJ-80 only provides an 8-bit interface. The aJ-100 can be clocked up
to 100~MHz and the aJ-80 up to 66~MHz. The power consumption is about
1mW per MHz.

aJile was a member of the initial Real-Time for Java Expert Group.
However, up to now, no implementation of the RTSJ on top of the aJile
processor emerged. One nice feature of this processor is its
availability. Low-level access to devices via the RTSJ
\code{RawMemoryAccess} objects has been shown on the aJile
processor~\cite{conf/isorc/HardinFWB02}. A relatively cheap
development system, the JStamp \cite{JStamp}, was used to compare
this processor with JOP.

The aJile processor is intended as a solution for real-time systems.
However, no information is available about bytecode execution times.
As this processor is a commercial product and has been on the market
for some time, it is expected that its JVM implementation confirms to
Java standards, as defined by Sun.

\subsection{Cjip} \index{Java processor!Cjip}

The Cjip processor \cite{Imsys, Cjip} supports multiple instruction
sets, allowing Java, C, C++, and assembler to coexist. Internally,
the Cjip uses 72 bit wide microcode instructions, to support the
different instruction sets. At its core, Cjip is a 16-bit CISC
architecture with on-chip 36~KB ROM and 18~KB RAM for fixed and
loadable microcode. Another 1~KB RAM is used for eight independent
register banks, string buffer and two stack caches. Cjip is
implemented in 0.35-micron technology and can be clocked up to
66~MHz. The logic core consumes about 20\% of the
1.4-million-transistor chip. The Cjip has 40 program controlled I/O
pins, a high-speed 8 bit I/O bus with hardware DMA and an 8/16 bit
DRAM interface.


The JVM is implemented largely in microcode (about 88\% of the Java
bytecodes). Java thread scheduling and garbage collection are
implemented as processes in microcode. Microcode is also used to
implement virtual peripherals such as watchdog timers, display and
keyboard interfaces, sound generators, and multimedia codecs.

Microcode instructions execute in two or three cycles. A JVM bytecode
requires several microcode instructions. The Cjip Java instruction
set and the extensions are described in detail in \cite{CjipRef}. For
example: a bytecode \code{nop} executes in 6 cycles while an
\code{iadd} takes 12 cycles. Conditional bytecode branches are
executed in 33 to 36 cycles. Object oriented instructions, such
\code{getfield}, \code{putfield}, or \code{invokevirtual} are not
part of the instruction set.



\subsection{Lightfoot}

The Lightfoot 32-bit core \cite{Lightfoot} is a hybrid 8/32-bit
processor based on the Harvard architecture. Program memory is 8 bits
wide and data memory is 32 bits wide. The core contains a 3-stage
pipeline with an integer ALU, a barrel shifter, and a 2-bit multiply
step unit. There are two different stacks with the top elements
implemented as registers and memory extension. The data stack is used
to hold temporary data -- it is not used to implement the JVM stack
frame. As the name implies, the return stack holds return addresses
for subroutines and it can be used as an auxiliary stack. The
processor architecture specifies three different instruction formats:
soft bytecodes, non-returnable instructions, and single-byte
instructions that can be folded with a return instruction. The core
is available in VHDL and can be implemented in less than 30K gates.
Lightfood is now part of the VS2000 Typhoon Family
Microcontroller.\footnote{\url{http://www.velocitysemi.com/processors.htm}}

\subsection{LavaCORE}

LavaCORE \cite{LavaCORE} is another Java processor targeted at Xilinx
FPGA architectures.\footnote{\url{http://www.lavacore.com/}} It
implements a set of instructions in hardware and firmware.
Floating-point operations are not implemented. A 32x32-bit
dual-ported RAM implements a register-file. For specialized embedded
applications, a tool is provided to analyze which subset of the JVM
instructions is used. The unused instructions can be omitted from the
design. The core can be implemented in 1926 CLBs (= 3800 LCs) in a
Virtex-II (2V1000-5) and runs at 20~MHz.

\subsection{Komodo, jamuth} \index{Java processor!Komodo} \index{Java processor!jamuth}
\label{subsec:related:komodo}

Komodo \cite{Zulauf00} is a multithreaded Java processor with a
four-stage pipeline. It is intended as a basis for research on
real-time scheduling on a multithreaded microcontroller
\cite{komodo2003}. Simple bytecodes are directly implemented, while
more complex bytecodes, such as \code{iaload}, are implemented as a
microcode sequence. The unique feature of Komodo is the instruction
fetch unit with four independent program counters and status flags
for four threads. A priority manager is responsible for hardware
real-time scheduling and can select a new thread after each bytecode
instruction. The follow-up project, jamuth~\cite{jamuth:jtres07}, is
a commercial version of Komodo.

Komodo's multithreading is similar to hyper-threading in modern
processors that are trying to hide latencies in instruction fetching.
However, this feature leads to very pessimistic WCET values (in
effect rendering the performance gain useless). The fact that the
pipeline clock is only a quarter of the system clock also wastes a
considerable amount of potential performance.

\subsection{FemtoJava} \index{Java processor!FemtoJava}

FemtoJava \cite{Femto01} is a research project to build an
application specific Java processor. The bytecode usage of the
embedded application is analyzed and a customized version of
FemtoJava is generated. FemtoJava implements up to 69 bytecode
instructions for an 8 or 16 bit datapath. These instructions take 3,
4, 7 or 14 cycles to execute. Analysis of small applications (50 to
280 byte code) showed that between 22 and 69 distinct bytecodes are
used. The resulting resource usage of the FPGA varies between 1000
and 2000 LCs. With the reduction of the datapath to 16 bits the
processor is not Java conformant.

\subsection{jHISC} \index{Java processor!jHISC}

The jHISC project \cite{jHISC:jnl2006} proposes a high-level
instruction set architecture for Java. This project is closely
related to picoJava. The processor consumes 15500~LCs in an FPGA and
the maximum frequency in a Xilinx Virtex FPGA is 30~MHz.
%
%However, the resulting design is probably not very well balanced.
%The processor consumes 15500 LCs compared to about 3000 LCs for JOP.
%The maximum frequency in a Xilinx Virtex FPGA is 30 MHz compared to
%100 MHz for JOP.
%
According to \cite{jHISC:jnl2006} the prototype can only run simple
programs and the performance is estimated with a simulation. In
\cite{jHISC2006} the clocks per instruction (CPI) values for jHISC
are compared against picoJava and JOP. However, it is not explained
with which application the CPI values are collected. We assume that
the CPI values for picoJava and JOP are derived from the manual and
do not include any effects of pipeline stalls or cache misses.

\subsection{SHAP} \index{Java processor!SHAP}

The SHAP Java processor \cite{shap}, although now with a different
pipeline structure and hardware assisted garbage collection, has its
roots in the JOP design. SHAP is enhanced with a hardware object
manager. That unit redirects field and array access during a copy
operation of the GC unit. SHAP also implements the method cache
\cite{shap:mcache}.


\subsection{Azul} \index{Java processor!Azul}

Azul Systems provides an impressive multiprocessor system for
transactions oriented server workloads \cite{azul}. A single Vega
chip contains 54 64-bit RISC cores, optimized for the execution of
Java programs. Up to 16 Vega processors can be combined to a cache
coherent multiprocessor system with 864 processors cores, supporting
up to 768~GB of shared memory.


%\section{Derived Work}
%\label{sec:derived}
%
%Quite common for open-source projects are derived projects.
%Especially the research community appreciates open-source projects.
%Following list describes projects that are either completely based
%on JOP or influenced to a great extent.
%
%JOP triggered research on implementation of the JVM in hardware for
%real-time systems. The publications on JOP and also the fact that
%JOP is open-source made the project and ideas easy accessible for
%other researchers. Several research projects are directly or
%indirectly based on the research project JOP:
%
%\begin{itemize}
%    \item Lund -- Flavius
%    \item Dresden
%    \item Graz
%    \item Albertos MS thesis
%    \item \cite{conf/iscas/KoT07} JOP based dual-issue Javaprocessor
%    \item WCET work by Rasmus, Trevor, Elena, and upcoming CISS
%\end{itemize}
