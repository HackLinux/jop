
This section describes the design flow for JOP --- how to build the
Java processor and a Java application from scratch (the VHDL and
Java sources) and download the processor to an FPGA and the Java
application to the processor.


\section{Introduction}

JOP \cite{jop:thesis}, the Java optimized processor, is an
open-source development platform available for different targets
(Altera and Xilinx FPGAs and various types of FPGA boards). To
support several targets, the design-flow is a little bit complicated.
There is a \code{Makefile} available and when everything is set up
correctly, a simple
%
\begin{lstlisting}
    make
\end{lstlisting}
%
should build everything from the sources and download a \emph{Hello
World} example. However, to customize the \code{Makefile} for a
different target it is necessary to understand the complete design
flow. It should be noted that an
Ant\footnote{\url{http://ant.apache.org/}} based build process is
also available.

\subsection{Tools}

All needed tools are freely available.
%
\begin{itemize}
    \item  \href{http://java.sun.com/javase/downloads/index.jsp}%
{Java SE Development Kit (JDK)}  Java compiler and runtime
    \item  \href{http://www.cygwin.com/}%
{Cygwin} GNU tools for Windows. Packages cvs, gcc and make are
needed
    \item  \href{https://www.altera.com/support/software/download/altera_design/quartus_we/dnl-quartus_we.jsp}%
{Quarts II Web Edition} VHDL synthesis, place and route for Altera
FPGAs
%    \item  \href{https://www.altera.com/support/software/download/programming/jam/dnl-byte_code_player.jsp}%
%{Jam STAPL Byte-Code Player} FPGA configuration in batch mode
%(\cmd{jbi32.exe})

\end{itemize}
%
The PATH variable should contain entries to the executables of all
packages (java and javac, Cygwin bin, and Quartus executables). Check
the PATH at the command prompt with:
%
\begin{lstlisting}
    javac
    gcc
    make
    git
    quartus_map
\end{lstlisting}
%
All the executables should be found and usually report their usage.

\subsection{Getting Started}

\label{sec:started}

This section shows a quick step-by-step build of JOP for the Cyclone
target in the minimal configuration. All directory paths are given
relative to the JOP root directory \dirent{jop}. The build process is
explained in more detail in one of the following sections.

\subsubsection{Download the Source}

Create a working directory and download JOP from the GIT server:
\begin{lstlisting}
    git clone git://www.soc.tuwien.ac.at/jop.git
\end{lstlisting}
For a write access clone (for developers) use following URL:
\begin{lstlisting}
    git clone ssh://user@www.soc.tuwien.ac.at/home/git/jop.git
\end{lstlisting}
All sources are downloaded to a directory \dirent{jop}. For the
following command change to this directory. Create the needed
directories with:
\begin{lstlisting}
    make directories
\end{lstlisting}

\subsubsection{Tools}

The tools contain \cmd{Jopa}, the microcode assembler, \cmd{JopSim},
a Java based simulation of JOP, and \cmd{JOPizer}, the application
builder. The tools are built with following make command:

\begin{lstlisting}
    make tools
\end{lstlisting}

\subsubsection{Assemble the Microcode JVM, Compile the Processor}

The JVM configured to download the Java application from the serial
interface is built with:

\begin{lstlisting}
    make jopser
\end{lstlisting}

This command also invokes Quartus to build the processer. If you
want to build it within Quartus follow the following instructions:

\label{subsubsec:quartus}

\begin{enumerate}
  \item Start Quartus II and open the project \code{jop.qpf} from
      directory \dirent{quartus/cycmin} in Quartus with
      \menuitem{File -- Open Project...}.
  \item Start the compiler and fitter with \menuitem{Processing
      -- Start Compilation}.
  \item After successful compilation the FPGA is configured with
      \menuitem{Tools -- Programmer} and \menuitem{Start}.
\end{enumerate}



\subsubsection{Compiling and Downloading the Java Application}

A simple \emph{Hello World} application is the default application
in the Makefile. It is built and downloaded to JOP with:

\begin{lstlisting}
    make japp
\end{lstlisting}

The ``Hello World" message should be printed in the command window.

For a different application change the Makefile targets or override
the \code{make} variables at the command line. The following example
builds and runs some benchmarks on JOP:

\begin{lstlisting}
    make japp -e P1=bench P2=jbe P3=DoAll
\end{lstlisting}

The three variables \code{P1}, \code{P2}, and \code{P3} are a
shortcut to set the directory, the package name, and the main class
of the application.

\subsubsection{USB based Boards}

Several Altera based boards use an FTDI FT2232 USB chip for the FPGA
and Java program download. To change the download flow for those
boards change the value of the following variable in the Makefile to
\code{true}:

\begin{lstlisting}
    USB=true
\end{lstlisting}

The Java download channel is mapped to a virtual serial port on the
PC. Check the port number in the system properties and set the
variable \code{COM\_PORT} accordingly.

\subsection{Xilinx Spartan-3 Starter Kit}

\index{Xilinx} The Xilinx tool chain is still not well supported by
the Makefile or the Ant design flow. Here is a short list on how to
build JOP for a Xilinx board:

\begin{lstlisting}
    make tools
    cd asm
    jopser
    cd ..
\end{lstlisting}


Now start the Xilinx IDE wirh the project file \code{jop.npl}. It
will be converted to a new (binary) \code{jop.ise} project. The
\code{.npl} project file is used as it is simple to edit (ASCII).

\begin{itemize}
    \item Generate JOP by double clicking 'Generate PROM, ACE, or JTAG File'
    \item Configure the FPGA according to the board type
\end{itemize}

The above is a one step build for the processor. The Java
application is built and downloaded by:

\begin{lstlisting}
    make java_app
    make download
\end{lstlisting}

Now your first Java program runs on JOP/Spartan-3!

\section{Booting JOP --- How Your Application Starts}

Basically this is a two step process: (a) configuration of the FPGA
and (b) downloading the Java application. There are different
possibilities to perform these steps.

\subsection{FPGA Configuration}

FPGAs are usually SRAM based and \emph{lose} their configuration
after power down. Therefore the configuration has to be loaded on
power up. For development the FPGA can be configured via a download
cable (with JTAG commands). This can be done within the IDEs from
Altera and Xilinx or with command line tools such as
\cmd{quartus\_pgm} or \cmd{jbi32}.

For the device to boot automatically, the configuration has to be
stored in non volatile memory such as Flash. Serial Flash is directly
supported by an FPGA to boot on power up. Another method is to use a
standard parallel Flash to store the configuration and additional
data (e.g. the Java application). A small PLD reads the configuration
data from the Flash and shifts it into the FPGA. This method is used
on the Cyclone and ACEX boards.

\subsection{Java Download}

\index{application!download} When the FPGA is configured the Java
application has to be downloaded into the main memory. This download
is performed in microcode as part of the JVM startup sequence. The
application is a \code{.jop} file generated by \cmd{JOPizer}. At the
moment there are three options:

\begin{description}
    \item[Serial line] JOP listens to the serial line and the
        data is written into the main memory. A simple echo
        protocol performs the flow control. The baud rate is
        usually 115~kBaud.
    \item[USB] Similar to the serial line version, JOP listens to
        the parallel interface of the FTDI FT2232 USB chip. The
        FT2232 performs the flow control at the USB level and the
        echo protocol is omitted.
    \item[Flash] For stand alone applications the Java program is
    copied from the Flash (relative Flash address 0, mapped Flash
    address is 0x80000\footnote{All addresses in JOP are counted in
    32-bit quantities. However, the Flash is connected only to the
    lower 8 bits of the data bus. Therefore a store of one word in
    the main memory needs four loads from the Flash.}) to the main
    memory (usually a 32-bit SRAM).
\end{description}


The mode of downloading is defined in the JVM (\code{jvm.asm}). To
select a new mode, the JVM has to be assembled and the complete
processor has to be rebuilt -- a full \code{make} run. The generation
is performed by the C preprocessor (\cmd{gcc}) on \code{jvm.asm}. The
serial version is generated by default; the USB or Flash version are
generated by defining the preprocessor variables \code{USB} or
\code{FLASH}.

\paragraph{VHDL Simulation}

\index{simulation!VHDL}To speed up the VHDL simulation in ModelSim
there is a forth method where the Java application is loaded by the
test bench instead of JOP. This version is generated by defining
\code{SIMULATION}. The actual Java application is written by
\cmd{jop2dat} into a plain text file (\code{mem\_main.dat}) and read
by the simulation test bench into the simulated main memory.

There are four small batch-files in directory \dirent{asm} that
perform the JVM generation: \cmd{jopser}, \cmd{jopusb},
\cmd{jopflash}, and \cmd{jopsim}.

\subsection{Combinations}

Theoretically all variants to configure the FPGA can be combined with
all variations to download the Java application. However, only two
combinations are useful:

\begin{enumerate}
    \item For VHDL or Java development configure the FPGA
    via the download cable and download the Java application
    via the serial line or USB.
    \item For a stand-alone application load the configuration and
    the Java program from the Flash.
\end{enumerate}

\subsection{Stand Alone Configuration}

The Cycore board can be configured to configure the FPGA and load the
Java program from Flash at power up. In order to prepare the Cycore
board for this configuration the Flash must be programmed. Depending
on the I/O capabilities several options are possible:

\begin{description}
  \item[SLIP] With a SLIP connection the Flash can be programmed
      via TFTP. For this configuration a second serial line is
      needed.
  \item[Ethernet] With an Ethernet connection (e.g., the baseio
      board) TFTP can be used for Flash programming.
  \item[Serial Line] With a single serial line the utilities
      \code{util.Mem.java} and \code{amd.exe} can be used to
      program the Flash.
\end{description}

The following text describes the Flash programming and PLD
reconfiguration for a stand alone configuration. Fist we have to
build a JOP version that will load a Java program from the Flash:
\begin{lstlisting}
    make jopflash
\end{lstlisting}
As usual a \code{jop.sof} file will be generated. For easier reading
of the configuration it will be converted to \code{jop.ttf}. This
file will be programmed into the Flash starting at address 0x40000.
Therefore, we need to save that file and rebuild a JOP version that
loads a Java program (the Flash programmer) from the serial line:
\begin{lstlisting}
    copy quartus\cycmin\jop.ttf ttf\cycmin.ttf
    make jopser
\end{lstlisting}
As a next step we will build the Java program that will be programmed
into the Flash and save a copy of the \code{.jop} file.
\code{Hello.java} is the embedded version of a \emph{Hello World}
program that blinks the WD LED at 1~Hz.
\begin{lstlisting}
    make java_app -e P1=test P2=test P3=Hello
    copy java\target\dist\bin\Hello.jop .
\end{lstlisting}
To program the Flash the programmer tool \code{util.Mem} will run on
JOP and \code{amd.exe} is used at the PC side:
\begin{lstlisting}
    make japp -e P1=common P2=util P3=Mem COM_FLAG=
    amd Hello.jop COM1
    amd ttf\cycmin.ttf COM1
\end{lstlisting}
As a last step the PLD will be programmed to enable FPGA
configuration form the Flash:
\begin{lstlisting}
    make pld_conf
\end{lstlisting}
The board shall now boot after a power cycle and the LED will blink.
To read the output from the serial line the small utility
\code{e.exe} can be used.


In the case the PLD configuration shall be changed back to JTAG FPGA
configuration following make command will reset the PLD:
\begin{lstlisting}
    make pld_init
\end{lstlisting}

Note, that in a stand alone configuration the watchdog (WD) pin has
to be toggled every second (e.g., by invoking \code{util.Timer.wd()}.
When the WD is not toggled the FPGA will be reconfigured after 1.6
seconds.

Due to wrong file permissions the Windows executables \code{amd.exe}
and \code{USBRunner.exe} will not have the execution permission set.
Change the setting with the Windows Explorer. The tool \code{amd.exe}
can also be rebuilt with:
\begin{lstlisting}
    make cprog
\end{lstlisting}


\section{The Design Flow}

This section describes the design flow to build JOP in greater
detail.

\subsection{Tools}

There are a few tools necessary to build and download JOP to the FPGA
boards. Most of them are written in Java. Only the tools that access
the serial line are written in C.\footnote{The Java JDK still comes
without the javax.comm package and getting this optional package
correctly installed is not that easy.}

\subsubsection{Downloading}

These little programs are already compiled and the binaries are
checked in into the repository. The sources can be found in directory
\dirent{c\_src}.

\begin{description}
    \item[\eei{down.exe}] The workhorse to download Java programs. The
    mandatory argument is the COM-port. Optional switch \code{-e}
    keeps the program running after the download and echoes the
    characters from the serial line (\code{System.out} in JOP) to
    stdout. Switch \code{-usb} disables the echo protocol to speed up the
    download over USB.
    \item[\eei{e.exe}] Echoes the characters from the serial line
        to stdout. Parameter is the COM-port.
    \item[\eei{amd.exe}] A utility to send data over the serial
        line to program the on-board Flash. The complementary
        Java program \code{util.Mem} must be running on JOP.
    \item[\eei{USBRunner.exe}] Download the FPGA configuration via
    USB with the FTDI2232C chip (dpsio board).
\end{description}

\subsubsection{Generation of Files}

These tools are written in Java and are delivered in source form.
The source can be found under \dirent{java/tools/src} and the class
files are in \code{jop-tools.jar} in directory
\dirent{java/tools/dist/lib}.

\begin{description}
    \item[\eei{Jopa}] The JOP assembler. Assembles the microcoded
    JVM and produces on-chip memory initialization files and VHDL
    files.
    \item[\eei{BlockGen}] converts Altera memory initialization
        files to VHDL files for a Xilinx FPGA.
    \item[\eei{JOPizer}] links a Java application and converts the
    class information to the format that JOP expects (a \code{.jop} file).
    \cmd{JOPizer} uses the bytecode engineering library\footnote{\url{http://jakarta.apache.org/bcel/}} (BCEL).

\end{description}

\subsubsection{Simulation}

\index{simulation!JopSim}
\begin{description}
    \item[\eei{JopSim}] reads a \code{.jop} file and executes it in
    a debug JVM written in Java. Command line option
    \code{-Dlog="true"} prints a log entry for each executed JVM
    bytecode.
    \item[\eei{pcsim}] simulates the BaseIO expansion board for Java
    debugging on a PC (using the JVM on the PC).
\end{description}

\subsection{Targets}

JOP has been successfully ported to several different FPGAs and
boards. The main distribution contains the ports for the FPGAs:

\begin{itemize}
    \item Altera Cyclone EP1C6 or EP1C12
    \item Xilinx Spartan-3
    \item Altera Cyclone-II (Altera DE2 board)
    \item Xilinx Virtex-4 (ML40x board)
    \item Xilinx Spartan-3E (Digilent Nexys 2 board)
\end{itemize}

For the current list of the supported FPGA boards see the list at the
web site.\footnote{\url{http://www.jopwiki.com/FPGA_boards}} Besides
the ports to different FPGAs there are ports to different boards.

\subsubsection{Cyclone EP1C6/12}

This board is the workhorse for the JOP development and comes in two
versions: with an Cyclone EP1C6 or EP1C12. The schematics can be
found in Appendix~\ref{appx:cycore}. The board contains:

\begin{itemize}
    \item Altera Cyclone EP1C6Q240 or EP1C12Q240 FPGA
    \item 1~MB fast SRAM
    \item 512~KB Flash (for FPGA configuration and program code)
    \item 32~MB NAND Flash
    \item ByteBlasterMV port
    \item Watchdog with LED
    \item EPM7064 PLD to configure the FPGA from the Flash (on watchdog reset)
    \item Voltage regulator (1V5)
    \item Crystal clock (20 MHz) at the PLL input (up to 640 MHz internal)
    \item Serial interface (MAX3232)
    \item 56 general purpose I/O pins
\end{itemize}

The Cyclone specific files are \code{jopcyc.vhd} or \code{jopcyc12}
and \code{mem32.vhd}. This FPGA board is designed as a module to be
integrated with an application specific I/O-board. There exist
following I/O-boards:
%
\begin{description}
    \item[simpexp] A simple bread board with a voltage regulator and
    a SUBD connector for the serial line
    \item[baseio] A board with Ethernet connection and EMC
        protected digital I/O and analog input
    \item[bg263] Interface to a GPS receiver, a GPRS modem, keyboard
    and a display for a railway application
    \item[lego] Interface to the sensors and motors of the LEGO
    Mindstorms. This board is a substitute for the LEGO RCX.
    \item[dspio] Developed at the University of Technology Vienna, Austria for
    digital signal processing related work. All design files for this
    board are open-source.
\end{description}
%
Table~\ref{tab:cycio} lists the related VHDL files and Quartus
project directories for each I/O board.

\begin{table}
    \centering

    \begin{tabular}{lll}
        \toprule
        I/O board & Quartus & I/O top level \\
        \midrule
        simpexp, baseio  & \dirent{cycmin} & \code{scio\_min.vhd} \\
        dspio  & \dirent{usbmin} & \code{scio\_dspiomin.vhd} \\
        baseio  & \dirent{cycbaseio} & \code{scio\_baseio.vhd} \\
        bg263  & \dirent{cybg} & \code{scio\_bg.vhd} \\
        lego  & \dirent{cyclego} & \code{scio\_lego.vhd} \\
        dspio  & \dirent{dspio} & \code{scio\_dspio.vhd} \\
        \bottomrule

    \end{tabular}
    \caption{Quartus project directories and VHDL files for the different I/O boards}
    \label{tab:cycio}

\end{table}


\subsubsection{Xilinx Spartan-3}

\index{Xilinx} The Spartan-3 specific files are \code{jop\_xs3.vhd}
and \code{mem\_xs3.vhd} for the Xilinx Spartan-3 Starter Kit and
\code{jop\_trenz.vhd} and \code{mem\_trenz.vhd} for the Trenz
Retrocomputing board.


\section{Eclipse}

In folder \dirent{eclipse} there are four Eclipse projects that you
can import into your Eclipse workspace. However, do not use
\emph{that} directory as your workspace directory. Choose a directory
outside of the JOP source tree for the workspace (e.g., your usual
Eclipse workspace) and copy the for project folders
\dirent{joptarget}, \dirent{joptools}, \dirent{pc}, and
\dirent{pcsim}.

All projects use the Eclipse path variable\footnote{Eclipse (path)
variables are workspace specific.} \code{JOP\_HOME} that has to point
to the root directory (\dirent{.../jop}) of the JOP sources. Under
\menuitem{Window -- Preferences...} select \menuitem{General --
Workspace -- Linked Resources} and create the path variable
\code{JOP\_HOME} with \menuitem{New...}.

Import the projects with \menuitem{File -- Import..} and
\menuitem{Existing Projects into Workspace}. It is suggested to an
Eclipse workspace that is not part of the jop source tree. Select as
the root directory (e.g., your Eclipse workspace), select the
projects you want to import, select \menuitem{Copy projects into
workspace}, and press \menuitem{Finish}. Table~\ref{tab:eclipse}
shows all available projects.

\begin{table}
    \centering

    \begin{tabular}{ll}
        \toprule
        Project & Content \\
        \midrule
        \dirent{jop} & The target sources \\
        \dirent{joptools} & Tools such as \code{Jopa}, \code{JopSim}, and \code{JOPizer} \\
        \dirent{pc} & Some PC utilities (e.g.\ Flash programming via UDP/IP) \\
        \dirent{pcsim} & Simulation of the basio hardware on the PC \\
        \bottomrule

    \end{tabular}
    \caption{Eclipse projects}
    \label{tab:eclipse}

\end{table}

Add the libraries from \dirent{.../jop/java/lib} (as external
archives) to the build path (right click on the \dirent{joptools}
project) of the project \dirent{joptools}.\footnote{Eclipse can't use
path variables for external .jar files.}

\section{Simulation}

This section contains the information you need to get a simulation
of JOP running. There are two ways to simulate JOP:
%
{\samepage
\begin{itemize}
    \item High-level JVM simulation with \cmd{JopSim}
    \item VHDL simulation (e.g. with ModelSim)
\end{itemize}
}
%

\subsection{JopSim Simulation}

\index{simulation!JopSim}

The high level simulation with \cmd{JopSim} is a simple JVM written
in Java that can execute the JOP specific application (the
\code{.jop} file). It is started with:
\begin{lstlisting}
    make jsim
\end{lstlisting}

To output each executing bytecode during the simulation run change in
the Makefile the logging parameter to \code{-Dlog="true"}.


\subsection{VHDL Simulation}

\index{simulation!VHDL}

This section is about running a VHDL simulation with ModelSim. All
simulation files are vendor independent and should run on any
versions of ModelSim or a different VHDL simulator. You can simulate
JOP even with the free ModelSim XE II Starter Xilinx version, the
ModelSim Altera version or the ModelSim Actel version.

To simulate JOP, or any other processor design, in a vendor neutral
way, models of the internal memories (block RAM) and the external
main memory are necessary. Beside this, only a simple clock driver is
necessary. To speed-up the simulation a little bit, a simulation of
the UART output, which is used for \code{System.out.print()}, is also
part of the package.

Table~\ref{tab:simfiles} lists the simulation files for JOP and the
programs that generates the initialization data. The non-generated
VHDL files can be found in directory \dirent{vhdl/simulation}.
%
\begin{table}
\small
    \centering

    \begin{tabular}{llll}
        \toprule
        VHDL file & Function & Initialization file & Generator \\
        \midrule
        \code{sim\_jop\_types\_100.vhd} & JOP constant definitions & - & - \\
        \code{sim\_rom.vhd} & JVM microcode ROM & \code{mem\_rom.dat} & \cmd{Jopa} \\
        \code{sim\_ram.vhd} & Stack RAM & \code{mem\_ram.dat} & \cmd{Jopa} \\
        \code{sim\_jbc.vhd} & Bytecode memory (cache) & - & - \\
        \code{sim\_memory.vhd} & Main memory & \code{mem\_main.dat} & \cmd{jop2dat} \\
        \code{sim\_pll.vhd} & A dummy entity for the PLL & - & - \\
        \code{sim\_uart.vhd} & Print characters to stdio & - & - \\
        \bottomrule

    \end{tabular}
    \caption{Simulation specific VHDL files}
    \label{tab:simfiles}

\end{table}
%
The needed VHDL files and the compile order can be found in
\code{sim.bat} under \dirent{modelsim}.


The actual version of JOP contains all necessary files to run a
simulation with ModelSim. In directory \dirent{vhdl/simulation} you
will find:
%
\begin{itemize}
    \item A test bench: \code{tb\_jop.vhd} with a serial receiver to
    print out the messages from JOP during the simulation
    \item Simulation versions of all memory components (vendor neutral)
    \item Simulation of the main memory
\end{itemize}
%
\cmd{Jopa} generates various \code{mem\_xxx.dat} files that are read
by the simulation. The JVM that is generated with \code{jopsim.bat}
assumes that the Java application is preloaded in the main memory.
\cmd{jop2dat} generates a memory initialization file from the Java
application file (\code{MainClass.jop}) that is read by the
simulation of the main memory (\code{sim\_memory.vhd}).

In directory \dirent{modelsim} you will find a small batch file
(\cmd{sim.bat}) that compiles JOP and the test bench in the correct
order and starts ModelSim. The whole simulation process (including
generation of the correct microcode) is started with:

\begin{lstlisting}
    make sim
\end{lstlisting}

After a few seconds you should see the startup message from JOP
printed in ModelSim's command window. The simulation can be continued
with \cmd{run -all} and after around 6~ms \emph{simulation time} the
actual Java \code{main()} method is executed. During those 6~ms,
which will probably be minutes of simulation, the memory is
initialized for the garbage collector.

\section{Files Types You Might Encounter}

As there are various tools involved in the complete build process,
you will find files with various extensions. The following list
explains the file types you might encounter when changing and
building JOP.

The following files are the \emph{source} files:

\begin{description}

\item[\eei{.vhd}] VHDL files describe the hardware part and are
compiled with either Quartus or Xilinx ISE. Simulation in ModelSim
is also based on VHDL files.
\item[\eei{.v}] Verilog HDL. Another hardware description language.
Used more in the US.

\item[\eei{.java}] Java --- the language that runs native on JOP.

\item[\eei{.c}] There are still some tools written in C.

\item[\eei{.asm}] JOP microcode. The JVM is written in this stack
oriented assembler. Files are assembled with \cmd{Jopa}. The result
are VHDL files, \code{.mif} files, and \code{.dat} files for
ModelSim.

\item[\eei{.bat}] Usage of these DOS batch files still prohibit
running the JOP build under Unix. However, these files get less used
as the \code{Makefile} progresses.

\item[\eei{.xml}] Project files for Ant. Ant is an attractive
substitution to \cmd{make}. Future distributions on JOP will be
\cmd{ant} based.

\end{description}


Quartus II and Xilinx ISE need configuration files that describe
your project. All files are usually ASCII text files.

\begin{description}

\item[\eei{.qpf}] Quartus II Project File. Contains almost no
information.
\item[\eei{.qsf}] Quartus II Settings File defines the project.
    VHDL files that make up your project are listed. Constraints
    such as pin assignments and timing constraints are set here.
\item[\eei{.cdf}] Chain Description File. This file
stores device name, device order, and programming file name
information for the programmer.
\item[\eei{.tcl}] Tool Command Language. Can be used in Quartus to
automate parts of the design flow (e.g. pin assignment).

\item[\eei{.npl}] Xilinx ISE project. VHDL files that make up
    your project are listed. The actual version of Xilinx ISE
    converts this project file to a new format that is not in
    ASCII anymore.
\item[\eei{.ucf}] Xilinx Foundation User Constraint File.
    Constraints such as pin assignments and timing constraints
    are set here.

\end{description}

The Java tools \cmd{javac} and \cmd{jar} produce following file types
from the Java sources:

\begin{description}

\item[\eei{.class}] A class file contains the bytecodes, a symbol table and other
ancillary information and is executed by the JVM.

\item[\eei{.jar}] The Java Archive file format enables you to bundle multiple files
into a single archive file. Typically a \code{.jar} file contains
the class files and auxiliary resources. A \code{.jar} file is
essentially a zip file that contains an optional \dirent{META-INF}
directory.

\end{description}

The following files are generated by the various tools from the
source files:

\begin{description}

\item[\eei{.jop}] This file makes up the linked Java application
    that runns on JOP. It is generated by \cmd{JOPizer} and can
    be either downloaded (serial line or USB) or stored in the
    Flash (or used by the simulation with \cmd{JopSim} or
    ModelSim)

\item[\eei{.mif}] Memory Initialization File. Defines the initial
content of on-chip block memories for Altera devices.

\item[\eei{.dat}] memory initialization files for the simulation
with ModelSim.

\item[\eei{.sof}] SRAM Output File. Configuration file for Altera
devices. Used by the Quartus programmer or by \cmd{quartus\_pgm}.
Can be converted to various (or too many) different format. Some are
listed below.

\item[\eei{.pof}] Programmer Object File. Configuration for Altera
devices. Used for the Flash loader PLDs.

\item[\eei{.jbc}] JamTM STAPL Byte Code 2.0. Configuration for Altera
devices. Input file for \cmd{jbi32}.

\item[\eei{.ttf}] Tabular Text File. Configuration for Altera
devices. Used by flash programming utilities (\cmd{amd} and
\cmd{udp.Flash} to store the FPGA configuration in the boards Flash.

\item[\eei{.rbf}] Raw Binary File. Configuration for Altera
devices. Used by the USB download utility (\cmd{USBRunner}) to
configure the dspio board via the USB connection.

\item[\eei{.bit}] Bitstream File. Configuration file for Xilinx
devices.

\end{description}

\section{Information on the Web}

Further information on JOP and the build process can be found on the
Internet at the following places:

\begin{itemize}
    \item \url{http://www.jopdesign.com/} is the main web site
        for JOP
    \item \url{http://www.jopwiki.com/} is a Wiki that can be
        freely edited by JOP users.
    \item
        \url{http://tech.groups.yahoo.com/group/java-processor/}
        hosts a mailing list for discussions on Java processors
        in general and mostly on JOP related topics
\end{itemize}


\section{Porting JOP}

Porting JOP to a different FPGA platform or board usually consists
of adapting pin definitions and selection of the correct memory
interface. Memory interfaces for the SimpCon interconnect can be
found in directory \dirent{vhdl/memory}.

\subsection{Test Utilities}

To verify that the port of JOP is successful there are some small
test programs in \dirent{asm/src}. To run the JVM on JOP the
microcode \code{jvm.asm} is assembled and will be stored in an
on-chip ROM. The Java application will then be loaded by the first
microcode instructions in \code{jvm.asm} into an external memory.
However, to verify that JOP and the serial line are working
correctly, it is possible to run small test programs directly in
microcode.

One test program (\code{blink.asm}) does not need the main memory and
is a first test step before testing the possibly changed memory
interface. \code{testmon.asm} can be used to debug the main memory
interface. Both test programs can be built with the \cmd{make}
targets \cmd{jop\_blink\_test} and \cmd{jop\_testmon}.

\subsubsection{Blinking LED and UART output}

The test is built with:
%
\begin{lstlisting}
    make jop_blink_test
\end{lstlisting}
%
After download, the watchdog LED should blink and the FPGA will print
out 0 and 1 on the serial line. Use a terminal program or the utility
\cmd{e.exe} to check the output from the serial line.

\subsubsection{Test Monitor}

Start a terminal program (e.g. HyperTerm) to communicate with the
monitor program and build the test monitor with:
%
\begin{lstlisting}
    make jop_testmon
\end{lstlisting}
%
After download the program prints the content of the memory at
address 0. The program understands following \emph{commands}:

\begin{itemize}
    \item A single CR reads the memory at the current addres
    and prints out the address and memory content
    \item \code{addr=val;} writes $val$ into the memory location at
    address $addr$
\end{itemize}

One tip: Take care that your terminal program does not send an LF
after the CR.


\section{Extending JOP}

%% Trevor: This section seems like esoteric information that might be better as an appendix.

JOP is a soft-core processor and customizing it for an application
is an interesting opportunity.

\subsection{Native Methods}

\index{native method} The \emph{native} language of JOP is microcode.
A native method is implemented in JOP microcode. The interface to
this native method is through a \emph{special} bytecode. The mapping
between native methods and the special bytecode is performed by
\code{JOPizer}. When adding a new (\emph{special}) bytecode to JOP,
the following files have to be changed:
\begin{enumerate}
    \item \code{jvm.asm} implementation
    \item \code{Native.java} method signature
    \item \code{JopInstr.java} mapping of the signature to the name
    \item \code{JopSim.java} simulation of the bytecode
    \item \code{JVM.java} (just rename the method name)
    \item \code{Startup.java} (only when needed in a class
        initializer)
    \item \code{WCETInstruction.java} timing information
\end{enumerate}

First implement the native code in \code{JopSim.java} for easy
debugging. The \emph{real} microcode is added in \code{jvm.asm} with
a label for the special byctecode. The naming convention is
\code{jopsys\_name}. In \code{Native.java} provide a method
signature for the native method and enter the mapping between this
signature and the name in \code{jvm.asm} and in
\code{JopInstr.java}. Provide the execution time in
\code{WCETInstruction.java} for the WCET analysis.

The native method is accessed by the method provided in
\code{Native.java}. There is no calling overhead involved in the
mechanism. The \emph{native} method gets substituted by
\cmd{JOPizer} with a \emph{special} bytecode.

\subsection{A new Peripheral Device}

Creation of a new peripheral devices involves some VHDL coding.
However, there are several examples in \dirent{jop/vhdl/scio}
available.

All peripheral components in JOP are connected with the SimpCon
\cite{simpcon} interface. For a device that implements the Wishbone
\cite{soc:wishbone} bus, a SimpCon-Wishbone bridge (\code{sc2wb.vhd})
is available (e.g., it is used to connect the AC97 interface in the
\code{dspio} project).

For an easy start use an existing example and change it to your
needs. Take a look into \code{sc\_test\_slave.vhd}. All peripheral
components (SimpCon slaves) are connected in one module usually named
\code{scio\_xxx.vhd}. Browse the examples and copy one that best fits
your needs. In this module the address of your peripheral device is
defined (e.g. 0x10 for the primary UART). This I/O address is mapped
to a negative memory address for JOP. That means 0xffffff80 is added
as a base to the I/O address.

By convention this address mapping is defined in
\code{com.jopdesign.sys.Const}. Here is the UART example:

\begin{lstlisting}
    // use negative base address for fast constant load
    // with bipush
    public static final int IO_BASE = 0xffffff80;
    ...
    public static final int IO_STATUS = IO_BASE+0x10;
    public static final int IO_UART = IO_BASE+0x10+1;
\end{lstlisting}

The I/O devices are accessed from Java by
\emph{native}\footnote{These are not real functions and are
substituted by special bytecodes on application building with
JOPizer.} functions: \code{Native.rdMem()} and \code{Native.wrMem()}
in pacakge \code{com.jopdesign.sys}. Again an example with the UART:

\begin{lstlisting}
    // busy wait on free tx buffer
    // no wait on an open serial line, just wait
    // on the baud rate
    while ((Native.rdMem(Const.IO_STATUS)&1)==0) {
        ;
    }
    Native.wrMem(c, Const.IO_UART);
\end{lstlisting}

Best practise is to create a new I/O configuration
\code{scio\_xxx.vhdl} and a new Quartus project for this
configuration. This avoids the mixup of the changes with a new
version of JOP. For the new Quartus project only the three files
\code{jop.cdf}, \code{jop.qpf}, and \code{jop.qsf} have to be copied
in a new directory under \dirent{quartus}. This new directory is the
project name that has to be set in the Makefile:

\begin{lstlisting}
    QPROJ=yourproject
\end{lstlisting}

The new VHDL module and the \code{scio\_xxx.vhdl} are added in
\code{jop.qsf}. This file is a plain ASCII file and can be edited
with a standard editor or within Quartus.

\subsection{A Customized Instruction}

A customized instruction can be simply added by implementing it in
microcode and mapping it to a native function as described before. If
you want to include a hardware module that implements this
instruction a new microinstruction has to be introduced. Besides
mapping this instruction to a native method the instruction has also
be added to the microcode assembler \cmd{Jopa}.

\subsection{Dependencies and Configurations}

As JOP and the JVM are a mix of VHDL and Java files, of changes some
configurations or changes in central data structures needs an update
in several files.

\subsubsection{Speed Configuration}

By default, JOP is configured for 80 Mhz. To build the 100 MHz
configuration, edit \code{quartus/cycmin/jop.qsf} and change
\code{jop\_config\_80} to \code{jop\_config\_100}.

\subsubsection{Method Cache Configuration}

The default configuration (for the Altera Cyclone) is a 4~KB method
cache configured with 16 blocks (i.e., a variable block cache). For
the Xilinx targets, the cache size is 2KB because Xilinx does not (or
did not) support easily configurable block RAMs.

To change from a variable block cache to a dual-block cache, you will
need to edit the top-level VHDL. Here is an example from
\code{vhdl/top/jopcyc.vhd}:

{\small
\begin{lstlisting}
entity jop is

generic (
   ram_cnt  : integer := 2;  -- clock cycles for external ram
-- rom_cnt  : integer := 3;  -- clock cycles for external rom OK for 20 MHz
   rom_cnt  : integer := 15; -- clock cycles for external rom for 100 MHz
   jpc_width : integer := 12; -- address bits of java bytecode pc = cache size
   block_bits : integer := 4  -- 2*block_bits is number of cache blocks
);
\end{lstlisting}
}

The power of 2 of the \code{jpc\_width} is the cache size, and the
power of 2 of the \code{block\_bits} is the number of blocks. To
simulate a dual block cache, \code{block\_bits} has to be set to 1.
(To use a single block cache, \code{cache.vhd} has to be modified to
force a miss at the cache lookup.)

\subsubsection{Stack Size}

The on-chip stack size can be configured by changing following
constants:

\begin{itemize}
    \item \code{ram\_width} in \code{jop\_config\_xx.vhd}
    \item \code{STACK\_SIZE} in \code{com.jopdesign.sys.Const}
    \item \code{RAM\_LEN} in \code{com.jopdesign.sys.Jopa}
\end{itemize}

\subsubsection{Changing the Class Format}

The constants for the offsets of fields in the class format are found
in:

\begin{itemize}
    \item \code{JOPizer}: \code{CLS\_HEAD, dump()}
    \item \code{GC.java} uses \code{CLASS\_HEADR}
    \item \code{JMV.java} uses \code{CLASS\_HEADR} + offset
        (\code{checkcast}, \code{instanceof})
\end{itemize}




%\section{Notes}
%
%TODO:
%
%\begin{itemize}
%    \item pcsim
%    \item JopSim
%\end{itemize}
%Formating ideas -- see Latax intro p 27
%\url{http://www.ctan.org/tex-archive/info/lshort/english/lshort.pdf}\\
