
This document describes the design flow for JOP --- how to build the
Java processor and a Java application from scratch (the VHDL and
Java sources) and download the processor to an FPGA and the Java
application to the processor.


\section{Introduction}

JOP \cite{jop:thesis}, the Java optimized processor, is an
open-source development available for different targets (Altera and
Xilinx FPGAs and various types of FPGA boards). To support several
targets the design-flow gets a little bit complicated. There is a
\code{Makefile} available and when everything is set up correct a
simple
%
\begin{verbatim}
    make
\end{verbatim}
%
should build everything from the sources and download a \emph{Hello
World} example. However, to costumize the \code{Makefile} for a
different target it is necessary to understand the complete design
flow.

\subsection{Tools}

All needed tools are freely available.
%
\begin{itemize}
    \item  \href{http://java.sun.com/j2se/1.4.2/download.html}%
{Java 2 Platform 1.4, Standard Edition}  Java compiler and runtime
    \item  \href{http://www.cygwin.com/}%
{Cygwin} GNU tools for Windows. Packages cvs, gcc and make are
needed
    \item  \href{https://www.altera.com/support/software/download/altera_design/quartus_we/dnl-quartus_we.jsp}%
{Quarts II Web Edition} VHDL synthesis, place and route for Altera
FPGAs
    \item  \href{https://www.altera.com/support/software/download/programming/jam/dnl-byte_code_player.jsp}%
{Jam STAPL Byte-Code Player} FPGA configuration in batch mode
(\cmd{jbi32.exe})

\end{itemize}
%
The PATH variable should contain entries to the executables of all
packages (java and javac, Cygwin bin, Quartus executables and
jbi32). Check the PATH at the command prompt with:
%
\begin{verbatim}
    javac
    gcc
    make
    cvs
    quartus_map
    jbi32
\end{verbatim}
%
All the executables should be found and usually report their usage.

\subsection{Getting Started}

This sections shows a quick step-by-step build of JOP for the
Cyclone target in the minimal configuration. All directory paths are
given relative to the JOP root directory \dirent{jop}. The build
process is explained in more detail in one of the following
sections.

\subsubsection{Download the Source}

Create a working directory and download JOP from the
\url{www.opencores.org} CVS server:

\begin{verbatim}
 cvs -d :pserver:anonymous@cvs.opencores.org:/cvsroot/anonymous -z9 co -P jop
\end{verbatim}

All sources are downloaded to a directory \dirent{jop}. For the
following command change to this directory. Create the needed
directories with:
\begin{verbatim}
    make directories
\end{verbatim}

\subsubsection{Tools}

The tools are built with following make command:

\begin{verbatim}
    make tools
\end{verbatim}

\subsubsection{Assemble the Microcode JVM, Compile the Processor}

The JVM configured to download the Java application from the serial
interface is built with:

\begin{verbatim}
    make jopser
\end{verbatim}

This command also invokes Quartus to build the processer. If you
want to build it within Quartus follow the following instructions:

\label{subsubsec:quartus}

Start Quartus II and open the project \code{jop.qpf} from directory
\dirent{quartus/cycmin} in Quartus with \menuitem{File -- Open
Project...}. Start the compiler and fitter with \menuitem{Processing
-- Start Compilation}. After successful compilation the FPGA is
configured with \menuitem{Tools -- Programmer} and \menuitem{Start}.

\subsubsection{Compiling and Downloading the Java Application}

A simple \emph{Hello World} application is the default application
in the Makefile. It is built and downloaded to JOP with:

\begin{verbatim}
    make japp
\end{verbatim}

The "Hello World" message should be printed in the command window
and the LED on the board should blink.

For a different application change the Makefile targets or override
the \code{make} variables at the command line. Following example
builds and runs some benchmarks on JOP:

\begin{verbatim}
    make japp -e P1=bench P2=jbe P3=DoAll
\end{verbatim}



\subsection{Xilinx Spartan-3 Starter Kit}

For the Xilinx starter kit the design flow is very similar:

\begin{enumerate}
    \item Compile the ISE project under \dirent{xilinx/s3sk}
    \item Download JOP to the FPGA
    \item Connect a serial cable from the board to your PC
    \item Open a command prompt in \dirent{java/target}
    \item Change the COM-port in \cmd{doit.bat} and disable the line
    with \cmd{jbi32.exe}
    \item type: \code{doit test test Hello}
\end{enumerate}

Now your first Java program runs on JOP/Spartan-3!

\section{Booting JOP --- How Your Application Starts}

Basically this is a two step process: a.) configure the FPGA and b.)
load the Java application. There are different ways to perform these
steps.

\subsection{FPGA Configuration}

FPGAs are usually SRAM based and \emph{lose} their configuration
after power down. Therefore the configuration has to be loaded on
power up. For development the FPGA can be configured via a download
cable (with JTAG commands). This can be done within the IDEs from
Altera and Xilinx or with command line tools such as
\cmd{quartus\_pgm} or \cmd{jbi32}.

When the device shall boot automatically the configuration has to be
stored in non volatile memory such as Flash. Serial Flash is
directly supported by an FPGA to boot on power up. Another method is
to use a standard parallel Flash to store the configuration and
additional data (e.g. the Java application). A small PLD reads the
configuration data from the Flash and shifts it into the FPGA. This
method is used on the Cyclone and ACEX boards.

\subsection{Java Download}

When the FPGA is configured the Java application has to be
downloaded into the main memory. This download is performed in
microcode as part of the JVM startup sequence. The application is a
\code{.jop} file generated by \cmd{JOPizer}. At the moment there are
three options:

\begin{description}
    \item[Serial line] JOP listens to the serial line and the data
    is written into the main memory. A simple echo protocol performs
    the flow control. The baud rate is usually 115kBaud.
    \item[USB] Similar to the serial line version, JOP listens to the
    parallel interface of the FTDI FT2232 USB chip. The FT2232
    performs the flow control on the USB level and the echo
    protocol is omitted.
    \item[Flash] For stand alone applications the Java program is
    copied from the Flash (relative Flash address 0, mapped Flash
    address is 0x80000\footnote{All addresses in JOP are counted in
    32-bit quantities. However, the Flash is connected only to the
    lower 8 bits of the data bus. Therefore a store of one word in
    the main memory needs four loads from the Flash.}) to the main
    memory (usually a 32-bit SRAM).
\end{description}


To select on method for downloading a customized version of the JVM
is generated and the complete processor has to be built. The
generation is performed by the C preprocessor (\cmd{gcc}) on
\code{jvm.asm}. The serial version is generated by default, the USB
or Flash version are generated by defining the preprocessor
variables \code{USB} or \code{FLASH}. To speed up simulation in
ModelSim there is a forth method where the Java application is
loaded by the test bench instead of JOP. This version is generated
by defining \code{SIMULATION}. There are four small batch-files in
directory \dirent{asm} that perform the JVM generation:
\cmd{jopser}, \cmd{jopusb}, \cmd{jopflash}, and \cmd{jopsim}.

\subsection{Combinations}

Theoretically all ways to configure the FPGA can be combined with
all ways to download the Java application. However, only two
combinations are useful:

\begin{enumerate}
    \item For VHDL or Java development configure the FPGA
    via the download cable and download the Java application
    via the serial line or USB.
    \item For a stand-alone application load the configuration and
    the Java program from the Flash.
\end{enumerate}



\section{The Design Flow}

This section describes the design flow to build JOP in greater
detail.

\subsection{Tools}

There are a few tools necessary to build and download JOP to the
FPGA boards. Most of them are written in Java. Only the tools that
access the serial line are written in C\footnote{The Java JDK still
comes without the javax.comm package and getting this optional
package correct installed is not that easy --- Blame Sun.}.

\subsubsection{Downloading}

These little programs are already compiled and the binaries are
checked in in the CVS. The sources can be found in directory
\dirent{c\_src}.

\begin{description}
    \item[\eei{down.exe}] The workhorse to download Java programs. The
    mandatory argument is the COM-port. Optional switch \code{-e}
    keeps the program running after the download and echoes the
    characters from the serial line (\code{System.out} in JOP) to
    stdout. Switch \code{-usb} disables the echo protocol to speed up the
    download over USB.
    \item[\eei{e.exe}] Echo the characters from the serial line to stdout.
    Parameter is the COM-port.
    \item[\eei{amd.exe}] An utility to send data over the serial line to program
    the on-board Flash. The complementary Java program
    \code{util.Amd} must be running on JOP.
    \item[\eei{USBRunner.exe}] Download the FPGA configuration via
    USB with the FTDI2232C chip (dpsio board).
\end{description}

\subsubsection{Generation of Files}

These tools are written in Java and are delivered in source form.
The source can be found under \dirent{java/tools/src} and the class
files are in \code{jop-tools.jar} in directory
\dirent{java/tools/dist/lib}.

\begin{description}
    \item[\eei{Jopa}] The JOP assembler. Assembles the microcoded
    JVM and produces on-chip memory initialization files and VHDL
    files.
    \item[\eei{BlockGen}] converts Alter memory initialization files
    to VHDL files for a Xilinx FPGA.
    \item[\eei{JOPizer}] links a Java application and converts the
    class information to the format that JOP expects (a \code{.jop} file).
    JOPizer uses the bytecode engineering library (BCEL).

\end{description}

\subsubsection{Simulation}


\begin{description}
    \item[\eei{JopSim}] reads a \code{.jop} file and executes it in
    a debug JVM written in Java. Command line option
    \code{-Dlog="true"} prints a log entry for each executed JVM
    bytecode.
    \item[\eei{pcsim}] simulates the BaseIO expansion board for Java
    debugging on a PC (using the JVM on the PC).
\end{description}

\subsection{Targets}

JOP has been successfully ported to several different FPGAs and
boards. The main distribution contains the ports for the FPGAs:

\begin{itemize}
    \item Altera Acex 1K30 or 1K50
    \item Altera Cyclone EP1C6 or EP1C12
    \item Xilinx Spartan-3
\end{itemize}

Besides the ports to different FPGAs there are ports to different
boards.

\subsubsection{ACEX EP1K50C144 Jopcore}

This board was one of the first targets (besides the KFL project)
for JOP and the design files of the board are now available as
open-source from
\url{http://www.opencores.org/projects.cgi/web/acxbrd/overview}. As
the FPGA is a little bit dated the latest features of JOP (e.g. the
enhancements in the method cache) are not available in the Acex
port. Use \code{jop\_20040913\_v37\_web.zip} from the archive
section. Two Quartus projects for this board are available:
\code{acxmin}, a minimum configuration containing only a serial
interface, and \code{acxtal}, a configuration for the \emph{baseio}
extension board. The ACEX specific files are \code{jopacx.vhd} and
\code{mem.vhd}.

\subsubsection{Cyclone EP1C6/12}

This board is pin-compatible to the ACEX board and comes in two
versions: with an Cyclone EP1C6 or EP1C12. The board contains:

\begin{itemize}
    \item Altera Cyclone EP1C6Q240 or EP1C12Q240 FPGA (see data sheet)
    \item 1MB fast SRAM
    \item 512KB Flash (for FPGA configuration and program code)
    \item 32MB NAND Flash
    \item ByteBlasterMV port
    \item Watchdog with LED
    \item EPM7064 PLD to configure the FPGA from the Flash (on watchdog reset)
    \item Voltage regulator (1V5)
    \item Crystal clock (20 MHz) at the PLL input (up to 640 MHz internal)
    \item Serial interface (MAX3232)
    \item 56 general purpose IO pins
\end{itemize}

The Cyclone specific files are \code{jopcyc.vhd} or \code{jopcyc12}
and \code{mem32.vhd}. This FPGA board is designed as a module to be
integrated with an application specific IO-board. There exist
following IO-boards:
%
\begin{description}
    \item[simpexp] A simple bread board with a voltage regulator and
    a SUBD connector for the serial line
    \item[baseio] A board with Ethernet connection and EMC protected
    digital IO and analog input
    \item[bg263] Interface to a GPS receiver, a GPRS modem, keyboard
    and a display for a railway application
    \item[lego] Interface to the sensors and motors of the LEGO
    Mindstorms. This board is a substitute for the LEGO RCX.
    \item[dspio] Developed at the University of Technology Vienna, Austria for
    digital signal processing related work. All design files for this
    board are open-source.
\end{description}
%
Table~\ref{tab:cycio} lists the related VHDL files and Quartus
project directories for each IO board.

\begin{table}
    \centering

    \begin{tabular}{lll}
        \toprule
        IO board & Quartus & IO top level \\
        \midrule
        simpexp  & \dirent{cycmin} & \code{scio\_min.vhd} \\
        baseio  & \dirent{cycbaseio} & \code{scio\_baseio.vhd} \\
        bg263  & \dirent{cybg} & \code{scio\_bg.vhd} \\
        lego  & \dirent{cyclego} & \code{scio\_lego.vhd} \\
        dspio  & \dirent{dspio} & \code{scio\_dspio.vhd} \\
        \bottomrule

    \end{tabular}
    \caption{Quartus project directories and VHDL files for the different IO boards}
    \label{tab:cycio}

\end{table}


\subsubsection{Xilinx Spartan-3}

The Spartan-3 specific files are \code{jop\_xs3.vhd} and
\code{mem\_xs3.vhd} for the Xilinx Spartan-3 Starter Kit and
\code{jop\_trenz.vhd} and \code{mem\_trenz.vhd} for the Trenz
Retrocomputing board.


\section{Eclipse}

In folder \dirent{eclipse} there are several Eclipse projects that
you can import into your Eclipse workspace. All projects use the
Eclipse path variable\footnote{Eclipse (path) variables are
workspace specific.} \code{JOP\_HOME} that has to point to the root
directory of the JOP sources. Under \menuitem{Window --
Preferences...} select \menuitem{General -- Workspace -- Linked
Resources} and create the path variable \code{JOP\_HOME} with
\menuitem{New...}.

Import the projects with \menuitem{File -- Import..} and
\menuitem{Existing Projects into Workspace}. Select as root
directory \dirent{.../jop/eclipse}, select the projects you want to
import and press \menuitem{Finish}. Table~\ref{tab:eclipse} shows
all available projects.

\begin{table}
    \centering

    \begin{tabular}{ll}
        \toprule
        Project & Content \\
        \midrule
        \dirent{jop} & The target sources \\
        \dirent{joptools} & Tools such as \code{Jopa}, \code{JopSim}, and \code{JOPizer} \\
        \dirent{pc} & Some PC utilities (e.g.\ Flash programming via UDP/IP) \\
        \dirent{pcsim} & Simulation of the basio hardware on the PC \\
        \bottomrule

    \end{tabular}
    \caption{Eclipse projects}
    \label{tab:eclipse}

\end{table}

Add the libraries from \dirent{.../jop/java/lib} (as external
archives) to the build path of the project
\dirent{joptools}\footnote{Eclipse can't use path variables for
external .jar files -- annoying}. If you prefer your workspace to be
out of the JOP directory tree just copy the content of
\dirent{.../jop/eclipse} to your Eclipse workspace and import the
projects from there.

\section{Simulation}

This section contains the information you need to get a simulation
of JOP running. There are two ways to simulate JOP:
%
{\samepage
\begin{itemize}
    \item High-level JVM simulation with \cmd{JopSim}
    \item VHDL simulation (e.g. with ModelSim)
\end{itemize}
}
%
This section is about running a VHDL simulation with ModelSim. All
simulation files are vendor independent and should run on any
versions of ModelSim or a different VHDL simulator. You can simulate
JOP even with the free ModelSim XE II Starter Xilinx version.

\subsection{Background Information}

To simulate JOP, or any other processor design, in a vendor neutral
way models of the internal memories (block RAM) and the external
main memory are necessary. Beside this, only a simple clock driver
is necessary. To speed-up the simulation a little bit a simulation
of the uart output, which is used for \code{System.out.print()}, is
also part of the package.

Table~\ref{tab:simfiles} lists the simulation files for JOP and
which program generates the initialization data. The non-generated
VHDL files can be found in directory \dirent{vhdl/simulation}.
%
\begin{table}
    \centering

    \begin{tabular}{llll}
        \toprule
        VHDL file & Function & Initilization file & Generator \\
        \midrule
        \code{sim\_jop\_types\_100.vhd} & JOP constant definitions & - & - \\
        \code{sim\_rom.vhd} & JVM microcode ROM & \code{mem\_rom.dat} & \cmd{Jopa} \\
        \code{sim\_ram.vhd} & Stack RAM & \code{mem\_ram.dat} & \cmd{Jopa} \\
        \code{sim\_jbc.vhd} & Bytecode memory (cache) & - & - \\
        \code{sim\_memory.vhd} & Main memory (Java application) & \code{mem\_main.dat} & \cmd{jop2dat} \\
        \code{sim\_pll.vhd} & A dummy entity for the PLL & - & - \\
        \code{sim\_uart.vhd} & Print characters to stdio & - & - \\
        \bottomrule

    \end{tabular}
    \caption{Simulation specific VHDL files}
    \label{tab:simfiles}

\end{table}
%
The needed VHDL files and the compile order can be found in
\code{sim.bat} under \dirent{modelsim}.


The actual version of JOP contains all necessary files to run a
simulation with ModelSim (or is there another VHDL simulator
available?). In directory \dirent{vhdl/simulation} you will find:
%
\begin{itemize}
    \item A test bench: \code{tb\_jop.vhd} with a serial receiver to
    print out the messages from JOP during the simulation
    \item Simulation versions of all memory components (vendor neutral)
    \item Simulation of the main memory
\end{itemize}
%
\cmd{Jopa} generates various \code{mem\_xxx.dat} files that are read
by the simulation. The JVM that is generated with \code{jopsim.bat}
assumes the Java application preloaded in the main memory.
\cmd{jop2dat} generates a memory initialization file from the Java
application file (\code{package\_App.jop}) that is read by the
simulation of the main memory (\code{sim\_memory.vhd}).

In directory \dirent{modelsim} you will find a small batch file
(\cmd{sim.bat}) that compiles JOP and the test bench in the correct
order and starts ModelSim.

The whole simulation process (including generation of the correct
microcode) is started with:

\begin{verbatim}
    make sim
\end{verbatim}

After a few seconds you should see the startup message from JOP
printed in ModelSims command window.

\section{Files Types You Might Encounter}

As there are various tools involved in the complete build, you will
find files with various extensions. The following list explains the
file types you might encounter when changing and building JOP.

The list is far from being complete --- it's just a start. The
following files are the \emph{source} files:

\begin{description}

\item[\eei{.vhd}] VHDL files describe the hardware part and are
compiled with either Quartus or Xilinx ISE. Simulation in ModelSim
is also based on VHDL files.
\item[\eei{.v}] Verilog HDL. Another hardware description language.
Used more in the US.

\item[\eei{.java}] Java --- the language that runs native on JOP.

\item[\eei{.c}] There are still some tools written in C.

\item[\eei{.asm}] JOP microcode. The JVM is written in this stack
oriented assembler. Files are assembled with \cmd{Jopa}. The result
are VHDL files, .mif files, and .dat files for ModelSim.

\item[\eei{.bat}] Usage of these DOS batch files still prohibit
running the JOP build under Unix. However, these files get less used
as the \code{Makefile} progresses.

\item[\eei{.xml}] Project files for Ant. Not used anymore --- they
are out dated.


\end{description}


Quartus II and Xilinx ISE need configuration files that describe
your project. All files are usually ASCII text files.

\begin{description}

\item[\eei{.qpf}] Quartus II Project File. Contains almost no
information.
\item[\eei{.qsf}] Quartus II Settings File defines the project. VHDL
files that make up your project are listed. Constraints such as pin
assignments and timing constraints set here.
\item[\eei{.cdf}] Chain Description File. This file
stores device name, device order, and programming file name
information for the programmer.
\item[\eei{.tcl}] Tool Command Language. Can be used in Quartus to
automate parts of the design flow (e.g. pin assignment).

\item[\eei{.npl}] Xilinx ISE project. VHDL
files that make up your project are listed. The actual version of
Xilinx ISE converts this project file to a new format that is not in
ASCII anymore. Very annoying.
\item[\eei{.ucf}] Xilinx Foundation User Constraint File. Constraints
such as pin assignments and timing constraints set here.

\end{description}

\cmd{javac} and \cmd{jar} produce following file types from the Java
sources:

\begin{description}

\item[\eei{.class}] A class file contains the bytecodes, a symbol table and other
ancillary information and is executed by the JVM.

\item[\eei{.jar}] The Java Archive file format enables you to bundle multiple files
into a single archive file. Typically a \code{.jar} file will
contain the class files and auxiliary resources. A \code{.jar} file
is essentially a zip file that contains an optional
\dirent{META-INF} directory.

\end{description}

The following files are generated by the various tools from the
source files:

\begin{description}

\item[\eei{.jop}] The file makes up the linked Java application that
runns on JOP. It is generated by \cmd{JOPizer} and can be either
downloaded (serial line or USB) or stored in the Flash (or used by
the simulation with \cmd{JopSim} or ModelSim)

\item[\eei{.mif}] Memory Initialization File. Describe the initial
content of on-chip block memories for Altera devices.

\item[\eei{.dat}] memory initialization files for the simulation
with ModelSim.

\item[\eei{.sof}] SRAM Output File. Configuration for Altera
devices. Used by the Quartus programmer or by \cmd{quartus\_pgm}.
Can be converted to various (or too many) different format. Some are
listed below.

\item[\eei{.pof}] Programmer Object File. Configuration for Altera
devices. Used for the Flash loader PLDs.

\item[\eei{.jbc}] JamTM STAPL Byte Code 2.0. Configuration for Altera
devices. Input file for \cmd{jbi32}.

\item[\eei{.ttf}] Tabular Text File. Configuration for Altera
devices. Used by flash programming utilities (\cmd{amd} and
\cmd{udp.Flash} to store the FPGA configuration in the boards Flash.

\item[\eei{.rbf}] Raw Binary File. Configuration for Altera
devices. Used by the USB download utility (\cmd{USBRunner}) to
configure the dspio board via the USB connection.

\item[\eei{.bit}] Bitstream File. Configuration for Xilinx
devices.

\end{description}

\section{Porting JOP}

Add a description about the files (memory).

\subsection{Test Utilities}

To verify that the port of JOP is successful there are some small
test programs in \dirent{asm/src}. To run the JVM on JOP the
microcode \code{jvm.asm} is assembled and will be stored in an
on-chip ROM. The Java application will than be loaded by the first
microcode instructions in \code{jvm.asm} into an external memory.
However, to verify that JOP and the serial line are working correct
it is possible to run small test programs directly in microcode.

One test program (\code{blink.asm}) does not need the main memory
and is a first test step before testing the possible changed memory
interface. \code{testmon.asm} can be used to debug the main memory
interface. Both test programs can be built with the \cmd{make}
targets \cmd{jop\_blink\-test} and \cmd{jop\_testmon}.

\subsubsection{Blinking LED and UART output}

In directory \dirent{asm} the blink test program is built with:
%
\begin{verbatim}
    build blink
\end{verbatim}
%
Compile and download the FPGA configuration as described in
Section~\ref{subsubsec:quartus}. After download the watchdog LED
should blink and the FPGA will print out 0 and 1 on the serial line.
Use a terminal program or the utility \cmd{e.exe} to check the
output from the serial line.

\subsubsection{Test Monitor}

In directory \dirent{asm} the test monitor is built with:
%
\begin{verbatim}
    build testmon
\end{verbatim}
%
Start a terminal program (e.g. HyperTerm) to communicate with the
monitor program. Compile and download the FPGA configuration as
described in Section~\ref{subsubsec:quartus}.

After download the program prints the content of the memory at
address 0. The program understands following \emph{commands}:

\begin{itemize}
    \item A single CR reads the memory at the current addres
    and prints out the address and memory content
    \item \code{addr=val;} writes $val$ into the memory location at
    address $addr$
\end{itemize}

One tip: Take care that your terminal program does not send a LF
after the CR.


\section{Extending JOP}

JOP is a soft-core processor and customizing it for an application
is an interesting opportunity.

\subsection{Native Methods}

The \emph{native} language of JOP is microcode. A native method is
implemented in JOP microcode. The interface to this native method is
through a \emph{special} bytecode. The mapping between native
methods and the special bytecode is performed by \code{JOPizer}.

When adding a new (\emph{special}) bytecode to JOP following files
have to be changed:
\begin{enumerate}
    \item \code{jvm.asm} implementation
    \item \code{Native.java} method signature
    \item \code{JopInstr.java} mapping of the signature to the name
    \item \code{JopSim.java} simulation of the bytecode
    \item \code{JVM.java} (just rename the method name)
    \item \code{Startup.java} only when needed in a class initializer
    \item \code{WCETInstruction.java} timing information
\end{enumerate}

First implement the native code in \code{JopSim.java} for easy
debugging. The \emph{real} microcode is added in \code{jvm.asm} with
a label for the special byctecode. The naming convention is
\code{jopsys\_name}. In \code{Native.java} provide a method
signature for the native method and enter the mapping between this
signature and the name in \code{jvm.asm} in \code{JopInstr.java}.
Provide the execution time in \code{WCETInstruction.java} for the
WCET analysis.

The native method is accessed by the method provided in
\code{Native.java}. There is no calling overhead involved in the
mechanism.

\subsection{A new Peripheral Device}


Creation of a new peripheral devices involves some VHDL coding.
However, there are several examples in \dirent{jop/vhdl/scio}
available.


All peripheral components in JOP are connected with the SimpCon
\cite{simpcon} interface. For a device that implements the Wishbone
\cite{soc:wishbone} bus, a SimpCon-Wishbone bridge
(\code{sc2wb.vhd}) is available (e.g.\ it is used to connect the
AC97 interface in the dspio project).

A simple start is using an example and change it to your needs. Take
a look into \code{sc\_test\_slave.vhd}. All peripheral components
(SimpCon slaves) are connected in one module usually named
\code{scio\_xxx.vhd}. Browse the examples and copy one that best
fits your needs. In this module the address of your peripheral
device is defined (e.g. 0x10 for the primary UART). This IO address
is mapped to a negative memory address for JOP. That means
0xffffff80 is added to the address.

By convention this address mapping is defined in
\code{com.jopdesign.sys.Const}. Here is the UART example:

\begin{verbatim}
    // use negative base address for fast constant load
    // with bipush
    public static final int IO_BASE = 0xffffff80;
    ...
    public static final int IO_STATUS = IO_BASE+0x10;
    public static final int IO_UART = IO_BASE+0x10+1;
\end{verbatim}

The IO devices are accessed from Java by
\emph{native}\footnote{These are not real functions and are
substituted by special bytecodes on application building with
JOPizer.} functions: \code{com.jopdesign.sys.Native.rd()} and
\code{wr()}. Again an example with the UART:

\begin{verbatim}
    // busy wait on free tx buffer
    // no wait on an open serial line, just wait
    // on the baud rate
    while ((Native.rd(Const.IO_STATUS)&1)==0) {
        ;
    }
    Native.wr(c, Const.IO_UART);
\end{verbatim}

Best practise is to create a new IO configuration
\code{scio\_xxx.vhdl} and a new Quartus project for this
configuration. So the changes are not mixed up with a new version of
JOP. For the new Quartus project only the three files jop.cdf,
jop.qpf, and jop.qsf have to be copied in a new directory under
\dirent{quartus}. This new directory is the project name that has to
be set in the Makefile:

\begin{verbatim}
    QPROJ=yourproject
\end{verbatim}

The new VHDL module and the \code{scio\_xxx.vhdl} are added in
jop.qsf. This file is a plain ASCII file and can be edit with a
standard editor or within Quartus.



\section{Acknowledgments}

Ed Anuff wrote \code{testmon.asm} to perform a memory interface test
and \code{BlockGen.java} to convert Altera .mif files to Xilinx
memory blocks. \code{BlockGen.java} was the key tool to port JOP to
Xilinx FPGAs in general and the Spartan-3 more specific. Flavius
Gruian wrote JOPizer to generate the .jop file from the application
classes. JOPizer is based on the open source BCEL and a substitute
to Sun's proprietary \code{JavaCodeCompact}.

\section{Notes}

TODO:

\begin{itemize}
    \item pcsim
    \item JopSim
\end{itemize}
%Formating ideas -- see Latax intro p 27
%\url{http://www.ctan.org/tex-archive/info/lshort/english/lshort.pdf}\\


\subsection{A Customized Instruction}

Add a description how to implement a customized \emph{microcode}
instruction with hardware support on JOP.
