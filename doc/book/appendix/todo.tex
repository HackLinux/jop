\section{Working Hours}

Start 130h counting down from 14.9.2008

Restrict to 40h from 11.8.09 (other count was 75h)

Count: 65h 18.8. first edition submitted!!!


New count: 30h


\section{TODO Handbook}

\begin{itemize}
    \item Cover: Blue boxes, better test, better formatting,
        smaller author on binding
    \item Cover and back cover including text on Martin
        Schoeberl\\
        MS is the main devel. of JOP, SCJ, Position (Ass. Prof
    TU/DTU)
\end{itemize}





\chapter{DONE}

\begin{itemize}
  \item Green book corrections of
  \begin{itemize}
    \item Chapter~1
    \item Chapter~2
    \item Chapter~3
    \item Chapter~4 RT Java (removed)
    \item Chapter~5 (HW Arch)
    \item Chapter~6
    \item Chapter~7
    \item Chapter~8
    \item Chapter~9
    \item Chapter~10
    \item Chapter~11
    \item Chapter~12
    \begin{itemize}
      \item Some text from Chapter~RT Java into the
          embedded section of Chapter~\ref{chap:java},
          Intro from it as part of the history section
    \end{itemize}
  \end{itemize}
    \item Trevor's changes in the Wiki design flow page
    \item Redraw block diagram
    \begin{itemize}
      \item I/O after memory controller, memory interface
      \item jopcpu
      \item B goes to memory controller
      \item Method instead of Bytecode cache
    \end{itemize}
    \item Low-level I/O access
    \item HWO copy or low level access w Native.rdMem()
    \item IH text from ISORC paper (and Journal) to the
        interrupt section in scheduler -- write about the JOP
        scheduler
    \item First JHAL text + removed text
    \item At the end of I/O the legacy Native stuff
    \item redraw Figure 4.3
    \item remove single thread stuff from GC section
    \item Update performance section with actual numbers
    \begin{itemize}
        \item picoJava, jamuth
    \end{itemize}
    \item IFAC paper into application section
    \item Acknowledgements
    \item Related Work
    \begin{itemize}
        \item Cut it down
        \item Add jHISC, jamuth
        \item Add derived work (again)
    \end{itemize}
  \item Chapter order? WCET before RTGC?
    \item Figure PDF check (Done in Runtime, RTGC, WCET,
        Low-level IO, SimpCon)
    \item check quality of pictures (base station) on printout
    \item Update Results section (and decide on a different name)
  \item \textbf{change all units in tables to (x)!}
    \item check line width for Visio graphics (HW pipeline
        stages, heap layout in GC section)
  \item change all IO to I/O
  \item check java* fonts -- has been 9pt for the green book
  \item check figure sizes for the orange book size
  \item check figure placements
  \item use same listing environment (Java)
  \item unbounded vs. unbound in the whole text
  \item remove 'proposed'
  \item the WCET analysis -- WCET analysis (for all analysis)
    \item Foreword
\end{itemize}

\chapter{TODO for the 2nd Edition}

\begin{itemize}
  \item Show a pipeline example (is in brown book) in
      Chapter~\ref{chap:arch}
  \item Interrupt figure + how to connect I/O interrupts
  \item Interrupt plus scheduling from ISORC/TECS paper
  \item Memory figure + section on memory controller with
      reference to SimpCon
  \item The benchmark section is gone (commented out in
      java.tex). To include it again redo the measurements with
      specJVM and other more interesting apps (Kfl, Lift,...).
  \item The memory subsystem
  \begin{itemize}
    \item Mem. controller (some stuff from the JTRES paper)
    \item Mem. interface
  \end{itemize}
  \item The I/O Subsystem
  \begin{itemize}
    \item scio -- the I/O top-level
    \item sys io
    \item other examples
  \end{itemize}
  \item JOP Tools chapter
  \begin{itemize}
    \item JOPizer
    \item Joptimizer
    \item WCA
  \end{itemize}
  \item Better scheduler description
\begin{itemize}
  \item Explain implicit context store on the stack -- it was
      a major insigt
  \item redo Figure~\ref{fig_arch_rt_user_interaction}
  \item show code from RtThread
  \item scheduler code, and also CMP scheduling
\end{itemize}
  \item Add GC object copy unit description to the GC section
  \item Use new JTRES/TECS results in the GC evaluation section
  \item References are missing in further reading section on RTGC
  \item Merge RT profile and SCJ sections and update the profile
      in JOP
  \item Show RT scheduler code -- also CMP scheduling
  \item Rewrite user scheduler to a scheduler description
    \item instruction frequency of Kfl, Lift and UDP/IP into Java
        benchmark section
    \item GC chapter: mention how GC default and concurrent
        version works
    \item Low-level IO additions:
    \begin{itemize}
      \item A section on I/O organization (scio moduels) and
          I/O top level into JOP Hardware Architecture.
      \item I/O examples
      \item recheck JHAL removed text
    \end{itemize}
    \item A section on the Memory subsystem (around, before
        method cache)
  \item list Makefile targets and variables
  \item JOP configurations should be in its own section instead
      of the design flow.
    \item Add interrupt controller description
    \item Explain performance measurements?
    \begin{itemize}
        \item use text from oohw paper for the explanation of
            the benchmark
        \item own jbe section?
        \item micro benchmarks? Example in jhal paper.
        \item comparison with picoJava from EURASIP paper?
    \end{itemize}
  \item WCET timings for get/putfield/ref/long
    \item Derived work: Flavius BlueJEP, Christof, Dresden! -- in
        related work or usage of JOP in academia in Evaluation
    \item Figure from Slides on configuration (FPGA + serial/USB
        channel)
    \item A list of extended bytecodes around
        Section~\label{subsec:flex:bc} -- see also nice listing
        in Beneidkt's thesis (Listing 3.2).
    \item scheduler code into the Runtime section
    \item A description (with code example) in the GC section how
        to start the parallel GC.
    \item Printout some chapters and read them
    \item Index
\end{itemize}

\subsection{Minor Change Ideas}

\begin{itemize}
  \item Add references to the work that is only described in the
      thesis.
\end{itemize}

\subsection{Editorial Stuff}

\begin{itemize}
  \item missing commas
  \item adjective 'ly'
  \item articles (the, a)
  \item compatible \emph{with} and compare \emph{with}
\end{itemize}


\section{Additional Sections}

\begin{itemize}
    \item VHDL Hello world with cycore
    \item RTS Introduction (based on thesis intro)
    \item SCJ definition (+ implementation)
    \item (Mission modes + implementation)
    \item Teaching (aids)
    \item Examples section
    \begin{itemize}
        \item RtThread examples
        \item ejip examples
        \item low-level access
    \end{itemize}
    \item About boards (Baseio, dspio,...)
    \item OOHW copy
    \item restructure with background info and related work in
        each chapter
    \item Document design alternatives you didn't take and why
        you didn't take them
    \item Javadoc of some classes
    \item Embedded Java section
    \begin{itemize}
        \item Add Sun/IBM solution to the class
            initialization issue in Chapter~\ref{chap:rtjava}
            and a detailed description how JOPizer handles
            it. Give a simple example of a cyclic dependency.
        \item Check if the Micro Edition layers are still
            used by Sun
        \item check the Sun statement on profiles and provide
            a reference
        \item add PhoneME
        \item Still only one profile (MIDP) defined for the
            CLDC?
        \item Check if CDC definition is still true
        \item RTSJ implementations: add Angelo and York to
            jRate and additional RTSJ implementations (aicas,
            IBM, appogee, Sun)
        \item add SCJ in Subsets of the RTSJ
        \item check status of JSR-50
    \end{itemize}
    \item JOP Runtime System
    \begin{itemize}
        \item SCJ to the real-time profile (plus
            implementation) -- the JTRES text (Jan!) is
            commented out in the handbook
    \end{itemize}
    \item Describe ejip (it's now commented out in handbook in
        intro)
    \item SC Java
\end{itemize}

\chapter{Other Stuff}

\section{Publishing the Book}

Target is real book with ISBN (= publication).

\begin{itemize}
    \item \url{http://www.createspace.com/} no setup, no
        copyright transfer
    \item \url{http://www.virtualbookworm.com/} \$ 360 -- \$ 440, incl. Amazon only, 4 computer
    books, form mail sent, e-mail sent
    \item \url{http://www.authorhouse.com/} \$ 700.- incl. Amazon,
    no control on layout, form mail sent
    \item \url{http://www.wingspanpress.com/} \$ 500 incl. Amazon,
    no computer books, e-mail possible
    \item \url{http://www.pagefreepublishing.com/} form mail sent,
    e-mail sent
    \item \url{http://www.aventinepress.com/} e-mail sent
    \item \url{http://www.iuniverse.com/} \$ 399,- (\$ 599,- incl.
    Amazon), Author discount, not many computer related books (old
    topics) -- e-mail only possible with a Friends name and address!
    \item \url{http://trafford.com/}, EUR 750,-(\$ 1600,- incl. Amazon), EUR 11,- per
    book, some computer related books
    \item \url{http://www.booksurgepages.com/amzn/ondemand/}
    \item \url{http://www.booksurge.com/} e-mail sent \$ 300.-
\end{itemize}

Wishlist:
\begin{itemize}
    \item ISBN
    \item A publisher who has serious technical books - no pornography!
    \item Cheap upfront
    \item Order possible from Amazon
    \item Cheap order for myself
    \item Right to sell it myself
\end{itemize}

Send an e-mail and ask following questions:
\begin{itemize}
    \item Can I provide my own layout (submitting a PDF)?
    \item Can I still provide a free PDF version of my book on my
    web site?
    \item Can I update the content of the book (new edition)?
    \item Can I sell the book myself?
    \item Add a link to the manuscript.
\end{itemize}

No go's:
\begin{itemize}
    \item \url{http://www.lulu.com/} cheap \$ 99, but not clearely
    listed, freedom (e.g.\ GNU licence), no Amazon
    \item \url{http://www2.xlibris.com/} fixed layout
    \item http://superiorbooks.com/
    \item http://www.ubooks.de/
\end{itemize}



\section{A Possible Structure of the Book}

\begin{itemize}
    \item Introduction
    \item A Quick Tour of JOP

    \begin{itemize}
        \item Intro what JOP is
        \item A Quick Start (Hello World) or Getting Started
        \item A Short History
        \item About this handbook (based on, organization)
        \item VHDL Hello World
    \end{itemize}
    \item Architecture (HW)
    \item Architecture (SW)
    \item Build.pdf
    \item Source organization
    \item Background information
    \begin{itemize}
        \item JVM
        \item GC
        \item Related Work
        \item Real-time systems
    \end{itemize}
    \item WCET
    \item Real-time threads
    \item Tools
    \begin{itemize}
        \item JOPizer
        \item WCET Analyzer
    \end{itemize}
    \item JDK + support
    \item Library: util, ejip, YAFFS,...
    \item Board descriptions
    \item Appendix
    \begin{itemize}
        \item Microcode
        \item Instruction timing
        \item Javadoc for classes
    \end{itemize}
    \item Further reading (related work) at the end of each chapter
    \item Browse other books for structural ideas
\end{itemize}

\section{JOPizer Description}
\index{JOPizer}

JOPizer is the tool to link all Java classes and translate the
application to a executable image for JOP. The output is a .jop file
(in ASCII with comments to debug the class structure). The executable
image is either downloaded via serial or USB interface, or programmed
into a Flash. The image is located at address 0.

JOPizer performs following steps:

\begin{enumerate}
    \item Find the transitive hull of all needed classes starting
    from the class that contains the main method.
    \item ...
    \item Find a static class initializer order
    (\code{ClinitOrder}). Throws an error on a cyclic dependency.
    \item Dump the application to the image file
\end{enumerate}

\section{YAFFS Notes}

\begin{itemize}
    \item new in yaffs2.port.yaffs\_guts\_C.yaffs\_GetObjectName(...)
\end{itemize}

\section{Debugger Description}

\begin{verbatim}

------------------------------------------------------------
Attachments: - Modified_jop_files.zip
  Some modified files needed to run the debugger

  Changed files:
  Makefile
  java/tools/source/com/jopdesign/build/MethodInfo.java
  java/tools/source/com/jopdesign/build/JOPizer.java
  java/tools/source/com/jopdesign/build/ClassInfo.java
  (there's no need to change RtThreadImpl anymore)

  Added (to the development tree):
  java/target/source/test/JopDebugKernel/
  java/tools/dist/lib/JopDebugger.jar   (after "make tools" run)

- JopDebugger.zip
  Latest development version of the debugger module.

- JopDebugKernel.zip
  This will run inside JOP.


  The changes in the makefile are to allow it to debug JOPizer and JopSim
  through the network and to create the symbol file.
  The last change modify one step in the build process,
  to call JOPizer from another class which will create the symbol file
  right after JOPizer runs.

------------------------------------------------------------



Instructions to run the example:

- Open two shells (one for Jop, other for the JopDebugger) - Shell
1: Download a clean development tree from Jop CVS.

- Copy modified Jop classes (from Modified_jop_files.zip)
  in the correct folders.
  Those small changes I did in the tool set were just to allow
  serialization and access to some fields.
  Copy also the "Makefile" included over your Makefile.

- Build the modified tools: make tools

- Shell 2:
  Decompress the attached JopDebugger.zip in a sibling folder.

- Build the project lib: cd JopDebugger; ant lib

- Copy the file JopDebugger.jar into java\tools\dist\lib
  Remeber that if you run "make tools" it will erase the
  lib folder, which will remove JopDebugger.jar.

- Unpack "JopDebugKernel.zip" and move it into
  jop\java\target\src\test\JopDebugKernel
  (but don't put it into JopDebugKernel\JopDebugKernel,
  this is a common mistake).

- Run the line below.
  This test will run a stand-alone test under JopSim. If it runs and print some
  simple messages, then it's possible to test the methods which access JOP's
  internal structures.

  make jsim -e P1=test/JopDebugKernel/source P2=debug P3=TestJopDebugKernel

This will build all the system as usual: compile all, run JOPizer,
bild the symbols file and launch JopSim.

However, there are a few differences in this makefile. 1) It allows
anyone to debug JopSim remotely (manually changing the makefile) 2)
It allows anyone to debug JOPizer remotely (manually changing the
makefile) 3) The modified makefile also change the build process to
store a symbol table by calling JOPizer from another tool (symbol
storage and loading is working) 4) It can connect JopSim to the
network by calling JopSim from another tool (a JopServer)


This test will run for a few seconds
 while and then will stop to wait for input, showing the text below.

 ----------------------------------------
 test4_read_write()
 Debug server. Current stack depth: 2
 ----------------------------------------

When this happens, force it to stop (ctrl-C). The remaining of the
test needs commands from a server (through the network), explained
next.


To connect JopSim to the network using JopServer:
--------------------------------------------------

- Open the .jop file created in the previous run.

- Locate two method pointers:
  search for: "TestJopDebugKernel.printValue(I)V" and also
  for "TestJopDebugKernel.printLine()V".

- Update the file com.jopdesign.debug.jdwp.test.TestJopServer
  (lines 152 and 162) with those addresses.

- Run the line below:

  make jsim_server -e P1=test/JopDebugKernel/source P2=debug P3=TestJopDebugKernel

- Wait until the server is running.
  It will print a message: "Server launched at port: 8004"

- After the Jop server start running, go back to the other shell and
  launch the ant target "TestJopServer". It should recompile the modified
  java sources and launch the test.

  ant TestJopServer

If the methods addresses are not corrected, there is a big chance
that the test will break with an error message like this: "Calling
method now: make: *** [jsim] Error 25584194"

Everytime JOPizer run, new method addresses are calculated when
classes change. This error is just due to a couple wrong method
pointers. So, if you want to change anything in the code which will
run inside Jop, it's necessary to fix this test (or comment the
method invocation calls). Otherwise it will break.

To fix method pointers: - Open the .jop file - Search for the two
method signatures printed by the TestJopServer class - Extract
manually the two method addresses (hey, this is not finished yet;)
it's just a test, you know). - Open class
com.jopdesign.debug.jdwp.test.TestJopServer. - Change the code by
pasting the method addresses

Launch again the JopServer using make Run again TestJopServer.

You may see that it's working by choosing other static void methods
to call. This works for static methods with one or zero parameters
only. Support for more is very easy to implement but was not needed
now, so I didn't (yet). Returned values are currently ignored.

--------------------------------------------------------------------------------
Well, that's it. There are a lot of details and other small tools,
but I hope now you have a much better idea about how my development
effort is going until here.

Those changes were tested on JOP from Sep. 11, so unless there was
some big change in the last days, they should work ok on the latest
version.

The first test I mentioned shows you evidence of methods that access
JOP structures.

The test with JopServer shows that it's possible to connect JopSim
to the network and send/receive commands which access and change JOP
internal structures. Since this was accomplished without changing
JopSim, it should be possible (at least in thery) to switch it with
an FPGA running JOP and get the same results.

This shows it's possible to control JOP from the network and perform
some tasks related to debugging (such as get/set values).

It's possible also to run another test which launch a real debugger
which connects to a server, handshake and receive JDWP commands. But
this test still does not do much, so I don't know if you want to see
it.

There's also a test to check the symbol file content.

About the JOP side, it's still necessary to write code to set/reset
breakpoints and handle interruptions properly.

It's also necessary to glue together the symbol file, the Jop debug
server and the JDWP module to control debugging. In short, it's a
lot of work.

If you need help to run or test it or if you have questions, please
let me know.

Regards, Paulo


\end{verbatim}


\section{TODO JOP}

\begin{itemize}
    \item Javadoc for the handbook
\end{itemize}
