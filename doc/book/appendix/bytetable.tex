%
% a changed copy from the thesis

%\documentclass[a4paper,10pt,twocolumn]{article}
% use option draft to check for typesetting problems.
% similar to original word document

% don't use textheight... with koma-script
\documentclass[%draft,
%    a4paper,12pt, % 'Standard'
    11pt, % use explicit paper format for Java book format
%    b5paper,10pt, % B5 printout
    % use BCOR to compensate for the two-side margin!
%    a4paper,12pt,BCOR37pt,DIV15,
%    a4paper,12pt,BCOR40pt,DIV14,
%    b5paper,10pt,DIV12,
    headinclude, footexclude,
    twoside, % this produces strange margins!
    openright, % for new chapters
    cleardoubleempty,
% normalheadings - smaller chapter or smallheadings - too small
    % abstracton, % the author of koma-script argues against the title
    headsepline,
%    5headlines, % standard is 1.25 -- wirkt net!
    pointlessnumbers,
    ]{scrreprt}

%\usepackage{html}

% headings
\usepackage{scrpage2} % for headers
 \setkomafont{pagehead}{\scshape\small}
 \setkomafont{pagenumber}{\scshape\small}
 \automark[section]{chapter}
 \ohead[]{\pagemark}
 \chead[]{}
 \ihead[]{\headmark}
 \ofoot[]{} \cfoot[]{} \ifoot[]{}

%\tolerance=500 % to avoid lines sticking out into the margin
               % needed for 'high-performance' in Intro - contributions
\emergencystretch=2em
% or tol. to 500 and emerg. to 1em?
% pagebreak was ok with 500 and 1em
\interfootnotelinepenalty=10000


% use BCOR = (paperwidth-textwidth)/4
% A4: 210mm x 297mm
% B5: 176mm x 250mm
% Java book: 185mm x 232mm
% Engblom: 120x188 (without head)
% Java: 127x187 (without head)
% 1pt = 1/72.27 in = 0.351 mm


% for book
% 'Java-format' 526pt x 660pt (Ghostscript)
\setlength{\paperwidth}{185mm} \setlength{\paperheight}{232mm}
% use that BCOR setting with twoside to compensate the margin
\areaset[13.75mm]{130mm}{200mm} % Java book format

% for book to A4 conversion:
% Set the following sizes and export with
% 'Variable Page Size' in gs.
% Print with 'fit to paper' in Acrobat - results in 110%
% and an effective text area of 142x215
%\setlength{\paperwidth}{180mm} \setlength{\paperheight}{232mm}
%\areaset[12.5mm]{130mm}{200mm} % Java book format

% for two page printout in .pdf
%\setlength{\paperwidth}{134mm} \setlength{\paperheight}{206mm}
%\areaset[1mm]{130mm}{200mm} % Java book format


% use 10pt for code instead of 11pt - but I still would prefer Lucida Typewriter
%\newfont{\myttfont}{cmss10 scaled 1000}
%\newfont{\myttbfont}{cmssdc10 scaled 1000}
%
% This IS Lucida Typewriter
%\newfont{\myttfont}{plsr8r scaled 950}
%\newfont{\myttbfont}{plsb8r scaled 950}
%\newfont{\myttifont}{plsro8r scaled 950}
%%\newfont{\mytttextfont}{plsr8r}

% Lucida is perhaps available in the new Tex installation!!!!
% does not really work!!!
\newfont{\myttfont}{hlsrt8r scaled 950}
\newfont{\myttbfont}{hlsbt8r scaled 950}
\newfont{\myttifont}{hlsrot8r scaled 950}

% I used these .ttf for the official Thesis
%..\ttf2pt1 -e -b LucidaTypewriterRegular.ttf plsr8a
%..\ttf2pt1 -e -b LucidaTypewriterBold.ttf plsb8a
%..\ttf2pt1 -e -b LucidaTypewriterOblique.ttf plsro8a
%..\ttf2pt1 -e -b LucidaTypewriterBoldOblique.ttf plsbo8a

%\newcommand{\javatt}{\myttfont}
%\newcommand{\javattb}{\myttbfont}
%\newcommand{\javatti}{\myttifont}
%\newcommand{\javatext}{\myttfont}
%
%\newcommand{\picscale}{0.909}
%\newcommand{\excelwidth}{11cm}

% end book

% for B5
%\newfont{\javatt}{cmss10}
%\newfont{\javattb}{cmssdc10}
%\newcommand{\picscale}{0.833}
%\newcommand{\excelwidth}{10cm}



% for A4
% 12pt A4 scaled from book
%\areaset[17.05mm]{142mm}{219mm}
\newfont{\javatt}{cmss12}
\newfont{\javattb}{cmssdc10 scaled 1200}
% TODO find an italic
\newfont{\javatti}{cmss12}
\newcommand{\javatext}{\javatt}

\newcommand{\picscale}{1}
\newcommand{\excelwidth}{12cm}


% for chapter head without a number
% \renewcommand{\chaptermark}[1]{\def\myleftmark{#1}}
% \ihead{\myleftmark} \chead{} \ohead{{\rightmark}}

\setkomafont{captionlabel}{\sffamily\bfseries}



% Do I need this package?
\usepackage{float}

% is this a correction for the <> problem?
% \usepackage[T1]{fontenc}

\usepackage{pslatex} % -- times instead of computer modern
% pslatex should be replaced by this:
%\usepackage{mathptmx}
%\usepackage[scaled=.90]{helvet}
%\usepackage{courier}
% pslatex does not work with T1 encoding. <> Problem?


\usepackage{latexsym}
\usepackage{graphicx}
\usepackage{amsmath}
\usepackage{longtable}
\usepackage{booktabs}

% I would need Lucida Console!!!
%
%\newfont{\javatt}{pltt12} % lucida teletype, better than normal but with serifs
%\newfont{\javatt}{plss12} % lucida no serifes, but variable spacing
%\newfont{\javatt}{plss10 scaled 1200}
%\newfont{\javattb}{plssdc10 scaled 1200}
% cmss is NOT a tt font....


\usepackage{listings}
\lstset{language=Java,keywordstyle=,
basicstyle=\javatt,emphstyle=\javattb,commentstyle=\javatti,
showstringspaces=false,captionpos=b}

\usepackage{array}
\usepackage{dcolumn}
\newcommand{\cc}[1]{\multicolumn{1}{c}{#1}}
\newcolumntype{d}[1]{D{.}{.}{#1}}

% f�r die Umlaute in der Kurzfassung
% bekomme ich dadurch Probleme???
%\usepackage[ansinew]{inputenc}


\usepackage{capt-of}
\usepackage[colorlinks=true,linkcolor=black,citecolor=black]{hyperref}
%\usepackage{hyperref}

% ----------------------

%\usepackage{makeidx}
%\makeindex


\usepackage{import} % for subimport text and graphics from subdirectory
% does not work with latex2html!


\newcommand{\codefoot}{\textsf}
\newcommand{\code}[1]{{\javatext#1}} % for LaTeX
\newcommand{\cmd}[1]{{\texttt{#1}}}
\newcommand{\dirent}[1]{{\texttt{#1}}}
%\newcommand{\menuitem}[1]{\textsf{\textbf{#1}}}
\newcommand{\menuitem}[1]{\textsf{\textsl{#1}}}

% for flow.tex - part of index helper
\newcommand{\eei}[1]{%
\index{extension!\texttt{#1}}\texttt{#1}}

% JVs et al
%\newcommand{\ea}{et al.\xspace}
\newcommand{\ea}{et al.\ }

%\begin{htmlonly}
%\renewcommand{\code}[1]{{\texttt{#1}}} % for html2LaTeX
%\newcommand{\toprule}{\hline}
%\newcommand{\midrule}{\hline}
%\newcommand{\bottomrule}{\hline}
%\end{htmlonly}

% net wirklich notwendig -- h�ngt von code generierung ab
%\begin{htmlonly}
%\renewcommand{\javatt}{\texttt}
%\renewcommand{\javattb}{\texttt\bfseries}
%\end{htmlonly}

%\code{\hyphenchar\font=-1}

\newcommand{\mycomment}[1]{}

\newcommand{\instr}[6]{
    \begin{table}
        \begin{tabular}{ll}
            \emph{\large\textbf{#1}} & \\
            \\ \\
            \textbf{Operation} & #2 \\ \\
            \textbf{Opcode} & \texttt{#3} \\ \\
            \textbf{Dataflow} & \parbox[t]{9.5cm}{\(#4\)}\\ \\
            \textbf{JVM equivalent} & \parbox[t]{9.5cm}{\code{#5}} \\ \\
            \textbf{Description} & \parbox[t]{9.5cm}{#6}\\
        \end{tabular}
    \end{table}
}


\tablename~\ref{tab:appendix:bytecode} lists the bytecodes of the
JVM with their opcode, mnemonics, the implementation type and the
execution time on JOP. In the implementation column \emph{hw} means
that this bytecode has a microcode equivalent, \emph{mc} means that
a microcode sequence implements the bytecode, \emph{Java} means the
bytecode is implemented in Java, and a `-' indicates that this
bytecode is not yet implemented. For bytecodes with a variable
execution time the minimum and maximum values are given.

\begin{longtable}{rllr}
    \toprule
    Opcode & Instruction & Implementation & Cycles \\
    \midrule
    \endhead
    \bottomrule
    \caption{Implemented bytecodes and execution time in cycles}
    \label{tab:appendix:bytecode}
    \endfoot
%   18 & ldc & mc & 3+m \\
    0 & nop & hw & 1 \\
1 & aconst\_null & hw & 1 \\
2 & iconst\_m1 & hw & 1 \\
3 & iconst\_0 & hw & 1 \\
4 & iconst\_1 & hw & 1 \\
5 & iconst\_2 & hw & 1 \\
6 & iconst\_3 & hw & 1 \\
7 & iconst\_4 & hw & 1 \\
8 & iconst\_5 & hw & 1 \\
9 & lconst\_0 & mc & 2 \\
10 & lconst\_1 & mc & 2 \\
11 & fconst\_0 & Java &  \\
12 & fconst\_1 & Java &  \\
13 & fconst\_2 & Java &  \\
14 & dconst\_0 & - &  \\
15 & dconst\_1 & - &  \\
16 & bipush & mc & 2 \\
17 & sipush & mc & 3 \\
18 & ldc & mc & 7+r \\
19 & ldc\_w & mc & 8+r \\
20 & ldc2\_w\footnotemark[20] & mc & 17+2*r \\
21 & iload & mc & 2 \\
22 & lload & mc & 11 \\
23 & fload & mc & 2 \\
24 & dload & mc & 11 \\
25 & aload & mc & 2 \\
26 & iload\_0 & hw & 1 \\
27 & iload\_1 & hw & 1 \\
28 & iload\_2 & hw & 1 \\
29 & iload\_3 & hw & 1 \\
30 & lload\_0 & mc & 2 \\
31 & lload\_1 & mc & 2 \\
32 & lload\_2 & mc & 2 \\
33 & lload\_3 & mc & 11 \\
34 & fload\_0 & hw & 1 \\
35 & fload\_1 & hw & 1 \\
36 & fload\_2 & hw & 1 \\
37 & fload\_3 & hw & 1 \\
38 & dload\_0 & mc & 2 \\
39 & dload\_1 & mc & 2 \\
40 & dload\_2 & mc & 2 \\
41 & dload\_3 & mc & 11 \\
42 & aload\_0 & hw & 1 \\
43 & aload\_1 & hw & 1 \\
44 & aload\_2 & hw & 1 \\
45 & aload\_3 & hw & 1 \\
%46 & iaload\footnotemark[46] & mc & 32+3*r \\
46 & iaload\footnotemark[46] & mc & 7+3*r \\
47 & laload & mc & 43+4*r \\
48 & faload\footnotemark[46] & mc & 7+3*r \\
49 & daload & - &  \\
50 & aaload\footnotemark[46] & mc & 7+3*r \\
51 & baload\footnotemark[46] & mc & 7+3*r \\
52 & caload\footnotemark[46] & mc & 7+3*r \\
53 & saload\footnotemark[46] & mc & 7+3*r \\
54 & istore & mc & 2 \\
55 & lstore & mc & 11 \\
56 & fstore & mc & 2 \\
57 & dstore & mc & 11 \\
58 & astore & mc & 2 \\
59 & istore\_0 & hw & 1 \\
60 & istore\_1 & hw & 1 \\
61 & istore\_2 & hw & 1 \\
62 & istore\_3 & hw & 1 \\
63 & lstore\_0 & mc & 2 \\
64 & lstore\_1 & mc & 2 \\
65 & lstore\_2 & mc & 2 \\
66 & lstore\_3 & mc & 11 \\
67 & fstore\_0 & hw & 1 \\
68 & fstore\_1 & hw & 1 \\
69 & fstore\_2 & hw & 1 \\
70 & fstore\_3 & hw & 1 \\
71 & dstore\_0 & mc & 2 \\
72 & dstore\_1 & mc & 2 \\
73 & dstore\_2 & mc & 2 \\
74 & dstore\_3 & mc & 11 \\
75 & astore\_0 & hw & 1 \\
76 & astore\_1 & hw & 1 \\
77 & astore\_2 & hw & 1 \\
78 & astore\_3 & hw & 1 \\
%79 & iastore\footnotemark[79] & mc & 35+2*r+w \\
79 & iastore\footnotemark[79] & mc & 9+2*r+w \\
80 & lastore\footnotemark[1] & mc & 48+2*r+2*w \\
81 & fastore\footnotemark[79] & mc & 9+2*r+w \\
82 & dastore & - &  \\
83 & aastore & Java & \\
84 & bastore\footnotemark[79] & mc & 9+2*r+w \\
85 & castore\footnotemark[79] & mc & 9+2*r+w \\
86 & sastore\footnotemark[79] & mc & 9+2*r+w \\
87 & pop & hw & 1 \\
88 & pop2 & mc & 2 \\
89 & dup & hw & 1 \\
90 & dup\_x1 & mc & 5 \\
91 & dup\_x2 & mc & 7 \\
92 & dup2 & mc & 6 \\
93 & dup2\_x1 & mc & 8 \\
94 & dup2\_x2 & mc & 10 \\
95 & swap\footnotemark[2] & mc & 4 \\
96 & iadd & hw & 1 \\
97 & ladd & Java &  \\
98 & fadd & Java &  \\
99 & dadd & - &  \\
100 & isub & hw & 1 \\
101 & lsub & Java &  \\
102 & fsub & Java &  \\
103 & dsub & - &  \\
104 & imul & mc & 35 \\
105 & lmul & Java &  \\
106 & fmul & Java &  \\
107 & dmul & - &  \\
108 & idiv & Java &  \\
109 & ldiv & Java &  \\
110 & fdiv & Java &  \\
111 & ddiv & - &  \\
112 & irem & Java &  \\
113 & lrem & Java &  \\
114 & frem & Java &  \\
115 & drem & - &  \\
116 & ineg & mc & 4 \\
117 & lneg & Java &  \\
118 & fneg & Java &  \\
119 & dneg & - &  \\
120 & ishl & hw & 1 \\
121 & lshl & Java &  \\
122 & ishr & hw & 1 \\
123 & lshr & Java &  \\
124 & iushr & hw & 1 \\
125 & lushr & Java &  \\
126 & iand & hw & 1 \\
127 & land & Java &  \\
128 & ior & hw & 1 \\
129 & lor & Java &  \\
130 & ixor & hw & 1 \\
131 & lxor & Java &  \\
132 & iinc & mc & 8 \\
133 & i2l & Java &  \\
134 & i2f & Java &  \\
135 & i2d & - &  \\
136 & l2i & mc & 3 \\
137 & l2f & - &  \\
138 & l2d & - &  \\
139 & f2i & Java &  \\
140 & f2l & - &  \\
141 & f2d & - &  \\
142 & d2i & - &  \\
143 & d2l & - &  \\
144 & d2f & - &  \\
145 & i2b & Java &  \\
146 & i2c & mc & 2 \\
147 & i2s & Java &  \\
148 & lcmp & Java &  \\
149 & fcmpl & Java &  \\
150 & fcmpg & Java &  \\
151 & dcmpl & - &  \\
152 & dcmpg & - &  \\
153 & ifeq & mc & 4 \\
154 & ifne & mc & 4 \\
155 & iflt & mc & 4 \\
156 & ifge & mc & 4 \\
157 & ifgt & mc & 4 \\
158 & ifle & mc & 4 \\
159 & if\_icmpeq & mc & 4 \\
160 & if\_icmpne & mc & 4 \\
161 & if\_icmplt & mc & 4 \\
162 & if\_icmpge & mc & 4 \\
163 & if\_icmpgt & mc & 4 \\
164 & if\_icmple & mc & 4 \\
165 & if\_acmpeq & mc & 4 \\
166 & if\_acmpne & mc & 4 \\
167 & goto & mc & 4 \\
168 & jsr & \emph{not used} &  \\
169 & ret & \emph{not used} &  \\
170 & tableswitch\footnotemark[170] & Java & \\
171 & lookupswitch\footnotemark[171] & Java &  \\
172 & ireturn\footnotemark[172] & mc &  23+r+l \\
173 & lreturn\footnotemark[173] & mc &  25+r+l \\
174 & freturn\footnotemark[172] & mc &  23+r+l \\
175 & dreturn\footnotemark[173] & mc &  25+r+l \\
176 & areturn\footnotemark[172] & mc &  23+r+l \\
177 & return\footnotemark[177] & mc &  21+r+l \\
178 & getstatic & mc & 12+2*r \\
179 & putstatic & mc & 13+r+w \\
180 & getfield & mc & 17+2*r \\
181 & putfield & mc & 20+r+w \\
182 & invokevirtual\footnotemark[182] & mc & 100+4r+l \\
183 & invokespecial\footnotemark[183] & mc &  74+3*r+l \\
184 & invokestatic\footnotemark[183] & mc &  74+3*r+l \\
185 & invokeinterface\footnotemark[185] & mc &  114+6r+l \\
186 & unused\_ba & - &  \\
187 & new\footnotemark[187] & Java &   \\
188 & newarray\footnotemark[188] & Java &  \\
189 & anewarray & Java &  \\
190 & arraylength & mc & 6+r \\
191 & athrow\footnotemark[3] & Java &  \\
192 & checkcast & Java &  \\
193 & instanceof & Java &  \\
194 & monitorenter & mc & 11 \\
195 & monitorexit & mc & 10/14 \\
196 & wide & \emph{not used} &  \\
197 & multianewarray\footnotemark[4] & Java &  \\
198 & ifnull & mc & 4 \\
199 & ifnonnull & mc & 4 \\
200 & goto\_w & \emph{not used} &  \\
201 & jsr\_w & \emph{not used} &  \\
202 & breakpoint & - &  \\
203 & reserved & - &  \\
204 & reserved & - &  \\
205 & reserved & - &  \\
206 & reserved & - &  \\
207 & reserved & - &  \\
208 & reserved & - &  \\
209 & jopsys\_rd\footnotemark[209] & mc & 4+r \\
210 & jopsys\_wr & mc & 5+w \\
211 & jopsys\_rdmem & mc & 4+r \\
212 & jopsys\_wrmem & mc & 5+w \\
213 & jopsys\_rdint & mc & 3 \\
214 & jopsys\_wrint & mc & 3 \\
215 & jopsys\_getsp & mc & 3 \\
216 & jopsys\_setsp & mc & 4 \\
217 & jopsys\_getvp & hw & 1 \\
218 & jopsys\_setvp & mc & 2 \\
219 & jopsys\_int2ext\footnotemark[219] & mc & 14+r+n*(23+w) \\
220 & jopsys\_ext2int\footnotemark[220] & mc & 14+r+n*(23+r) \\
221 & jopsys\_nop & mc & 1 \\
222 & jopsys\_invoke & mc &  \\
223 & jopsys\_cond\_move & mc & 5 \\
224 & getstatic\_ref & mc & \\
225 & putstatic\_ref & mc & \\
226 & getfield\_ref & mc & \\
227 & putfield\_ref & mc & \\
228 & getstatic\_long & mc & \\
229 & putstatic\_long & mc & \\
230 & getfield\_long & mc & \\
231 & putfield\_long & mc & \\
232 & reserved & - \\
233 & reserved & - \\
234 & reserved & - \\
235 & reserved & - \\
236 & reserved & - \\
237 & reserved & - \\
238 & reserved & - \\
239 & reserved & - \\
240 & sys\_int\footnotemark[240] & Java \\
241 & sys\_exc\footnotemark[240] & Java \\
242 & reserved & - \\
243 & reserved & - \\
244 & reserved & - \\
245 & reserved & - \\
246 & reserved & - \\
247 & reserved & - \\
248 & reserved & - \\
249 & reserved & - \\
250 & reserved & - \\
251 & reserved & - \\
252 & reserved & - \\
253 & reserved & - \\
254 & sys\_noimp & Java \\
255 & sys\_init & \emph{not used} \\

\end{longtable}

\footnotetext[1]{The exact value is given below.}

\footnotetext[2]{Not tested as javac does not emit the \code{swap}
bytecode.}

\footnotetext[3]{A simple version that stops the JVM. No catch
support.}

\footnotetext[4]{Only dimension 2 supported.}

\footnotetext[20]{The exact value is
    $17+\left\{\begin{array}{r@{\quad:\quad}l}
    r-2 & r>2 \\
    0   & r\le2
    \end{array} \right.
    +
    \left\{\begin{array}{r@{\quad:\quad}l}
    r-1 & r>1 \\
    0   & r\le1
    \end{array} \right.
    $
}

\footnotetext[46]{The exact value is
%    $19+r+\left\{\begin{array}{r@{\quad:\quad}l}
%    r-2 & r\ge6 \\
%    4   & r<6
%    \end{array} \right. $
 \emph{no hidden wait states at the moment.}
}

\footnotetext[79]{The exact value is
%    $22+\left\{\begin{array}{r@{\quad:\quad}l}
%    r-2 & r\ge6 \\
%    4   & r<6
%    \end{array} \right.
%    +w
%    $
 \emph{no hidden wait states at the moment.}
}

\footnotetext[170]{\codefoot{tableswitch} execution time depends to
a great extent on the caching of the corresponding Java method or
the memory transfer time for the method.}

\footnotetext[171]{\codefoot{lookupswitch} execution time depends to
a great extent on the caching of the corresponding Java method or
the memory transfer time for the method. \codefoot{lookupswitch}
also depends on the argument as it performs a linear search in the
jump table.}

%172 & ireturn & mc &  23+r+b\footnotemark[172] \\
\footnotetext[172]{The exact value is:
    $
    23+\left\{\begin{array}{r@{\quad:\quad}l}
    r-3 & r>3 \\
    0   & r\le3
    \end{array} \right.
    +
% the saved cycles are counted from the instruction after stbcrd
% up to and including the last wait
    \left\{\begin{array}{r@{\quad:\quad}l}
    l-10 & l>10 \\
    0   & l\le10
    \end{array} \right.
    $
}

%173 & lreturn & mc &  25+r+b\footnotemark[173] \\
\footnotetext[173]{The exact value is:
    $
    25+\left\{\begin{array}{r@{\quad:\quad}l}
    r-3 & r>3 \\
    0   & r\le3
    \end{array} \right.
    +
    \left\{\begin{array}{r@{\quad:\quad}l}
    l-11 & l>11 \\
    0   & l\le11
    \end{array} \right.
    $
}



%177 & return & mc &  21+r+b\footnotemark[177] \\
\footnotetext[177]{ The exact value is:
    $
    21+\left\{\begin{array}{r@{\quad:\quad}l}
    r-3 & r>3 \\
    0   & r\le3
    \end{array} \right.
    +
    \left\{\begin{array}{r@{\quad:\quad}l}
    l-9 & l>9 \\
    0   & l\le9
    \end{array} \right.
    $
}

%182 & invokevirtual & mc & 82+4r+b\footnotemark[182] \\
\footnotetext[182]{The exact value is:
    $
    100+2r+
    \left\{\begin{array}{r@{\quad:\quad}l}
    r-3 & r>3 \\
    0   & r\le3
    \end{array} \right.
    +
    \left\{\begin{array}{r@{\quad:\quad}l}
    r-2 & r>2 \\
    0   & r\le2
    \end{array} \right.
    +
    \left\{\begin{array}{r@{\quad:\quad}l}
    l-37 & l>37 \\
    0   & l\le37
    \end{array} \right.
    $
}

%183 & invokespecial & mc &  74+3r+b\footnotemark[182] \\
%184 & invokestatic & mc &  74+3r+b\footnotemark[182] \\
\footnotetext[183]{The exact value is:
    $
    74+r+
    \left\{\begin{array}{r@{\quad:\quad}l}
    r-3 & r>3 \\
    0   & r\le3
    \end{array} \right.
    +
    \left\{\begin{array}{r@{\quad:\quad}l}
    r-2 & r>2 \\
    0   & r\le2
    \end{array} \right.
    +
    \left\{\begin{array}{r@{\quad:\quad}l}
    l-37 & l>37 \\
    0   & l\le37
    \end{array} \right.
    $
}

%185 & invokeinterface & mc &  112+6r+b\footnotemark[182] \\
\footnotetext[185]{The exact value is:
    $
    114+4r+
    \left\{\begin{array}{r@{\quad:\quad}l}
    r-3 & r>3 \\
    0   & r\le3
    \end{array} \right.
    +
    \left\{\begin{array}{r@{\quad:\quad}l}
    r-2 & r>2 \\
    0   & r\le2
    \end{array} \right.
    +
    \left\{\begin{array}{r@{\quad:\quad}l}
    l-37 & l>37 \\
    0   & l\le37
    \end{array} \right.
    $
}



\footnotetext[187]{\codefoot{new} execution time depends to a great
extent on the caching of the corresponding Java method or the memory
transfer time for the method. \codefoot{new} also depends on the
size of the created object as the memory for the object is filled
with zeros -- This will change with the GC}

%188 & newarray & mc & 12+w-7\footnotemark[188] \\
\footnotetext[188]{\codefoot{newarray} execution time depends to a
great extent on the caching of the corresponding Java method or the
memory transfer time for the method. \codefoot{newarray} also
depends on the size of the array as the memory for the object is
filled with zeros -- This will change with the GC}

\footnotetext[209]{The native instructions \codefoot{jopsys\_rd} and
\codefoot{jopsys\_wr} are alias to the \codefoot{jopsys\_rdmem} and
\codefoot{jopsys\_wrmem} instructions just for compatibility to
existing Java code. IO devices are now memory mapped. In the case
for simple IO devices there are no wait states and the exact values
are 4 and 5 cycles respective.}

%14+r+n*(23+w)
\footnotetext[219]{The exact value is
    $14+r+n(23+\left\{\begin{array}{r@{\quad:\quad}l}
    w-8 & w>8 \\
    0   & w\le8
    \end{array} \right. )$.
$n$ is the number of words transferred.}

%14+r+n*(23+w)
\footnotetext[220]{The exact value is
    $14+r+n(23+\left\{\begin{array}{r@{\quad:\quad}l}
    r-10 & r>10 \\
    0   & r\le10
    \end{array} \right. )$.
$n$ is the number of words transferred.}

\footnotetext[240]{\emph{Is the interrupt and the exception still a
bytecode or is it now inserted just as microcode address?}}


\subsection*{Memory Timing}

The external memory timing is defined in the top level VHDL file
(e.g.\ \code{jopcyc.vhd}) with \code{ram\_cnt} for the number of
cycles for a read and write access. At the moment there is no
difference for a read and write access. For the 100MHz JOP with 15ns
SRAMs this access time is two cycles (\code{ram\_cnt}=2, 20ns).
Therefore the wait state $n_{ws}$ is 1 (\code{ram\_cnt-1}).
%
A basic memory read in microcode is as follows:
\begin{verbatim}
    stmra    // start read with address store
    wait     // fill the pipeline with two
    wait     // wait instructions
    ldmrd    // push read result on TOS
\end{verbatim}
%
In this sequence the \emph{last} \code{wait} executes for $1+n_{ws}$
cycles. Therefore the whole read sequence takes $4+n_{ws}$ cycles.
For the example with \code{ram\_cnt}=2 this basic memory read takes
5 cycles.

A memory write in microcode is as follows:
\begin{verbatim}
    stmwa    // store address
    stmwd    // store data and start the write
    wait     // fill the pipeline with wait
    wait     // wait for the memory ready
\end{verbatim}
The last wait again executes for $1+n_{ws}$ cycles and the basic
write takes $4+n_{ws}$ cycles. For the native bytecode \code
{jopsys\_wrmem} an additional \code{nop} instruction for the
\code{nxt} flag is necessary.

The read and write wait states $r_{ws}$ and $w_{ws}$ are:
\begin{equation*}
    r_{ws} = w_{ws} =
    \left\{\begin{array}{r@{\quad:\quad}l}
    ram\_cnt-1 & ram\_cnt>1 \\
    0   & ram\_cnt\le1
    \end{array} \right.
\end{equation*}

In the instruction timing we use $r$ and $w$ instead of $r_{ws}$ and
$w_{ws}$. The wait states can be hidden by other microcode
instructions between \code{stmra/wait} and \code{stmwd/wait}. The
exact value is given in the footnote.

\subsection*{Instruction Timing}

The bytecodes that access memory are indicated by an $r$ for a
memory read and an $w$ for a memory write at the cycles column ($r$
and $w$ are the additional wait states). The wait cycles for the
memory access have to be added to the execution time. These two
values are implementation dependent (clock frequency versus RAM
access time, data bus width); for the Cyclone EP1C6 board with 15ns
SRAMs and 100MHz clock frequency these values are both 1 cycle
(\code{ram\_cnt}-1).

For some bytecodes, part of the memory latency can be hidden by
executing microcode during the memory access. However, these cycles
can only be subtracted when the wait states (\emph{r} or \emph{w})
are larger then 0 cycles. The exact execution time with the
subtraction of the saved cycles is given in the footnote.

\subsubsection*{Cache Load}

For the method cache load the cache wait state $c_{ws}$ is:
\begin{equation*}
    c_{ws} =
    \left\{\begin{array}{r@{\quad:\quad}l}
    r_{ws}-1 & r_{ws}>1 \\
    0   & r_{ws}\le1
    \end{array} \right.
\end{equation*}

On a method invoke or return the bytecode has to be loaded into the
cache on a cache miss. The load time $l$ is:
\[
    l =
    \left\{\begin{array}{r@{\quad:\quad}l}
    6+(n+1)(2+c_{ws}) & \mbox{cache miss} \\
    4   & \mbox{cach hit}
    \end{array} \right.
\]
with $n$ as the length of the method in number of 32-bit words. For
short methods the load time of the method on a cache miss, or part
of it, is hidden by microcode execution. The exact value is given in
the footnote.

% We count the hidden cycles in the same way as for a read or write:
%   the instructions between stbcr and the first wait
%

\subsubsection*{lastore}

% 48+2*r+2*w
\begin{equation*}
    t_{lastore} = 48+2r_{ws}+w_{ws} + \left\{\begin{array}{r@{\quad:\quad}l}
    w_{ws}-3 & w_{ws}>3 \\
    0   & w_{ws}\le3
    \end{array} \right.
\end{equation*}

\subsubsection*{get/putfield/ref/long}

TODO: add different values for 32-bit, 64-bit and reference type.

TODO: add invokesuper - a special version of invokespecial

%\end{document}
