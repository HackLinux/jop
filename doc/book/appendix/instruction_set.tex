%
% a changed copy from the thesis

%\documentclass[a4paper,10pt,twocolumn]{article}
% use option draft to check for typesetting problems.
% similar to original word document

% don't use textheight... with koma-script
\documentclass[%draft,
%    a4paper,12pt, % 'Standard'
    11pt, % use explicit paper format for Java book format
%    b5paper,10pt, % B5 printout
    % use BCOR to compensate for the two-side margin!
%    a4paper,12pt,BCOR37pt,DIV15,
%    a4paper,12pt,BCOR40pt,DIV14,
%    b5paper,10pt,DIV12,
    headinclude, footexclude,
    twoside, % this produces strange margins!
    openright, % for new chapters
    cleardoubleempty,
% normalheadings - smaller chapter or smallheadings - too small
    % abstracton, % the author of koma-script argues against the title
    headsepline,
%    5headlines, % standard is 1.25 -- wirkt net!
    pointlessnumbers,
    ]{scrreprt}

%\usepackage{html}

% headings
\usepackage{scrpage2} % for headers
 \setkomafont{pagehead}{\scshape\small}
 \setkomafont{pagenumber}{\scshape\small}
 \automark[section]{chapter}
 \ohead[]{\pagemark}
 \chead[]{}
 \ihead[]{\headmark}
 \ofoot[]{} \cfoot[]{} \ifoot[]{}

%\tolerance=500 % to avoid lines sticking out into the margin
               % needed for 'high-performance' in Intro - contributions
\emergencystretch=2em
% or tol. to 500 and emerg. to 1em?
% pagebreak was ok with 500 and 1em
\interfootnotelinepenalty=10000


% use BCOR = (paperwidth-textwidth)/4
% A4: 210mm x 297mm
% B5: 176mm x 250mm
% Java book: 185mm x 232mm
% Engblom: 120x188 (without head)
% Java: 127x187 (without head)
% 1pt = 1/72.27 in = 0.351 mm


% for book
% 'Java-format' 526pt x 660pt (Ghostscript)
\setlength{\paperwidth}{185mm} \setlength{\paperheight}{232mm}
% use that BCOR setting with twoside to compensate the margin
\areaset[13.75mm]{130mm}{200mm} % Java book format

% for book to A4 conversion:
% Set the following sizes and export with
% 'Variable Page Size' in gs.
% Print with 'fit to paper' in Acrobat - results in 110%
% and an effective text area of 142x215
%\setlength{\paperwidth}{180mm} \setlength{\paperheight}{232mm}
%\areaset[12.5mm]{130mm}{200mm} % Java book format

% for two page printout in .pdf
%\setlength{\paperwidth}{134mm} \setlength{\paperheight}{206mm}
%\areaset[1mm]{130mm}{200mm} % Java book format


% use 10pt for code instead of 11pt - but I still would prefer Lucida Typewriter
%\newfont{\myttfont}{cmss10 scaled 1000}
%\newfont{\myttbfont}{cmssdc10 scaled 1000}
%
% This IS Lucida Typewriter
%\newfont{\myttfont}{plsr8r scaled 950}
%\newfont{\myttbfont}{plsb8r scaled 950}
%\newfont{\myttifont}{plsro8r scaled 950}
%%\newfont{\mytttextfont}{plsr8r}

% Lucida is perhaps available in the new Tex installation!!!!
% does not really work!!!
\newfont{\myttfont}{hlsrt8r scaled 950}
\newfont{\myttbfont}{hlsbt8r scaled 950}
\newfont{\myttifont}{hlsrot8r scaled 950}

% I used these .ttf for the official Thesis
%..\ttf2pt1 -e -b LucidaTypewriterRegular.ttf plsr8a
%..\ttf2pt1 -e -b LucidaTypewriterBold.ttf plsb8a
%..\ttf2pt1 -e -b LucidaTypewriterOblique.ttf plsro8a
%..\ttf2pt1 -e -b LucidaTypewriterBoldOblique.ttf plsbo8a

%\newcommand{\javatt}{\myttfont}
%\newcommand{\javattb}{\myttbfont}
%\newcommand{\javatti}{\myttifont}
%\newcommand{\javatext}{\myttfont}
%
%\newcommand{\picscale}{0.909}
%\newcommand{\excelwidth}{11cm}

% end book

% for B5
%\newfont{\javatt}{cmss10}
%\newfont{\javattb}{cmssdc10}
%\newcommand{\picscale}{0.833}
%\newcommand{\excelwidth}{10cm}



% for A4
% 12pt A4 scaled from book
%\areaset[17.05mm]{142mm}{219mm}
\newfont{\javatt}{cmss12}
\newfont{\javattb}{cmssdc10 scaled 1200}
% TODO find an italic
\newfont{\javatti}{cmss12}
\newcommand{\javatext}{\javatt}

\newcommand{\picscale}{1}
\newcommand{\excelwidth}{12cm}


% for chapter head without a number
% \renewcommand{\chaptermark}[1]{\def\myleftmark{#1}}
% \ihead{\myleftmark} \chead{} \ohead{{\rightmark}}

\setkomafont{captionlabel}{\sffamily\bfseries}



% Do I need this package?
\usepackage{float}

% is this a correction for the <> problem?
% \usepackage[T1]{fontenc}

\usepackage{pslatex} % -- times instead of computer modern
% pslatex should be replaced by this:
%\usepackage{mathptmx}
%\usepackage[scaled=.90]{helvet}
%\usepackage{courier}
% pslatex does not work with T1 encoding. <> Problem?


\usepackage{latexsym}
\usepackage{graphicx}
\usepackage{amsmath}
\usepackage{longtable}
\usepackage{booktabs}

% I would need Lucida Console!!!
%
%\newfont{\javatt}{pltt12} % lucida teletype, better than normal but with serifs
%\newfont{\javatt}{plss12} % lucida no serifes, but variable spacing
%\newfont{\javatt}{plss10 scaled 1200}
%\newfont{\javattb}{plssdc10 scaled 1200}
% cmss is NOT a tt font....


\usepackage{listings}
\lstset{language=Java,keywordstyle=,
basicstyle=\javatt,emphstyle=\javattb,commentstyle=\javatti,
showstringspaces=false,captionpos=b}

\usepackage{array}
\usepackage{dcolumn}
\newcommand{\cc}[1]{\multicolumn{1}{c}{#1}}
\newcolumntype{d}[1]{D{.}{.}{#1}}

% f�r die Umlaute in der Kurzfassung
% bekomme ich dadurch Probleme???
%\usepackage[ansinew]{inputenc}


\usepackage{capt-of}
\usepackage[colorlinks=true,linkcolor=black,citecolor=black]{hyperref}
%\usepackage{hyperref}

% ----------------------

%\usepackage{makeidx}
%\makeindex


\usepackage{import} % for subimport text and graphics from subdirectory
% does not work with latex2html!


\newcommand{\codefoot}{\textsf}
\newcommand{\code}[1]{{\javatext#1}} % for LaTeX
\newcommand{\cmd}[1]{{\texttt{#1}}}
\newcommand{\dirent}[1]{{\texttt{#1}}}
%\newcommand{\menuitem}[1]{\textsf{\textbf{#1}}}
\newcommand{\menuitem}[1]{\textsf{\textsl{#1}}}

% for flow.tex - part of index helper
\newcommand{\eei}[1]{%
\index{extension!\texttt{#1}}\texttt{#1}}

% JVs et al
%\newcommand{\ea}{et al.\xspace}
\newcommand{\ea}{et al.\ }

%\begin{htmlonly}
%\renewcommand{\code}[1]{{\texttt{#1}}} % for html2LaTeX
%\newcommand{\toprule}{\hline}
%\newcommand{\midrule}{\hline}
%\newcommand{\bottomrule}{\hline}
%\end{htmlonly}

% net wirklich notwendig -- h�ngt von code generierung ab
%\begin{htmlonly}
%\renewcommand{\javatt}{\texttt}
%\renewcommand{\javattb}{\texttt\bfseries}
%\end{htmlonly}

%\code{\hyphenchar\font=-1}

\newcommand{\mycomment}[1]{}

\newcommand{\instr}[6]{
    \begin{table}
        \begin{tabular}{ll}
            \emph{\large\textbf{#1}} & \\
            \\ \\
            \textbf{Operation} & #2 \\ \\
            \textbf{Opcode} & \texttt{#3} \\ \\
            \textbf{Dataflow} & \parbox[t]{9.5cm}{\(#4\)}\\ \\
            \textbf{JVM equivalent} & \parbox[t]{9.5cm}{\code{#5}} \\ \\
            \textbf{Description} & \parbox[t]{9.5cm}{#6}\\
        \end{tabular}
    \end{table}
}


The instruction set of JOP, the so-called microcode, is described in
this appendix. Each instruction consists of a single instruction
word (8 bits) without extra operands and executes in a single
cycle\footnote{The only multicycle instruction is \codefoot{wait}
and depends on the access time of the external memory}.
\tablename~\ref{tab:appendix:hwreg} lists the registers and internal
memory areas that are used in the dataflow description.

\begin{table}[h]
  \centering
  \begin{tabular}{ll}
    \toprule
    Name & Description \\
    \midrule
    A & Top of the stack\\
    B & The element one below the top of stack\\
    stack[] & The stack buffer for the rest of the stack\\
    sp & The stack pointer for the stack buffer\\
    vp & The variable pointer. Points to the first local in
    the stack buffer\\
    ar & Address register for indirect stack access\\
    pc & Microcode program counter\\
    jpc & Program counter for the Java bytecode\\
    opd & 8 bit operand from the bytecode fetch unit\\
    opd$_{16}$ & 16 bit operand from the bytecode fetch unit\\
    memrda & Read address register of the memory subsystem\\
    memwra & Write address register of the memory subsystem\\
    memrdd & Read data register of the memory subsystem\\
    memwrd & Write data register of the memory subsystem\\
    mula, mulb & Operands of the hardware multiplier\\
    mulr & Result register of the hardware multiplier\\
    membcr & Bytecode address and length register of the memory
    subsystem\\
    bcstart & Method start address register in the method cache\\
	memidx & Index register for native field access \\
    \bottomrule
  \end{tabular}
  \caption{JOP hardware registers and memory areas}\label{tab:appendix:hwreg}
\end{table}

\clearpage
\input{appendix/microcode}


%\end{document}
