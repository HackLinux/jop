\documentclass[a4paper,12pt]{scrartcl}
\usepackage{pslatex} % -- times instead of computer modern

\usepackage[colorlinks=true,linkcolor=black,citecolor=black]{hyperref}
\usepackage{booktabs}
\usepackage{graphicx}


\usepackage[latin1]{inputenc}

\newcommand{\code}[1]{{\textsf{#1}}}
\newcommand{\excelwidth}{10cm}


\begin{document}

\title{Change Log for JOP}
%\author{Martin Schoeberl\\ Vienna University of Technology, Austria\\ mschoebe@mail.tuwien.ac.at}
\maketitle \thispagestyle{empty}

\begin{abstract}

Major changes in the JOP project.

\end{abstract}


\section{First Change Log}

\subsubsection*{2005-06-20}

Added GC info field in class structure: ClassInfo.java (JOPizer),
JVM.java (f\_new). Use handle, changed multianewarray to work with
handles.

\subsubsection*{2005-06-25}

Working stop-the-world, two space GC committed to CVS.



\subsubsection*{2005-12-01}

IO devices are memory mapped (SimpCon) - no more stioa, stiod,
ldiod.

\subsubsection*{2005-12-17}

Prototype of periodic scheduled real-time GC for the ISORC paper.

\subsubsection*{2006-01-11}

Start to generate HW exceptions similar to the timer interrupt.
Exception logic is in \code{cnt.vhd} (should perhaps get renamed).
First exception is generated by a stack overflow (SP=0xff).

\subsubsection*{2006-01-12}

Add a new register (\code{ar}) to address the local memory:

\begin{itemize}
    \item New microcode instructions: star, ldmi, stmi
    \item Use it instead of vp on various bytecodes
    \item reduce vp to 7 bits (only the upper half of the local
    memory)
    \item ld and st instruction changed coding, different address
    mux coding -- document in JOP instruction set
    \item iinc is now 8 instead of 11 cycles
\end{itemize}

\subsubsection*{2006-01-20}

Add individual field instructions for long and reference fields
(\emph{special} bytecodes).



\subsubsection*{2006-01-23}

Long instructions in JVM.java added (Peter and Christof jvmhw work).
Replace CP index in field instructions by the field offset.

\subsubsection*{2006-06-15}

Enhanced memory/cache interface: Avoid one state, load one word
less.

\subsubsection*{2006-08-19}

System.arraycopy: Hello World is 11 KB code, 19 KB all. Changes in:
\begin{itemize}
    \item GC: use type info of array + types for ref. and plain
    objects in handle
    \item 20 KB with arraycopy() and additional Exceptions
    \item 21 KB with Java 1.5 corrections (16.9.2006)
\end{itemize}

\subsubsection*{2006-09-16}

Substitute IINC by ILOAD, PUSH const, IADD, and ISTORE in JOPizer to
avoid Java 1.5 compiler issues and improve the performance a little
bit. Java 1.5 compiler can now be used (with -target 1.4).

\subsubsection*{2006-10-08}

Optimize with \url{http://sourceforge.net/projects/proguard/}

\subsubsection*{2006-10-26}

Merged Nelsons JDK port into the source tree. Moved \code{jdk} to
three different directories: \code{jdk\_base}, \code{jdk11}, and
\code{jdk14}. TODO: Check size of Hello World.

\subsubsection*{2006-11-04}

Move mtab pointer and array size to the handle. Small optimization
in array load and store (one cycle). invokevirtual, invokeinterface
and arraylength should be faster.

\subsubsection*{2006-12-30}

Microcode ROM extended to 2K (with arom.vhd again), long array
bytecodes enabled again, additional stack manipulation bytecodes.
Class structure extended by pointer to super class, checkcast,
instanceof.



\subsubsection*{2007-01-11}

Check new TCP/IP stack (TU Graz). Application size:
\begin{itemize}
    \item Hello World 33 KB (!)
    \item ejip.Main 47 KB (old stack)
    \item ejip.Main 49 KB (some changes)
\end{itemize}

20060818: Hello World 19 KB


%Questions:
%
%Warum zusaetzlich Payload und nicht Packet weiterverwenden?
%
%Das Zusammenstellen eines IP Paketes ist sehr aufwendig (z.B.:
%prepareIPPacket())

\subsubsection*{2007-03-17}

Restructure of JOP components: Additional \code{jopcpu} contains
\code{core}, \code{extension}, and \code{mem\_sc}. Simple AMBA slave
interface for SimpCon paper (FPL 2007).

\subsubsection*{2007-04-11}

New version of JavaBenchEmbedded (1.1) with an additional
\emph{Lift} benchmark.

\subsubsection*{2007-04-14}

Original cycmin: 2691 LCs, 88.4 MHz (extension: 62, mem\_sc: 113).
Original iaload timing: 32 + 3r = 35 and iastore timing: 35 + 2r + w
= 38.

Implement xaload and xastore in hardware. New timings: iaload: 7 +
3r = 10, iastore: 9 + 2r + w = 12.

cycmin w array HW: 2906 LCs, 93 MHz (extension: 70, mem\_sc: 304).

\subsubsection*{2007-05-28}

putfield and putstatic for references invoke JVM.java (for GC
write-barriers).

\subsubsection*{2007-08-03}

Add pointer to static primitive fields into the class structure (for
Hardware objects).

\subsubsection*{2007-08-31}

Change stack pointer handling: no wrapping, exception generation at
maxstack-8, use constants for stack size and start, changed start to
64. With 512 stack words LC is 2948 (instead of 2906).

\subsubsection*{2008-02-05}

Added invokesuper bytecode: the case of invoking the method of a
super-class was not handled correctly by invokespecial. JOPizer
replaces invokespecial with invokesuper where appropriate.

\section{Performance Tracker}

Table~\ref{tab:perf} shows the changing performance due to changes
in JOP.

\begin{table}
    \centering

    \begin{tabular}{rrrrrrl}
        \toprule
        Date & Version & Kfl & UdpIp & Lift & LCs & Comment \\
        \midrule
        ??? & 20050509 & 16582 & 6849 & &  & Austrochip \\
        13.6.2005 & & 16599 & 6838 & &  & with direct object pointers \\
        13.6.2005 & & 16384 & 6405 & &  & with handle indirection \\
        14.6.2005 & 20050614 & 16591 & 6838 & & 2711 & without
        handles again \\
        27.6.2005 & 20050620 & 16367 & 6405 & & 2716 & handles again \\
        26.7.2005 & 20050620 & 16367 & 6405 & & 2716 & \\
        31.7.2005 & 20050728 & 16384 & 6405 & & 2714 & \\
        11.8.2005 & 20050728 & 16367 & 6405 & & 2714 & \\
        16.8.2005 & 20050816 & 16253 & 6258 & & 2726 & new .jop format, Null pointer check \\
        & & & & & &in xaload/xastore is moved! \\
        23.11.2005 & 20050827 & 16237 & 6262 & & 2927 & cycmin (w. WB test slaves) \\
        23.11.2005 & 20050827 & 16270 & 6272 & & 2827 & new memory interface (in cycwrk) \\
        & & & & & & avoid 'register packing' -- Quartus crash \\
        1.1.2006 & 20051220 & 16270 & 6262 & & 2680 & new IO interface \\
        11.1.2006 & 20060111 & 16270 & 6262 & & 2691 & HW for exception (stack
        overflow)\\
        12.1.2006 & 20060111 & 16270 & 6262 & & 2703 & with add. SW
        exceptions (div. by zero)\\
        12.1.2006 & 20060112 & 16591 & 6365 & & 2700 & add ar, iinc
        is now 8 cycles\\
        22.1.2006 & 20060122 & 16591 & 6365 & & 2717 & additional field instructions \\
        23.1.2006 & 20060123 & 16591 & 6522 & & 2722 & offset in field instructions \\
        15.6.2006 & 20060615 & 16633 & 6537 & & 2722 & less cycles
        on cache load\\
        16.9.2006 & 20060615 & 17075 & 6692 & & 2722 & substitute
        IINC\\
        8.10.2006 & 20060615 & 17093 & 6809 & & 2776 & optimized
        with ProGuard\\
        4.11.2006 & 20061104 & 17111 & 6775 & & 2778 & changed
        object layout\\
        4.11.2006 & 20061104 & 17138 & 6889 & & 2778 & changed
        object layout + ProGuard\\
        30.12.2006 & 20061230 & 17111 & 6775 & & 2687 & 2K ROM (Q 6.1)\\
        11.04.2007 & 20070317 & 17120 & 6781 & 13574 & 2691 & no real
        change (Q 7.0)\\
        14.04.2007 & 20070414 & 18347 & 8520 & 16425 & 2906 & array HW \\
        31.08.2007 & 20070831 & 18347 & 8371 & 16425 & 2930 & use more stack \\
        04.12.2007 & 20071203 & 16347 & 8308 & 16425 & 3091 & added interrupt controller \\
        \bottomrule

    \end{tabular}
    \caption{JOP performance with 100MHz, 4KB/16 cache, cycmin}
    \label{tab:perf}

\end{table}

\section{Changelog}

\subsection{Hardware}

\paragraph{20070909} sc\_ram16.vhd violated SimpCon rule to hold data in the
register till new data arrives (used data input register to store
first 16-bit value). Results in an error in the array bounds check.
Added another register in sc\_ram16.vhd -- leads to on input MUX for
RAM data register. Therefore, relax tsu constraint for the SRAM.

\paragraph{20071012} Port to digilent nexys2 board: Added top-level
jop\_nexys2.vhd, nexys2 project in xilinx folder. Quick hack uses
only 4~MB of the 16~MB PSDRAM with a lot of wait states.

\paragraph{20071121} more space in stack.vhd for the stack
trace. Use 33 bit for the comparison (compare bug for diff $>$
2**31 in stack.vhd corrected).

\subsubsection{Error with YAFFS code} is probably an issue with
the cache. Run EraseNAND and TestWithNAND to see the error.

\paragraph{20071122} Christof checked in the work on JOP CMP
(jvm.asm, jop\_types.vhd, cmpsync.vhd, sc\_sys.vhd,
jopmul.vhd).

\paragraph{20071202} Additional signal from \code{bcfetch} to
\code{sc\_sys} (interrupt acknowledgement). Move most interrupt
logic to \code{sc\_sys}. Interrupt ack from \code{jfetch},
interrupt disable on handling.

\paragraph{20071203} prioritized interrupt processing, affects
RtThread scheduling.

\paragraph{20071222} Correction of data MUX bug in \code{extension.vhd} for array read
access -- was set to IO after an IO access, array read did not set
it.

\paragraph{20080218} Faster multiplier: 15\% larger, but twice as
fast; \code{imul} now takes 19 cycles.

\subsection{Software}

\paragraph{20070911} Changes in ejip for TCP addition and some
cleanup in packet manipulation. TCP connection works for a very
simple Telnet server.

\paragraph{20071002} Add changes for the debugger (in JOPizer, ClassInfo, MethodInfo).
Correct wrong Makefile (Wolfgang) and add debugger changes.

Removed build.bat/xml and jcc.jar in tools directory.

Added debugger code in tools directory (com.jopdesign.debug,
com.sun), added debug test kernel in target/src/test.

\paragraph{20071006} Changed JOPizer to generate a correct static
class initializer order.

\paragraph{20071008} Correction of clinit error: Long methods have
the length set to 0, but got invoked instead of interpreted (len<256
corrected to len<256 \&\& len!=0).

\paragraph{20071115} Null pointer exception in ClinitOrder
(from test/testrt/Periodic) corrected.

\paragraph{20071121} stack tracing enabled, better trace output.

\paragraph{20080113} Moving IO-device models from JopSim into their
own classes. Can also be loaded dynamically (use the \code{ioclass}
property with the class name, e.g., \code{-Dioclass="IOSimWD"}).

\paragraph{20080115} GC bug fix by Paulo. GC did not use the
constants for stack start and size for the root scanning in
stop-the-world mode.

\subsection{Documentation}

\paragraph{20080108} corrected timing for \code{xastore}. It's $10 +
2r + w$.

\section{Ant}

\begin{itemize}
    \item Download ant (about 30 MB unzipped)
    \item add ant/bin to your path
\end{itemize}

\section{Possible Enhancements}

\subsection{VHDL}

\begin{itemize}
    \item tristate handling inside the memory component
    \item try to minimize the top-level -- more connections into the
    core
    \item rethink SC address decoding: more flexible for IO devices,
    additional SC components such as NoC memory
    \item check \code{scio\_rd <= sc\_rd;} in \code{extension.vhd}.
    Shouldn't it by \code{scio\_rd <= sc\_rd and not ain(31);}?
\end{itemize}

check \url{http://lipforge.ens-lyon.fr/www/divgen/} for division (a
divisor generator).

\section{Issues, Questions}

\begin{itemize}
    \item Flag \emph{opd} has to be immediately before usage by ld\_opd.
    Why? (see instruction getfield).
    \item Do we really need the full class info for interfaces? Why?
\end{itemize}

\section{JDK based on phoneME change}

\subsection{System dependent classes}

\begin{itemize}
  \item System
  \item Object
  \item String ??
  \item special IO classes (JOP-streams)
  \item Class (we don't have one)
\end{itemize}

\subsection{Take care}

\begin{itemize}
  \item Our current StringBuffer contains a StringBuffer append
      (since JDK 1.4)
  \item How to avoid the big time zone thing?
  \item How to avoid all the encoding stuff?
  \item find all native methods -- check on Jopizer
  \item Wouldn't a JDK 1.4 -- 1.6 Classpath or Sun's JDK be a
      better base?
\end{itemize}

\subsection{Ideas}

\begin{itemize}
  \item What about moving com.jopdesign.sys stuff into java.lang?
  \item Can we get rid of some public methods/classes from the
      c.j.sys package?
\end{itemize}



\section{Optimization}

see \url{http://www.arm.com/pdfs/JazelleWhitePaper.pdf}

Test with \url{http://www.cs.purdue.edu/s3/projects/bloat/}:

\begin{verbatim}
    -classpath /usr/cpu/jop/java/target/dist/classes -verbose
    -stack-alloc jbe.kfl.Mast jbe.kfl.Msg jbe.kfl.Timer jbe.kfl.JopSys
    jbe.kfl.Triac /usr/cpu/jop/java/target/dist/new_classes

   with the main class: EDU.purdue.cs.bloat.optimize.Main
\end{verbatim}

\subsection{JMM}

check \url{http://g.oswego.edu/dl/jmm/cookbook.html}

\section{Thesis Errata}

In the thesis you state jop can handle "16 different immediate
values ..." (pg 68). However ldi microcode can address 32 constants.

There are 32 JVM local variables and 32 constants. In the Thesis the
numbers are 16 and 16.

Stack figure.

\section{CLDC TODO}

\begin{itemize}
    \item Object.clone()
    \item add a Duke image from \url{https://duke.dev.java.net/}
    \item
\end{itemize}



\end{document}
